%% This is ConTeXt
%% http://wiki.contextgarden.net/Main_Page
%% Compiling:
%$ context manual.tex

%Документация:
% ConTEXt the manual https://ctex.org/documents/context/cont-enp.pdf
% краткое руководство https://github.com/contextgarden/not-so-short-introduction-to-context


\enableregime[utf]
\mainlanguage[ru]
\language[ru]

%Рабочее название: Электроообереги. Инженерные амулеты защищающие ваш дом от беды
\def\homepage{https://serkov.su/blog/}
\def\manualver{1.0}

\definepapersize[RIDERO-A5][width=145mm,height=205mm]
\setuppapersize[RIDERO-A5]
\setuplayout [width=middle]
%convert to grayscale for RIDERO: gs -sOutputFile=manual_bw.pdf -sDEVICE=pdfwrite -sColorConversionStrategy=Gray -dProcessColorModel=/DeviceGray -dCompatibilityLevel=1.4 -dNOPAUSE -dAutoRotatePages=/None -dBATCH manual.pdf

\setuppagenumbering [location=footer]
%\setuppagenumbering [location=footer, alternative=doublesided]

%Стиль заголовков разделов
\setuphead[subject, subsubject][style={\bf}]		%Заголовки жирным
\setuphead[subject][style={\switchtobodyfont[14.4pt]}]	%Подзаголовки крупнее
\setuphead[subject][incrementnumber=yes, number=no, page=no, continue=yes]	%Считать номера=да, показывать номер=да, начинать новую страницу=да, продолжать главу=нет (последний мараметр если дефолтный то page=yes игнорится)

\setupcombinedlist[content][list={chapter, title, section, subsection, subsubsection}] %список того что попадает в список содержания


%стиль подписей к картинкам и таблицам
\setupcaptions [style=small, way=bytext, prefixsegments=none]

\definereferenceformat	%ссылки на картинки
   [infig]

\definedescription		%определения терминов
   [param]
   [alternative=serried,
	width=fit,
	margin=yes,
%	before={\startnarrower},
%	after={\stoptnarrower},
	headcommand={\bi}]

% for the document info/catalog (reported by 'pdfinfo', for example)
\setupinteraction[state=start,  % make hyperlinks active, etc.
  title={Электрообереги},
  subtitle={Что и как в вашем электрощите спасает вас от погибели},
  author={Серков Павел},
  keyword={Электробезопасность, автоматика}]

% Документация: http://www.pragma-ade.com/general/manuals/units-mkiv.pdf
\setupunit[language=russ]
\setupprefixtext[russ][femto=ф, pico=п, nano=н, micro=мк, milli=м, 
				 centi=с, deci=д, deca=да, hecto=г, kilo=к, 
				 mega=М, giga=Г, tera=Т, peta=П
				]

\setupunittext[russ][meter=м, second=с, gram=г, 
			   ampere=A, volt=В, ohm=Ом, watt=Вт, joule=Дж, hour=ч, hertz= Гц 
				]

%Оформление абзацев. Расстояие меж параграфов
\setupwhitespace [small]

%Абзацный отступ
\setupindenting[yes, first, medium]
\setupheads[indentnext=yes] 

%URL
\setupinteraction[state=start,style=]
\define[1]\hyperlink{\goto{\hyphenatedurl{#1}}[url(#1)]}
%Использовать \hyperlink{урл}
%для отслеживания
%\goto{\hyphenatedurl{https://serkov.su}}[url(https://serkov.su/blog/?utm_source=ebook&utm_medium=organic&utm_campaign=serkov)].

%Шаманство с версткой

%\brokenpenalty        100     %This penalty is added after each line that ends with a hyphenated word. High values will discourage TEX in breaking a page there.
%\hyphenpenalty        50      %This penalty is added after each location in the paragraph where TEX can hyphenate and therefore this penalty determines the way TEX splits the paragraph into lines.
%\doublehyphendemerits 10000   %While TEX is breaking a paragraph into lines, it calculates demerits for potential linebreaks. This value is added to the demerits of a line if that line as well as the previous one both end with a hyphen.
%\finalhyphendemerits  5000    %When the pre-last line ends with a hyphen, TEX adds this value to the demerits of that line, thereby discouraging a line break at that point when the paragraph is split into lines.
%\widowpenalty         0       %This is the penalty added before the last line of a paragraph. This penalty determines how eager TEX will be in splitting a page before the last line.
%\clubpenalty          0       %This is the penalty added after the first line of a paragraph. This penalty determines TEX’s willingness to split a page after the first line.
%\adjdemerits          10000   %In the process of breaking a paragraph into lines, TEX tags each of the lines as very loose, loose, decent or tight. If two lines are tagged differently, TEX qualifies them as being visually incompatible. In that case the value of thisvariable is added to the demerits of the lines.


\define\attention						%значок важного момента. применять \attention{}
    {\ininnermargin[line=1, align=outer]{\externalfigure[attention.svg][width=0.7cm]}}


\setupbodyfont[computer-modern-unicode,10pt,rm]

\setupexternalfigures[directory={images_LQ}]


%\definequote[advertisment]
\setupheader
  [text]
  [before={\startframed[frame=off,bottomframe=on,framecolor=black,]},
   after={\stopframed},
  ]
%\setupheadertexts[{\framed[background=color,backgroundcolor=red] {ЭТО ЧЕРНОВИК!!! ЭТО ЧЕРНОВИК!!!}}]
\setupheadertexts[{\it{\getmarking[chapter]}}]


%====================Книга начинается с этого места===============================

\starttext
\setuppagenumber [state=stop]


% Обложка. Не должно быть нумерации на этой странице
\startTEXpage
    \externalfigure[cover.svg][width=\paperwidth, height=\paperheight]
 \stopTEXpage
 \startTEXpage
    \externalfigure[cover_ridero.svg][width=\paperwidth, height=\paperheight]
 \stopTEXpage
 \startTEXpage
    \externalfigure[Legal.svg][width=\paperwidth, height=\paperheight]
 \stopTEXpage



\page[yes]
\setuppagenumbering [location=footer, alternative=doublesided]
\setuppagenumber [state=start]

%Требование ридеро - если не вставлять технические страницы то нумерация с 3х
\setcounter[userpage][3]

Версия этой книги \bold{\manualver} от \currentdate . %\date[d=15,m=5,y=2018][day, month,year].

Книга распространяется свободно, проверить и скачать новые версии книги можно на страничке в блоге автора: 
\goto{https://serkov.su}[url(https://serkov.su/blog/?utm_source=ebook&utm_medium=organic&utm_campaign=serkov)].

%\framed[background=color,backgroundcolor=red] {ЭТО ЧЕРНОВИК!!! ЭТО ЧЕРНОВИК!!! НЕДОДЕЛАН!}



%Данный документ сгенерирован: \currentdate

% visual debug
%\showlayout
%\showlayoutcomponents
%\showsetups
%показывает настройки бумаги
%\showmakeup
%\showframe
%\showgrid
%\showstruts
%http://wiki.contextgarden.net/Visual_Debugging

\page[yes]



% Генерация оглавления
\completecontent


%\framed[background=color,backgroundcolor=red] {картинку график нарисовть}

%TODO
%		Добавить про изменение цвета чековых наклеек
%		Добавить ссылку на стандарты расцептелей типа K Z https://studyelectrical.com/2014/07/miniature-circuit-breakers-mcb-types-characteristic-curves.html
%
%    Type B, C, and D are used for overcurrent protection of cables in accordance with IEC/EN 60898-1
%    Type K for the protecting motors and transformers and simultaneous overcurrent protection of cables with overload tripping based on IEC/EN 60947-2
%    Type Z for control circuits with high impedances, voltage converter circuits, and semi cable protection and simultaneous overcurrent protection of cables with overload tripping based on IEC/EN 60947-2.

%https://info.e-t-a.com/ES-2017-02-IndustrialAutomation-DCurve_LP-Destination.html
%https://library.e.abb.com/public/114371fcc8e0456096db42d614bead67/2CDC400002D0201_view.pdf
%	добавить обозначение на схемах
% добавить про многополюсность автоматов




\startchapter[title={Введение, для кого и зачем эта книга}]
	Вас смущали страшные аббревиатуры в речи электриков, такие как 
	АВ, ВДТ, УЗО, УЗДП, УЗИП? Или электрощитки в небольшую квартиру, в которых 
	почему-то очень много разных модулей непонятного назначения? Эта книжка 
	призвана максимально просто и полно, рассказать 
	про устройства защиты, как они работают и, главное, зачем они были придуманы. 
	В книге есть рассказ как о предохранителях, используемых уже более века, так и о 
	новейших устройствах вроде устройств защиты от дугового пробоя, про которые  
	не каждый электрик слышал. Материал частично уже публиковался у меня в блоге 
	на сайте www.serkov.su (а так же на ютубе, телеграме, пикабу и различных соцсетях, 
	все ссылки есть на моем сайте), но я решил собрать его и 
	доработать в аккуратную и полезную книжку, которую можно не только в руках подержать, 
	но отправить или подарить хорошему человеку.

 
	Мощной мотивацией к написанию книги были различные "электрики", позиционирующие 
	огромные электрощиты в небольшие квартиры с огромным количеством автоматических 
	выключателей как «современные». Для неискушённого заказчика они кажутся солидными, 
	ведь вместо трёх автоматических выключателей теперь на вводе в квартиру стоит щит 
	с тремя десятками устройств, чуть ли не по одному на каждую розетку! Конечно, за 
	такое внушительное изделие не стыдно и запросить солидную цену. Но специалисты, 
	присмотревшись к установленным в щите устройствам, лишь ухмыльнутся, и вы вскоре 
	поймёте почему.

	Автор хоть и является дипломированным инженером АСУ ТП, но может ошибаться. 
	Поэтому хочу выразить благодарность Павлу Компавлову, Артищеву Евгению, Виктору Буракову,
	Сергею Долину, выступивших в роли рецензентов, встав заслоном от возможных фактических 
	ошибок в тексте. Отдельное спасибо Наталье Долиной и Олегу Уржумцеву за помощь в редактуре текста.
    Также хотелось бы поблагодарить отечественные  компании IEK, 
	Меандр, EKF, Термоэлектрика и завод Электроавтомат из славного города Алатырь, 
	предоставивших свою продукцию по моей просьбе на растерзание. Хочется также выразить 
	благодарность за терпение моей семье, книга отнимала мое время прежде всего 
	у родных и близких.

	
	Эту книгу я намеренно выпускаю бесплатной, под свободной лицензией Creative
	Commons BY-NC-SA, поскольку считаю важным распространять знания свободно, 
	\ininnermargin[line=1, align=outer]{\externalfigure[Lenin_profile_view_(stylised).svg][width=.7cm]}
	вне зависимости от денежного состояния искателя знаний. В {\bf некоммерческих целях} книгу можно распространять
	свободно, пересылать друзьям, коллегам, выкладывать в группах и каналах. 
	Я буду благодарен оценке моей работы 
	и вкладу в мои будущие проекты в виде пожертвования. Для этого можно купить 
	бумажную версию книги на Ridero,
	или воспользоваться формой пожертвования на моем сайте \goto{https://serkov.su}[url(https://serkov.su/blog/?utm_source=ebook&utm_medium=organic&utm_campaign=serkov)].

			

\stopchapter

\startchapter[title={Сотня лет прогресса в деле электробезопасности}]
	Мир становится сложнее благодаря попыткам сделать его безопаснее. Простые 
	автомобили, в 1920 году имевшие из систем безопасности разве что только тормоз, 
	сегодня стали довольно сложными механизмами. Подушки безопасности, 
	ремни безопасности, антиблокировочная система, система стабилизации 
	курсовой устойчивости, зоны деформации... Стараниями десятков тысяч инженеров на 
	протяжении века автомобили стали гораздо безопаснее и в случае ДТП позволяют 
	выжить, отделавшись звоном в ушах и мелкими травмами. Разница в безопасности 
	особенно заметна при столкновении двух автомобилей разных поколений. 
	Некоторые лаборатории, проводящие краш-тесты, иногда устраивают в рекламных 
	целях такие столкновения, используя модели, выпущенные с разницей в 30--50 лет. 
	Эти видео несложно найти  в интернете, столкновение в замедленной съемке очень наглядно 
	показывает, насколько технический прогресс, и реализованные за время между выходом этих моделей стандарты и решения, повышают шансы выжить.
	\footnote{например \hyperlink{https://www.youtube.com/watch?v=ePYO0-Ig0VU}}


	Такой же прогресс, менее заметный глазу, произошёл и в электрохозяйстве. 
	Если в 1920 году в электрощите из устройств защиты были только плавкие вставки, 
	то сегодня некоторые щитки, из-за количества оборудования, стали  похожи на пульт 
	космического корабля. Инженеры придумали множество разных устройств, 
	которые выявляют аномальные режимы работы электроцепей, отключая их до того, 
	как зародится пожар или разовьётся иная неисправность. К тому же, удешевление устройств за счёт технологического 
	развития позволило некоторым устройствам защиты (например, устройствам 
	защиты от дугового пробоя) спуститься с небес промышленной энергетики 
	в щитки бытового назначения, став доступными не только энтузиастам электробезопасности, но и стать нормативно обязательным устройством защиты.

	Наглядно плоды стараний инженеров видны на рисунке~\infig[electrocutions] -- это статистика 
	США по количеству погибших от удара электрическим током в быту. Чёткое снижение 
	количества таких смертей -- результат распространения выключателей 
	дифференциального тока, они обнаруживают, что человек коснутся проводников 
	и отключают ток до того, как он нанесёт тяжёлый вред организму.

	\placefigure[here][electrocutions]{Графики количества домов, оснащённых УЗО 
		и количество поражений электрическим током. График взят из журнала 
		международной ассоциации электрических инспекторов IAEI Magazine.\footnote{ \hyperlink{https://iaeimagazine.org/electrical-safety/ground-fault-circuit-interrupters-preventing-electrocution-since-1971/}}
			}{\externalfigure[Electrocutions_graph.svg][width=\textwidth]}


	%https://www.cpsc.gov/Research—Statistics/Electrocutions1
%https://iaeimagazine.org/electrical-safety/ground-fault-circuit-interrupters-preventing-electrocution-since-1971/
	%https://www.verobeachelectrical.com/2017/07/08/purpose-and-history-of-electrical-service-panels-deaths-and-injury-due-to-electrical-fires-and-accidents/
	
	Государства в заботе о своих гражданах вводят правила и нормативы в том 
	числе по обязательному применению некоторых видов устройств защиты, 
	причём эти правила почти всегда идут по пути ужесточения. Поэтому в новых 
	домах электрические щиты становятся больше и сложнее -- к обычным 
	автоматическим выключателям добавляются выключатели дифференциального тока, 
	устройства защиты от дугового пробоя и т. д.

	Не скрою, имеют место и гримасы капитализма, когда коммерческие предприятия 
	в погоне за прибылью преувеличивают серьёзность и опасность проблемы, 
	нагнетая страх, а затем получают прибыль, героически продавая устройство, решающее поднятую проблему. 
	Особенно циничные дельцы, используя коррупционные связи, продавливают 
	через органы власти принятие стандартов, требующих покупку и установку 
	устройств в интересах определённого (чьего надо) производителя. 
	Поэтому при появлении новых устройств и технологий в среде специалистов 
	не утихают споры о необходимости внедрения этих устройств, особенно когда 
	специалист материально заинтересован во внедрении альтернативного решения 
	от своего производителя. Если вы наблюдаете за подобными дискуссиями -- не 
	теряйте критического мышления, не верьте эмоциональным красивым демонстрациям, 
	а вникайте в технические детали. 

	Главы книги, посвящённые различным устройствам защиты, построены  
	от простого к сложному, чтобы читатель мог проследить, как 
	развивалась техническая мысль разработчиков, и как конструкция устройств 
	стала такой, какая она есть, какие технические проблемы решает тот или иной узел.
	
	Места, где по мнению автора выражена важная мысль, помечены 
	значком восклицательного знака на полях. \attention{}
\stopchapter


\startchapter[title={От чего защищаемся?}]
	Наверное, с этого вопроса и стоило начать книгу. С определённой долей натяжки все 
	устройства защиты в электрощите можно разделить на две большие категории.

	{\bi Первая категория} устройств защищает от выделения лишнего тепла в электрической 
	цепи в процессе работы. Особенно в тех местах, где нагрев изначально не планировался.

	Так, {\bf автоматический выключатель} и {\bf плавкая вставка} отключит цепь при 
	перегрузке по току. Такое бывает при коротких замыканиях или если нагрузка 
	в сети свыше расчётной. От этого греются провода и разрушается изоляция.

	{\bf Выключатель дифференциального тока} (оно же устройство защитного 
	отключения — УЗО) отключит цепь, если ток находит путь в землю в обход 
	цепи. Иногда причиной этому может быть повреждение изоляции, что вызывает 
	локальный нагрев в месте утечки, с обугливанием изоляции и воспламенением окружения.

	{\bf Устройство защиты от дугового пробоя} отключит цепь, если обнаружит 
	характерные для дугового пробоя искажения тока в цепи. Если где-то плохо 
	зажат провод в клемме или переломана жила провода, то из-за искрения в 
	этом месте и плохого контакта будет нагрев, который может перерасти в пожар.

	{\bf Термоиндикаторные наклейки}, вместе с {\bf газовыделяющими наклейками} позволяют 
	вовремя обнаружить плохой контакт, который начал греться, но пока не до фатальных температур.

	Наглядно все эти устройства я попытался изобразить на рисунке~\infig[protectiontypes].

	\placefigure[here][protectiontypes]{Причины нагрева в цепи и устройства защиты
			}{\externalfigure[Protection_types.svg][width=\textwidth]}
	

	{\bi Вторая категория} устройств защищает от некондиционного электричества, 
	когда показатели напряжения (последовательности фаз и т. д.) выходят за 
	допустимые пределы и представляют опасность для электроприборов.

	{\bf Реле контроля напряжения} защитит, если напряжение будет слишком низким или 
	слишком высоким, например при обрыве нуля\footnote{"Нуль"-жаргонное название нейтрального проводника N}. 
	Это устройство, которое окупается за 1/100 секунды, спасая бытовую технику от поломки.

	{\bf Устройство защиты от импульсных перенапряжений} защитит приборы в цепи, 
	если по линии электропередач прилетит, наведённый молнией, импульс 
	высокого напряжения. Такая «иголка» малой длительности, но высокого 
	напряжения разрушает изоляцию, к таким помехам наиболее уязвима полупроводниковая электроника.

	Особенно стоит выделить выключатели дифференциального тока, они же УЗО. 
	Помимо противопожарной функции, это единственные устройства в электрощите, 
	которые защищают не только электроооборудование, но и непосредственно человека 
	от удара электрическим током. 

	Кроме того, электрохозяйство, выполненное в строгом соответствии с 
	нормативными документами (ПУЭ, СНиП, ГОСТ и т. д.) обладает свойством 
	пассивной безопасности, когда авария оказывается локализована и не 
	приводит к тяжёлым последствиям, так как на её пути оказываются 
	заслоны из нераспространяющей горения изоляции проводов, стальных труб, 
	негорючих корпусов щитов и так далее. Хотя вопросы правильного 
	электромонтажа в данную книгу не входят, но они влияют на итоговую 
	безопасность электрохозяйства не меньше, чем умные устройства установленные в щит. 
\stopchapter

\startchapter[title={Предохранители}]

	Плавкие вставки, они же предохранители -- самое старое из устройств защиты 
	электроцепей. В основе их народная мудрость «где тонко -- там и рвётся». 
	Это намеренно созданное слабое место в цепи, где сечение проводника уменьшено, 
	чтобы при коротком замыкании или перегрузке сгорала специально предназначенная 
	для этого плавкая вставка в щите, а не случайное место в цепи.

	Предохранители, я уверен, видели все, даже не будучи связанными с электрикой. 
	Это тонкий проводник-волосок в стеклянной трубочке. Предохранители на 
	большие токи могут иметь форму прямоугольного паралелепипеда с ножевыми 
	контактами. Но неизменно одно -- наличие внутри проводника, который должен 
	расплавиться при превышении тока.

%	\framed[background=color,backgroundcolor=red] {Фоточку сделать c внешним видом цилиндрического предохранителя}

	Но не всё так просто, и что бы это продемонстрировать, поставим себя на место 
	производителя. Представим, что мы решили производить и продавать предохранители.
	Купив тонкой медной проволоки разных диаметров, стеклянной трубки и колпачков, 
	нарежем трубочки, вставим проволочку, закроем колпачками, упакуем в коробки 
	и напишем в рекламном буклете:

	\startnarrower \it
	Новейшее средство защиты ваших электрических цепей от токов короткого 
	замыкания и от перегрузки! С нашими инновационными предохранителями ваши 
	электроустановки не будут загораться при коротком замыкании! 
	Быстрые, качественные, недорогие! Всего по 15,99 руб. \tf
	\stopnarrower

	И здесь встаёт первый вопрос -- а какой ток на них написать? Первое  желание 
	написать на них ток, при котором они будут отключаться. Например, написать 
	\unit{10 ampere}, тогда понятно, что при токе в \unit{9,99 ampere} предохранитель 
	пропускает ток, а при токе \unit{10 Аmpere} р-р-раз и перегорел. Но, увы, это невозможно, 
	потому что мы живём в неидеальном мире, где поставщик не может 
	обеспечить номинальный диаметр проволоки на всём протяжении с точностью в пределах \pm \unit{0,01 mm}. 
	У потребителя температура тоже разная, на стенде при комнатной температуре 
	предохранитель срабатывает при \unit{10 ampere}, а на морозе, например, вообще при 
	\unit{13 ampere}. Чтобы не оказаться в глупом положении, напишем на корпусе 
	номинальный ток:

	\param{номинальный ток плавкой вставки} --- значение тока, который плавкая 
	вставка может длительно проводить в установленных условиях без повреждений. 
	(ГОСТ Р 50339.0-2003)

	Получилось удобно. Знаешь максимально допустимую нагрузку -- такой 
	предохранитель и ставь, при коротком замыкании сгорит точно. А вот если 
	хочется знать точную величину тока, при которой сгорит -- печатаем в 
	документации кучу таблиц поправочных коэффициентов: учёта температуры воздуха,  
	скорости воздушного потока, температуры контактов и даже высоты над уровнем 
	моря -- разреженный воздух хуже отводит тепло.  Но самое важное -- выразим  
	 графиком зависимость времени сработки предохранителя от протекающего 
	через него тока, изображённый на рисунке~\infig[FuseCurve], который назовём 
	времятоковой характеристикой.

	\placefigure[here][FuseCurve]{Времятоковые характеристики предохранителей серии ПНБ5. 
	\footnote{График воспроизведен из книги Намитоков К.К. Плавкие предохранители. М.: Энергия 1979. Страница 13}
			}{\externalfigure[Fuse_curves.svg][width=.9\textwidth]}

	Из графика очевидно -- чем сильнее превышен ток, тем быстрее сгорает 
	предохранитель. Обратите внимание, иногда по горизонтали указывают не 
	абсолютное значение тока в амперах, а кратность превышения номинального тока.
	Графики обычно изображаются в логарифмическом масштабе.

	Предохранители хорошо продаются, и к нам приходят недовольные электрики. 
	Говорят мы замучались с вашими предохранителями, в свете фонарика вынимать 
	по одному в поисках сгоревшего. Специально для удобства электриков делаем 
	флажок-индикатор, который выскакивает, если проволочка перегорела. 
    Эту небольшую доработку будем показывать, как "инновационный предохранитель с функцией индикации 
	сработавшего для удобства потребителя и экономии времени электрика". 
	Радуемся, что появилось выражение "выбило пробки", благодаря такой нехитрой 
	рационализации.

	\placefigure[left][fuseflag]{Промышленный предохранитель на 100А с внутренней конструкцией. 
			Красный флажок выталкивается пружиной при сгорании плавкой вставки и пережигании фиксирующей проволочки. Фото
			пользователя Medvedev с  сайта Wikimedia Commons, CC BY-SA.
			}{\externalfigure[Industrial_NH1_Fuse_100A.jpg][width=0.4\textwidth]}

	Но вот нам, как к производителю, поступают первые серьёзные жалобы: говорят, 
	предохранитель сработал, но разрыв получился маленьким, и его начало иногда 
	пробивать искрой. Получилось нехорошо, вроде как проволочка внутри перегорела, 
	разорвав цепь, но иногда через зазор проскакивает искра, и нагрузка оказывается 
	под напряжением. Почесав затылок, начинаем писать на предохранителе {\bf рабочее 
	напряжение}. Оказалось, что наши предохранители нормально работали при напряжении \unit{220 volt},
	а покупатель запихнул их в цепь защиты высокого напряжения микроволновки, 
	где \unit{2000 volt}. Для таких условий разработаем модельный ряд высоковольтных предохранителей, в которые добавим 
	пружинку -- она гарантированно растащит концы проволочки, когда та перегорит, что обеспечит должный 
	зазор. Фотографию такого предохранителя можно посмотреть на рисунке~\infig[HVfuse].

	\placefigure[here][HVfuse]{Высоковольтные предохранители от микроволновых печей, \unit{750 milliampere} \unit{5 kilo volt} 
			}{\externalfigure[HVfuse.jpg][width=\textwidth]}


	Продав много предохранителей, попиваем  кофе, пока к нам в офис не прибегает 
	злющий электрик. Говорит, что в щитке взорвался наш предохранитель, 
	да так мощно, что осколками чуть не убило. Успокоив электрика, выясняем, 
	что в момент перегорания проволочки в месте разрыва зажигается электрическая 
	дуга, которая горит на воздухе и хорошо проводит электрический ток. А если ток через 
	дугу очень большой -- то сама уже не затухает. Модернизируем предохранитель: 
	корпус предохранителя оказывается должен не только быть хорошим изолятором, 
	но и обязан обладать прочностью, поэтому меняем стекло корпуса на более крепкую керамику. 
	Засыпаем кварцевым песком внутреннее пространство между корпусом и проволочкой --
	оказалось, что это помогает погасить дугу, пока она не разорвала корпус 
	предохранителя. Так мы получаем уже довольно брутальный предохранитель, 
	способный прерывать токи в десятки тысяч ампер. Такой предохранитель изображён на фотографии
	\infig[fusesand], кварцевый песок внутри, увы, не видно, придётся поверить на слово.

	\placefigure[inner][fusesand]{Брутальный предохранитель ППНИ-33 от компании IEK. При номинальном токе 100А
			способен отключать ток до \unit{120000 ampere} при \unit{500 Volt} и до \unit{50000 ampere} при \unit{660 volt}
			}{\externalfigure[fuse_sand.jpg][height=.6\textheight]}

	Чтобы взрывов больше не повторялось, начнём писать на корпусе {\bf отключающую способность}:

	\param{отключающая способность плавкой вставки} --- значение (для переменного 
	тока -- действующее значение симметричной составляющей) ожидаемого тока, 
	который способна отключать плавкая вставка при установленном напряжении в 
	установленных условиях эксплуатации и обслуживания. (ГОСТ Р 50339.0-2003)


	На рисунке~\infig[fuseexplode] показано, что будет, 
	если предохранитель с отключающей способностью на \unit{10000 ampere} заставить 
	разрывать цепь при протекании тока в \unit{50000 ampere}. Выделение энергии электрической дугой
	столь мощное, что происходит взрыв предохранителя.

	\placefigure[page][fuseexplode]{Кадры взрыва корпуса предохранителя от тока 
	короткого замыкания свыше отключающей способности устройства. Иллюстрация 
	из книги BUSSMANN Fuseology, стр 15.
			}{\externalfigure[fuse_explode.jpg][width=\textwidth]}


	Что ещё можно усовершенствовать в конструкции предохранителя? Новые гости с 
	радикально противоположными претензиями помогут. Пришедший электронщик 
	говорит, что наши предохранители говно, потому что слишком медленные, и пока 
	они сработают, у него все полупроводники уже успели догореть. А пришедший 
	энергетик говорит, что наши предохранители говно, потому что слишком быстрые, 
	пока разгоняется двигатель вентилятора, он секунд десять кушает стартовые 
	токи, превышающие номинальные в несколько раз. Вроде превышение, но 
	вынужденное и даже нормальное, если длится недолго. Но предохранитель успевает сгореть.

	Придётся снова менять конструкцию предохранителя, чтобы изменить скорость, 
	с  которой он срабатывает. Для замедления увеличим длину проволочки, тем 
	самым увеличив её тепловую инерцию, да ещё и накрутим её на корд из стекловолокна -- 
	теперь при превышении тока она будет нагреваться медленнее, и ток 
	успеет вернуться в норму прежде, чем она расплавится.

	\placefigure[here][slowfuse]{Предохранители. Слева направо: ускоренный, обычный,
			замедленный и сверхмедленный.
			}{\externalfigure[fuses.jpg][width=\textwidth]}

	Совсем медленный предохранитель делаем так: припаиваем пружинку легкоплавким 
	припоем к нагревателю, в качестве которого будет низкоомный резистор или 
	даже кусок проволоки. Если будет короткое замыкание --  пружинку сразу расплавит током. 
	Если же превышение тока будет небольшим -- капля припоя будет нагреваться от 
	потерь в резисторе, и если пройдёт достаточно времени -- капля расплавится, 
	и пружинка разорвёт цепь.
	
	Для ускорения будем использовать металлургический эффект (в англоязычной 
	литературе просто M-effect, вроде как из-за того что его обнаружил в 1930-е 
	проф. A.W. Metcalf) — на проволочку нанесём каплю олова. Когда из-за 
	протекающего тока проволочка нагреется до температуры плавления олова, 
	жидкое олово начнёт растворять медь, сечение будет уменьшаться (олово 
	проводит ток в разы хуже меди), нагрев усилится, и такая конструкция перегорит 
	быстрее, чем простая проволочка.

	Довольные собой, расширяем каталог, введя дополнительную маркировку по скорости 
	работы предохранителей. В лучших традициях запутывания потребителей,
	у нас будут обозначения по разным стандартам. По ГОСТ Р МЭК 60127-1-2005 
	\footnote{"Миниатюрные плавкие предохранители", введен взамен ГОСТ Р МЭК 127}
	скорость работы предохранителя будем обозначать первые буквы, перед указанием 
	номинального тока, например {\it F1A} или {\it T1.25A}:


	\startitemize[packed]
		\item FF -- Ультрабыстрые, для защиты полупроводниковых приборов
		\item F  -- Быстродействующие, или стандартные.
		\item M  -- С небольшой временной задержкой. 
		\item T  --	С временной задержкой.
		\item TT --	С большой временной задержкой.
	\stopitemize


	По ГОСТ Р МЭК 60269-1-2016 \footnote{"Предохранители плавкие низковольтные"} 
	обозначать тип предохранителя будет двумя буквами, маленькой и большой, 
	вида {\it хХ}.

	Первая буква обозначает диапазон токов, в которых предохранитель отключается. 
	g -- если перегорает при любом превышении, a -- перегорает при большом превышении 
	(т.е. только при коротких замыканиях, но не при перегрузке).

	Вторая буква -- категорию применения, которая учитывает необходимую скорость срабатывания:
	\startitemize[packed]
		\item G	(General purpose) -- общего применения, обычная скорость.
		\item R	(Rectifiers) и S (Semiconductor) -- для использования с полупроводниковыми 
		ключами, очень быстрые. (Отдельный ГОСТ МЭК 60269-4).
		\item M	(Motor) -- для применения с моторами, медленные.
		\item PV (Photovoltaic) -- для солнечных батарей (Отдельный ГОСТ МЭК 60269-6).
		\item N	--  совместим по контактам с используемыми в северной Америке 
		предохранителями стандарта UL 248.
		\item D --	совместим по контактам с используемыми в северной Америке 
		замедленными предохранителями для двигателей UL248.
	\stopitemize

	Небольшое примечание по предохранителям солнечных батарей: защита цепей 
	солнечных батарей -- задача более сложная, чем может показаться на первый 
	взгляд. КПД солнечной батареи максимален при потребляемом токе немногим 
	меньше тока короткого замыкания. А при коротком замыкании солнечная батарея 
	работает как источник тока, \attention{} например, в рабочем режиме ток солнечной батареи 
	\unit{8,5 ampere}, а при коротком замыкании ток будет не более \unit{10 ampere}. 
	Отсюда весьма специфические требования к устройствам защиты для солнечных 
	батарей и регулярные пожары на крышах домов с неправильно собранными солнечными панелями.

	Будучи ответственным производителем, дотошно собираем отзывы клиентов. 
	И вот недовольный отзыв клиентки: жалоба, что когда замкнуло плойку, из-за 
	короткого замыкания сгорел предохранитель в плойке, на вводе в квартиру, в 
	этажном щитке и даже на вводе в здание! Формально случилось короткое замыкание, 
	ток в цепи мгновенно вырос до неприличных значений, и предохранители сгорели. 
	Вздохнув и припомнив проектантов, рассказываем, что это не проблема конструкции 
	предохранителя, а проблема правильного применения, и рассказываем про селективность:

	\param{селективность при сверхтоке} --- координация соответствующих характеристик 
	двух или более устройств для защиты от сверхтоков с таким расчётом, 
	чтобы при появлении сверхтоков в установленных пределах срабатывало устройство, 
	рассчитанное на эти пределы, в то время как другие устройства не срабатывали. 
	(ГОСТ Р 50339.0-2003)

	Если соблюсти селективность, при коротком замыкании будет срабатывать 
	предохранитель, ближайший к короткому замыканию, даже если все предохранители 
	соединены последовательно в одной цепи. То есть, у клиентки сгорел бы предохранитель 
	в плойке. Если короткое замыкание было в розетке -- то сгорел бы предохранитель 
	на вводе в квартиру, а этажный и на вводе в дом, остался бы цел.

	Для соблюдения селективности необходимо, чтобы отношение номинальных токов 
	предохранителей было не менее 1,6 к 1. (При условии, что предохранители одного 
	типа gG, если предохранители разные, например gG и gPV то тут внимательно 
	нужно смотреть документацию). Если вы посмотрите на рисунок~\infig[FuseCurve] с графиком 
	времятоковых характеристик в начале главы, то все сразу станет понятно -- кривые 
	токов плавления предохранителей параллельны и не пересекаются, так что в 
	определённых рамках, если соблюсти отношение 1,6 к 1, селективность будет соблюдена. 
	Если бы в плойке был предохранитель на \unit{6 ampere}, на вводе в квартиру 
	на \unit{16 ampere}, на этаже на \unit{25 ampere}, а на вводе в здание на \unit{40 ampere}, 
	то предохранители срабатывали бы селективно.

	Как хороший и ответственный производитель, мы ведём соцсети для общения с 
	нашими клиентами и замечаем ехидный комментарий. Пользователь Anonymous_troll 
	написал вконтакте, что "ваши предохранители -- устаревшее говно мамонта, 
	есть автоматические выключатели". Успокоившись, пишем ответный комментарий, 
	что пользователь во многом прав, но есть несколько нюансов, благодаря которым 
	предохранители точно не умрут ещё лет сто, как они  сотню лет уже прожили:

	\startitemize[n]
		
		\item Вряд ли можно придумать защиту дешевле, чем плавкие вставки. Особенно 
		разница заметна для больших токов, просто посмотрите сколько стоит предохранитель на 
		номинальный ток \unit{250 ampere} и сколько стоит автоматический выключатель 
		на \unit{250 ampere}.
			
		\item Отключающая способность предохранителя гораздо выше, чем отключающая 
		способность автоматического выключателя сопоставимых габаритов.

		\item Минимально возможная индуктивность. В некоторых цепях это 
		критически важно, особенно с ограничителями импульсных перенапряжений.

	\stopitemize

	Ну и рассказ был бы неполным, если не затронуть самовосстанавливающиеся 
	предохранители. Это весьма специфическая вещь, их можно увидеть на рисунке~\infig[polyfuse].

	\placefigure[left][polyfuse]{Полимерные самовосстанавливающиеся предохранители. 
			Обычно чем он крупнее, тем больший у него номинальный ток.
			}{\externalfigure[polyfuse.jpg][width=0.6\textwidth]}


	Они изготовлены из специального материала, который резко повышает своё 
	электрическое сопротивление при нагреве, почти скачкообразно. Если ток 
	превышает номинальный, предохранитель нагревается и разрывает цепь. 
	Если короткое замыкание ликвидировали, он остынет и снова будет проводить 
	ток, как тепловое реле, но без движущихся частей. Но есть ряд недостатков:

	\startitemize[2, packed]

		\item Рабочее напряжение, как правило, не выше 50--60 вольт.

		\item Когда предохранитель срабатывает -- через него продолжается утечка тока, 
		на порядки меньше номинального тока, но достаточная, чтобы о ней помнить.

		\item Из-за тепловой инерционности они срабатывают очень медленно.

		\item Номинальный ток зависит от температуры среды, в корпусе горячего устройства 
		на жаре могут давать ложные срабатывания.

		\item Ограниченное число срабатываний, прежде чем деградация даст о себе знать. 
		А это десятки, реже сотни срабатываний.

	\stopitemize

	Зачем они такие нужны? Например для защиты USB порта, если в него подключат 
	что-то излишне мощное -- предохранитель не даст сжечь дорожки на плате. 
	Пользователь, уяснив, что устройство не работает, отсоединяет его, 
	предохранитель остывает и порт снова готов к работе. Штука очень нишевая и 
	обычно встречается не в электрощитах, а внутри электронных приборов.


		\startsubject[title={Про ремонт предохранителей.}]
		Стоит сказать несколько слов о ремонте предохранителей. Исторически 
		первые предохранители были бескорпусными и представляли собой просто 
		плавкую проволочку. Реклама такой проволочки из каталога 1894 года приведена на рисунке~
		\infig[fusewire].

		\placefigure[here][fuseboard]{Распределительный щит с плавкими вставками, 1890 год. \footnote{Иллюстрация из журнала Western Electrician за 12 апреля 1890 г.}
			}{\externalfigure[fuseboard 1890.jpg][width=\textwidth]}

		В электроцепи предусматривалось место под их установку. Это могла быть 
		как специальная доска как на гравюре 1890 года на рисунке~\infig[fuseboard], 
		так и специальный зазор в распределительных коробках или внутри корпусов 
		светильников и выключателей.

		\placefigure[left][fusewire]{Катушка проволоки, играющей роль плавкой вставки. \footnote{Иллюстрация из журнала Western Electrician за 6 января 1894 г.}
			}{\externalfigure[fusewire 1894.jpg][width=.4\textwidth]}

		

		Электрик просто носил в кармане катушку калиброванной проволочки и 
		заменял сгоревшую вставку в изделии, если это было нужно. Но от такой 
		конструкции быстро отказались. Во-первых, такая открытая прокладка 
		проволочки небезопасна -- при сгорании плазма может поджечь окружающие 
		предметы. Во-вторых, брызги металла и копоть при сгорании портила изоляционные 
		свойства диэлектрика, и если вставка сгорала очень часто, требовалась 
		трудоёмкая работа по очистке и ремонту поверхности. Поэтому довольно 
		быстро на смену им пришла конструкция, в которой проволочка заключена в 
		трубочку из стекла или фарфора. Идее держателя предохранителя в виде вкручивающейся 
		пробки более сотни лет, на рисунке~\infig[fusethread] реклама такого держателя.

		
		

		Но желание экономии никогда не исчезнет. Поэтому почти сразу производители 
		стали предлагать разборные конструкции предохранителей, где можно заменить 
		плавкую вставку отдельно от всего корпуса. Реклама такого изделия 1917 года на рисунке~\infig[fusereplace]

		\placefigure[left][fusethread]{Реклама предохранителей -- "пробок" 1908 г. \footnote{Иллюстрация из журнала Canadian Electrical News Vol. XVII No.1}
			}{\externalfigure[fusethread.jpg][width=0.5\textwidth]}
		

		Чем же всё кончилось? Не было достаточного спроса на предохранители с 
		возможностью ремонта. В СССР прорабатывался вопрос производства ремкомплектов 
		для замены сгоревшей плавкой вставки в корпусах предохранителей на большие 
		токи (в предохранителе на 500А используется не проволочка, а лента с 
		калиброванной просечкой, сам корпус предохранителя уже не такой дешёвый). 
		Но ремонт предохранителей не получил должной популярности, хотя унификация 
		и  строгая система ГОСТов в Советском Союзе очень сильно упрощала внедрение 
		такой ремонтопригодности. В западном мире ремонт предохранителей 
		тоже не прижился,  этому препятствовали патенты и конкуренция. Поэтому 
		можно с уверенностью сказать, что предохранители -- это одноразовые изделия, 
		штатного способа ремонта которых не предусмотрено.
		
		\placefigure[right][fusereplace]{Реклама ремонтируемых предохранителей 1917 год. \footnote{Иллюстрация из журнала Electrical News от 1 июля 1917 г.}
			}{\externalfigure[renewablefuses.jpg][width=.4\textwidth]}

		Но если очень нужно? К примеру, вы на необитаемом острове вдали от цивилизации, и 
		единственный предохранитель сгорел? (Кстати, вы уже установили и устранили 
		причину, почему он сгорел? Иначе следующий предохранитель повторит его судьбу, а может и усугубит поломку.) Существует практика ремонта предохранителей 
		как мера крайней необходимости, и важно отметить два важных требования для безопасности мероприятия:

		\startitemize[n]

			\item Проволочка должна быть {\bf внутри} корпуса предохранителя. Намотанная 
			снаружи (так называемый «жучок») предохранителя, при плавлении она может поджечь 
			окружающие предметы и покрыть всё проводящим налётом копоти, по которому
			начнётся пробой. Возможно, придётся просверлить тонким сверлом торцы сгоревшего 
			предохранителя, чтобы просунуть и припаять новую проволочку.
			
			\item Проволочка должна быть соразмерна току. Можно вооружиться микрометром, 
			таблицей токов плавления и отремонтировать предохранитель, даже попав в 
			номинальный ток с погрешностью 10-50\%. Но чаще всего проволочку берут 
			«на глаз», и из предохранителя на \unit{0,25 ampere} получается предохранитель 
			на \unit{5 ampere}. Необходимо задать себе вопросы и понять, в какой цепи и для чего стоит 
			предохранитель. Если это просто защитный предохранитель от возгорания в 
			светодиодном светильнике, то при коротком замыкании он всё равно сработает. Но если 
			это предохранитель защищающий трансформатор от перегрузки, то мы рискуем 
			получить перегретый трансформатор, плачущий каплями изоляции. А если это 
			предохранитель gPV для солнечных батарей, то можно получить пожар на 
			крыше, так как они должны срабатывать при небольшом превышении тока и 
			требуют повышенной точности от производителя.
		\stopitemize

		\stopsubject
\stopchapter


\startchapter[title={Автоматические выключатели}]

	Продолжаем играть в производителя электротехники. Для повышения прибылей 
	решаем выпустить новый продукт -- автоматический выключатель, который 
	отключал бы цепь при превышении тока как плавкая вставка, но при этом был бы
	многоразовым и не требовал расходных элементов. Пользуясь капиталистическими 
	правилами рынка, создаём презентацию для потенциальных инвесторов в новый продукт:

	\startnarrower \it
	Новейшие инновационные автоматические выключатели! Защитят вашу проводку от 
	перегрузки и короткого замыкания, отключив напряжение за доли секунды, 
	спасая ваш дом от пожара, а генератор от поломки. Хватит покупать 
	одноразовые предохранители, наши инновационные автоматические выключатели 
	сохранят ваши деньги, ведь сработав, они снова готовы к работе, стоит лишь 
	повторно их включить.\tf
	\stopnarrower

	Технически реализовать такой автоматический выключатель несложно. 
	Ток, протекая по проводнику, создаёт магнитное поле. Чем больше ток -- тем 
	сильнее магнитное поле. Если свернуть проводник в катушку, то мы получим 
	электромагнит, который втягивает в себя стальной якорь с силой, пропорциональной 
	протекающему в цепи току. Если включить этот электромагнит в цепь 
	последовательно с нагрузкой, а якорь подпружинить и соединить с защёлкой, 
	получим простейший автоматический выключатель.\footnote{Если протекающий ток
	небольшой, то якорь в магнитном поле может вибрировать. Поэтому не очень качественные
	автоматические выключатели могут "гудеть" под нагрузкой. Обычно это не влияет
	на работоспособность, но неприятно} Ток срабатывания будем просто 
	регулировать натяжением пружины. Результат изображён на рисунке~\infig[breaker1903].

	\placefigure[here][breaker1903]{Автоматический выключатель. Иллюстрация из книги 
	Ф. Грюнвальдъ. Справочная книжка по электрическому освѣщенiю. 1903 год. Спасибо Антону Кошкину за присланный скан.
			}{\externalfigure[breaker1903.jpg][width=0.6\textwidth]}

	
	Регулировочный винт натяжения пружинки оснастим приблизительной шкалой, 
	и автоматический выключатель готов. Если где-то произошло короткое замыкание, 
	ток резко возрастает, магнитное поле в катушке увеличивается настолько, 
	что втягивает якорь, пересиливая пружинку. Защёлка освобождает рубильник, 
	и он под своим весом или от усилия возвратной пружины разрывает цепь. При этом, 
	механизмом можно пользоваться как обычным неавтоматическим рубильником -- 
	отключать и включать цепь, но это не совсем подходящий для него режим работы.

	Практически сразу нам начнут жаловаться на недостатки, и оба связаны с 
	работой контактной группы. Во-первых, оказалось, что у нерешительных электриков наш 
	выключатель быстро выходит из строя. Если очень плавно включать и выключать 
	наш выключатель, то в месте контактов сильно искрит, и от этого они оплавляются.
	Во-вторых, все проводники мы делали из меди, которая отлично 
	проводит ток, но на воздухе окисляется. Слой окислов плохо проводит ток, 
	возникает нагрев, из-за которого окисный слой формируется ещё быстрее... 
	в общем контакт  греется и плавится.

	Первый недостаток мы поборем, добавив пару пружинок и рычагов. При движении 
	рукоятки контакты будут резко перещёлкиваться, что минимизирует время, когда 
	искра горит и портит контакты. Бонусом сделаем механизм свободного расцепления -- 
	если упорный идиот будет держать рычаг в режиме "вкл", то автоматический 
	выключатель всё равно сработает и разомкнёт цепь. Со вторым недостатком сложнее. 
	Просто покрытие меди чем-то, что не окисляется так сильно (золото, никель, 
	олово и т.д.) поможет буквально на пару включений -- искрение при коммутации 
	(а от него полностью никогда не избавиться) быстро испортит покрытие и оголит медь. 
	Положив глаз на столовое серебро шефа, решаем проблему, сделав небольшие 
	напайки на контакты из серебра. В один выключатель его уходит буквально 
	доли грамма, поэтому цена выросла не сильно, зато контакт оказался гораздо 
	надёжнее. На рисунке~\infig[agrelay] видны серебряные контакты у реле, сделанные с той 
	же целью — надёжная коммутация.

	\placefigure[left][agrelay]{Контакты реле имеют напайки из серебросодержащего сплава.
			}{\externalfigure[agrelay.jpg][width=0.4\textwidth]}

	Получилось прекрасно, автоматические выключатели можно рекламировать и продавать, как замену 
	предохранителям. Для простоты замены, точно так же, напишем на них 
	номинальный ток --- ток, который гарантированно будет проходить через 
	автоматический выключатель, не вызывая отключения. А вот узнать ток, 
	при котором произойдёт отключение, можно будет по графикам в документации.

	По секрету, в личной беседе, вам расскажут, что получившееся изделие отлично 
	работает при коротком замыкании, когда ток резко возрастает до десятков, 
	а то и сотен раз больше номинального. Электромагнит быстро втягивает якорь 
	и всё отключается за доли секунды. А вот при небольших перегрузках, в полтора-два раза превышающих номинальный ток,  механизм 
	работает плохо. Если ток в цепи лишь немного меньше тока, при котором 
	срабатывает автоматический выключатель, электромагниту не хватает сил 
	втянуть якорь до конца, но вполне хватит, чтобы заставить его громко вибрировать. 
	Кроме того, регулировка механизма становится весьма трудоёмкой и капризной, 
	но написать в документации на автоматический выключатель "Номинальный ток \unit{10 ampere} 
	\pm \unit{5 ampere}" запретил рекламный отдел.
	
	Конкурирующая группа решала нашу задачу изобретения автоматического выключателя и пошла по другому пути.
	Вместо электромагнитного расцепителя они использовали тепловой, на базе 
	биметаллической пластинки! Биметаллическая пластинка, это пластинка из двух 
	слоёв разных металлов с разными коэффициентами теплового расширения. 
	Если её нагревать -- она изгибается, причём довольно предсказуемо. Техническая реализация такого автоматического выключателя проста:
	намотаем на биметаллическую пластинку кусочек проволоки с 
	высоким удельным сопротивлением и включим защищаемую цепь через неё. Протекающий 
	в цепи ток будет разогревать проволочку, из-за чего будет нагреваться и изгибаться 
	биметаллическая пластинка.  Если ток в цепи превысит номинальный, от нагрева биметаллическую пластинку изогнет столь сильно, что освободится защелка, размыкающая цепь.
	Механизм весьма компактный и умещается в корпусе обычного выключателя.

	\placefigure[here][thermalbreaker]{Автоматический выключатель на \unit{10 ampere}
			}{\externalfigure[thermalbreaker.jpg][width=\textwidth]}

	На рисунке~\infig[thermalbreaker] изображена начинка такого теплового 
	предохранителя. Вы их часто можете видеть на сетевых фильтрах и удлинителях. 
	Выключение реализовано подпружиненным пластиковым флажком, который запрыгивает 
	между контактов, если биметаллическая пластинка изогнулась и развела контакты, 
	флажок не даст цепи включиться после остывания и возврата пластинки в 
	исходное положение. Для упрощения ток пустили по самой биметаллической пластинке -- 
	сама себя нагревает, сама изгибается.

	Но автоматический выключатель с биметаллической пластинкой обладает недостатком --
	он инерционен и работает медленно, но точно. При небольшой перегрузке, в 1,25--2
	номинальных тока он сработает, возможно даже на это уйдут минуты. Но при коротком замыкании
	выделение тепла слишком интенсивное -- проволочка нагревателя биметаллической пластинки перегорит, автоматический выключатель станет одноразовым, как предохранитель.
	
	Идеальным решением оказалось скомбинировать эти два типа расцепителя! Если в 
	конструкции автоматического выключателя есть два расцепителя -- биметаллическая пластинка 
	и электромагнит, то он хорошо работает и при  небольших (1,5--2~раза) и при 
	больших (5--10~раз) превышениях номинального тока, так как есть расцепитель 
	хорошо работающий при таких диапазонах перегрузок. Для упрощения замены предохранителей "пробок",
	автоматический выключатель можно уместить в прежний габарит предохранителя с держателем. Остается
	вывернуть одноразовый предохранитель-пробку, и ввернуть многоразовый автоматический выключатель-пробку.
	Такое изделие изображено на рисунке~\infig[par16], вместе с одноразовым предохранителем, заменить который оно предназначено.
	


\placefigure[here][par16]{Автоматический выключатель ПАР-16 производства Чебоксарского Электроаппаратного завода.
			}{\externalfigure[fuse-par-par.jpg][width=\textwidth]}


	Но снова поступают жалобы: в новых домах проводка толстая, номинальные токи 
	большие, как итог -- токи короткого замыкания огромные, тысячи ампер. 
	Когда контакты расходятся -- зажигается электрическая дуга, которая сама по 
	себе гаснуть не хочет, ток проводит, да ещё сильно греется, расплавляя металл. 
	А иногда из-за большого тока контакты свариваются меж собой, и пружина не 
	может их разъединить. Так мы снова столкнулись с отключающей способностью --
	это максимальная величина тока, которую гарантированно может разорвать 
	автоматический выключатель без вреда для себя. Если ток короткого замыкания 
	будет больше, чем отключающая способность, то штатная работа превращается в 
	лотерею.

	Если в предохранителях для увеличения отключающей способности мы засыпали 
	внутрь кварцевый песок, то в автоматический выключатель мы добавим дугогасительную 
	камеру. Это набор металлических пластинок рядом с контактами. Если при 
	размыкании контактов зажигается дуга, то её втягивает в пластинки камеры, 
	дробит на много маленьких дуг, которые быстро остывают, отдав тепло в металл 
	пластинок, и дуга гаснет. Единственное наглядное видео работы дугогасительной 
	камеры в замедленной съёмке я нашёл  у фирмы ABB. Но в книгу видео не вставишь, 
	поэтому на рисунке~\infig[arcbreaker] привожу фотографию контактов и 
	пластин дугогасительной камеры. \footnote{\hyperlink{https://www.youtube.com/watch?v=850aO98OAyI}}

	\placefigure[page][arcbreaker]{Дугогасительная камера и контакты в автоматическом выключателе
			}{\externalfigure[arcbreaker.jpg][width=\textwidth]}

	Теперь наш автоматический выключатель с дугогасительной камерой способен 
	отключить цепь с током в несколько тысяч ампер и не сломаться. Нанесём 
	маркировку отключающей способности в виде значения тока в амперах в прямоугольнике.
	\attention{} Важно отметить, что отключающая способность указывается для рода тока,
	на который рассчитан автоматический выключатель, обычно это переменный ток. При 
	постоянном токе дуга горит устойчивее, и отключающая способность может быть 
	в десятки раз ниже. Поэтому не используйте автоматические выключатели для 
	   переменного тока бездумно в цепях с постоянным током!

	Чтобы горячая электрическая дуга, пока остывает в дугогасительной камере, не 
	прожгла дыру в корпусе (и не повредила соседнее оборудование), добавим теплоизолирующий 
	вкладыш, который изображён на рисунке~\infig[heatshield].

	\placefigure[here][heatshield]{Механизм защиты от прогара вбок у разных моделей 
	автоматических выключателей. Слева -- керамический вкладыш. 
	В центре -- пластиковый вкладыш, справа -- вместо вкладыша сделано оребрение корпуса
			}{\externalfigure[heatshield.jpg][width=\textwidth]}

	Засунем всё в более удобный корпус, добавим всякие мелкие приятности, 
	вроде дырочек для пломбирования, флажочки-индикаторы состояния и т.д. 
	Так получаем современное устройство, изображенное на рисунке~\infig[modernbreaker].

	\placefigure[here][modernbreaker]{Современный модульный автоматический выключатель (в разрезе)
			}{\externalfigure[modernbreaker.jpg][width=\textwidth]}
%TODO сфотать армат
	Пока конкуренты делали рычаги механизма расцепления из металла, мы поработали 
	и сделали механизм из высококачественного пластика. За счёт меньшей массы 
	пластика мы уменьшили моменты инерции деталей механизма и добились того, 
	что наш автоматический выключатель срабатывает ещё быстрее -- не за десятки, а за единицы миллисекунд, с таким быстродействием ток короткого 
	замыкания не успевает вырасти до максимально возможных значений, что даёт 
	одни преимущества. Для отличия будем писать на корпусе класс токоограничения -- 
	показатель того, насколько быстро автоматический выключатель может отключиться. 
	Класс токоограничения просто обозначим цифрой в квадратике 1, 2 или 3 и 
	приведём табличку в документации, какие параметры соответствуют классу.

	Но снова поступают жалобы! Говорят, наши автоматы через раз вышибает при включении 
	мощной нагрузки. Особенно возмущались установщики светодиодного освещения. 
	Говорят, у них всего 10 светильников светодиодных с драйверами по \unit{35 watt}: 
	всего в сумме \unit{350 watt}, это примерно \unit{1,5 ampere} потребления из сети. 
	Тем не менее, при включении автомат на \unit{16 ampere} отключается. Вздыхаем, произносим 
	мантру "ну-почему-никто-не-читает-инструкции", показываем пальцем на рисунок \infig[LEDCurrent].

	\placefigure[here][LEDCurrent]{Вырезка из документации на светодиодный драйвер Meanwell LPV-35
			}{\externalfigure[inrush][width=\textwidth]}

	Большинство оборудования при включении потребляет стартовые токи в несколько 
	раз больше, чем в рабочем режиме. Именно поэтому, когда вы включаете что-то 
	мощное, свет на долю секунды притухает. Тепловой расцепитель медленный, и обычно 
	на кратковременные перегрузки не реагирует, а вот электромагнитный расцепитель 
	успевает сработать. Поэтому расширим наш ассортимент автоматических выключателей, 
	сделав разные электромагнитные расцепители, и обозначим их тип буквой:

	\startitemize[packed]
		\item B -- электромагнитный расцепитель сработает  при превышении номинального тока в 3--5 раз. 
		Подойдёт для освещения, бытовых нагревательных приборов, большинства электронных устройств.
		\item C -- электромагнитный расцепитель сработает при превышении номинального тока в 5--10 раз. 
		Подойдёт для потребителей с двигателями, мощными трансформаторами, групп осветительных приборов.
		\item D -- электромагнитный расцепитель сработает при превышении номинального тока в 10--20 раз. 
		Подойдёт для использования в промышленном производстве: для приборов с могучими моторами, 
		систем с множеством мощных импульсных блоков питания и т.д. (Правда появляется опасность, 
		что на слабой проводке тока короткого замыкания окажется недостаточно для срабатывания.)
		\item Для промышленности  будем поставлять ещё автоматические выключатели с 
		маркировкой K (8--12 раз) и Z (2--3 раза).\footnote{Стандарт на промышленные автоматические выключатели,
		в отличие от бытовых, отдаёт тип электромагнитного выключателя на совесть 
		производителя, поэтому может встретиться любая буква, а скрытые за ней  
		характеристики записаны в документации. Но K и Z фактически стали типовыми 
		и наиболее распространёнными.}
	\stopitemize

	Ну и троллинга ради, номинальный ток автоматического выключателя напишем 
	специально мелким, трудночитаемым шрифтом (передаю привет компании schneider electric).

	Снабдим всё это графиком времятоковых характеристик. График наглядно 
	показывает время, за которое сработает автоматический выключатель при разных 
	превышениях номинального тока. График изображён на рисунке~\infig[breakercurve].

	\placefigure[left][breakercurve]{Времятоковые характеристики автоматических выключателей с
			типами электромагнитных расцепителей B, C и D
			}{\externalfigure[curvesBCD.svg][width=0.4\textwidth]}

	Так как у нас ощутимый разброс параметров, то вместо тонких линий на графике 
	изображены области, в которых окажется времятоковая характеристика случайно 
	выбранного из партии автоматического выключателя. Мы видим, что при небольшом 
	превышении тока тепловой расцепитель работает одинаково, более-менее точно и 
	медленно, при превышении тока в 1,45 раза (т.е. на автомате написано С16, а 
	через него протекает \unit{23 ampere}) он отключится за время менее 1 часа. 
	А если ток превышает номинальный в 2,55 раза – то менее чем за 1 минуту. 
	Зато, если у нас ток на всего лишь на секунду превысит номинальный в 4 раза, 
	то автомат "В" у нас сработает, а вот автоматы "C" и "D"  не сработают.

	Базовая конструкция автоматического выключателя -- тепловой, электромагнитный 
	расцепитель, механизм свободного расцепления и дугогасительная камера --
	устоялась, и не меняется десятилетиями. Это позволило ввести стандарты и 
	обеспечить взаимозаменяемость автоматических выключателей разных производителей, 
	если обозначенные на их лицевой панели параметры совпадают. Учитывая, что форма корпуса автоматических выключателей тоже регламентируется стандартами, то в итоге простор 
	для конкуренции производителей очень узок. Если надёжность работы, качество 
	контактов и другие параметры невидимы для покупателя, то остается рекламировать совсем мелочи. Например контакты, облегчающие использование шин при серийной сборке щитов,
	или отворот-язычок на клемме, который не позволит воткнуть провод неправильно, работая вслепую. Ну и, конечно же, самая важная "инновация", патентованный рычажок включения эргономичной формы
	в стильной, модной, молодежной цветовой гамме.

\stopchapter

\startchapter[title={Выбор автоматического выключателя}]

	Из предыдущей главы мы разобрались с конструкцией автоматического выключателя, 
	но для нас чаще важен другой вопрос: не как он устроен, а как его правильно подобрать по параметрам?  

	\startsubject[title={Определимся с целью}]
			Для начала нужно напомнить,  для чего нам автоматический выключатель 
		в электрощите. Задача автоматического выключателя --- защитить стационарную 
		кабельную линию от протекания токов свыше предельно допустимых. Если ток 
		превышен, то проводники нагреваются, из-за этого может деградировать раньше 
		времени изоляция или даже расплавиться сам проводник. Даже если не случится пожара, то может понадобиться 
		дорогостоящий ремонт с работами по замене замурованной в стенах электропроводки.
		А ток может быть превышен, если к линии подключили слишком много потребителей 
		(происходит перегрузка), или если происходит короткое замыкание.  
		Поэтому главный параметр, который нам надо знать для правильного подбора 
		автоматического выключателя -- какого сечения проводник будет подключаться к 
		автоматическому выключателю и, следовательно,  какой допустимый рабочий ток этого проводника. Именно соблюдение условий по нагреву позволяет кабелю служить десятки лет.
	\stopsubject

	\startsubject[title={Номинальный ток}]
		Поняв, что автоматический выключатель должен защитить кабельную линию от 
		протекания тока свыше допустимого, мы должны понять, какой же ток является допустимым. 
		Чаще всего используют табличку из правил устройства электроустановок (таблица 1.3.4 источника), продублированную для удобства на рисунке~\infig[currents]


		\placefigure[here][currents]{Таблица из правил устройства электроустановок.
			}{\externalfigure[pue.svg][width=\textwidth]}

%	\framed[background=color,backgroundcolor=red]{свертать таблицу}

%		\starttabulate[|cp(6em)|cp(6em)|cp(6em)|cp(6em)|cp(6em)|cp(6em)|]
%		\HL
%		\VL Сечение проводника жилы, \unit{square millimeter} \VL Ток, ампер, для проводов и кабелей \VL \AR
%		\HL
%		\VL \VL одножильных \VL двухжильных \VL трехжильных \VL	\AR
%		\VL \VL в воздухе \VL в воздухе \VL в земле \VL в воздухе \VL в земде\VL \AR
%		\HL
%		\VL 1.5 \VL 23 \VL 19 \VL 33 \VL 19 \VL 27 \VL \AR
%		\VL 2.5 \VL 30 \VL 27 \VL 44 \VL 25 \VL 38 \VL \AR
%		\VL 4   \VL 41 \VL 38 \VL 55 \VL 35 \VL 49 \VL \AR
%		\VL 6   \VL 50 \VL 50 \VL 70 \VL 42 \VL 60 \VL \AR
%		\VL 10  \VL 80 \VL 70 \VL 105 \VL 55 \VL 90 \VL \AR
%		\VL 16  \VL 100 \VL 90 \VL 135 \VL 75 \VL 115 \VL \AR
%		\VL 25  \VL 140 \VL 115 \VL 175 \VL 95 \VL 150 \VL \AR
%		\VL 35  \VL 170 \VL 140 \VL 210 \VL 120 \VL 180 \VL \AR
%		\VL 50  \VL 215 \VL 175 \VL 265 \VL 145 \VL 225 \VL \AR
%		\VL 70  \VL 270 \VL 215 \VL 320 \VL 180 \VL 275 \VL \AR
%		\VL 95  \VL 325 \VL 260 \VL 385 \VL 220 \VL 330 \VL \AR
%		\VL 120 \VL 385 \VL 300 \VL 445 \VL 260 \VL 385 \VL \AR
%		\VL 150 \VL 440 \VL 350 \VL 505 \VL 305 \VL 435 \VL \AR
%		\VL 185 \VL 510 \VL 405 \VL 570 \VL 350 \VL 500 \VL \AR
%		\VL 240 \VL 605 \VL -   \VL -   \VL -   \VL -   \VL \AR
 %		\stoptabulate

		Если открыть источник и посмотреть то, что написано мелким шрифтом в 
		сноске, то мы увидим -- эта табличка составлена для окружающей температуры \unit{+25 celsius}, 
		температуры земли \unit{+15 celsius} и температуры токоведущей жилы (!!!) \unit{+65 celsius}. 
		Длительная работа изоляции при повышенной температуре ускоряет процесс старения 
		полимеров, поэтому моё личное мнение -- для долгой и надёжной работы, указанные 
		цифры стоит уменьшить  хотя бы на четверть. Если кабель проложен таким образом, 
		что его охлаждение затруднено, то предельно допустимый рабочий ток также 
		уменьшают -- например, если кабель расположен в пучке с другими кабелями или под 
		слоем теплоизоляции.

        И вот в этом месте \attention{} подходим к самой неочевидной вещи. В таблице указаны 
		предельно допустимые токи кабеля, а на автоматических выключателях указан номинальный 
		ток. Номинальный ток автоматического выключателя, указанный  на нём -- это ток, 
		который может длительно проходить через автоматический выключатель и не 
		вызывать его отключения. Для определения тока отключения заглянем в документацию, 
		в график времятоковых характеристик.

		\placefigure[here][tippingtolerance]{Слева изображён график времятоковой характеристики 
			единичного экземпляра автоматического выключателя. Стрелками указано определение
			времени срабатывания при токе 4\times I\low{ном}, оно равно 5 секундам.
			Справа изображены время-токовые характеристики нескольких экземпляров
			автоматических выключателей и поле допуска, в которое укладывается любой взятый
			из партии автоматический выключатель. По графику видно, что при токе 4\times I\low{ном}
			время срабатывания автоматического выключателя будет от 0,4 до 20 сек.
			}{\externalfigure[tippingtolerance.svg][width=\textwidth]}


		Но это график конкретного экземпляра автоматического выключателя. В реальном 
		мире, у автоматических выключателей есть разброс характеристик, даже у 
		выключателей, взятых из одной коробки. Если нанести на график кривые времятоковых 
		характеристик пары десятков экземпляров и очертить диапазон, в котором лежат графики, 
		то мы получим область на графике. Любой случайно взятый из коробки автоматический 
		выключатель уложится своей времятоковой характеристикой в эту область на графике. 
		В результате, если воспользоваться графиком и попытаться определить время 
		срабатывания автоматического выключателя для конкретного тока,  то мы получим 
		диапазон значений времени, за которое сработает автоматический выключатель. 


		Думаю  очевидно, что в расчётах стоит полагать, что нам попался самый 
		плохой экземпляр, и берётся самое худшее значение.

		Из прошлой главы мы знаем, что в типовом автоматическом выключателе есть два 
		расцепителя -- тепловой, который достаточно точный, но медленный, и 
		электромагнитный -- очень быстрый, но не точный.  В итоге зависимость  
		времени срабатывания от протекающего тока становится нелинейной. Для наглядности возьмём 
		автоматический выключатель, на котором указан номинальный ток \unit{16 ampere}. При перегрузке 
		будет работать тепловой расцепитель:

		До тока в 1,13 от номинального, расцепления совсем не произойдет 
		($16 \times 1,13=\unit{18,08 ampere}$)

		При токе в 1,45 от номинального тепловой расцепитель сработает, но за 
		время менее 1 часа (!) ($16 \times 1,45=\unit{23,2 ampere}$)

		При токе в 2,55 от номинального тепловой расцепитель сработает за время 
		менее 60 сек. ($16 \times 2,55 = \unit{40 ampere}$)

    	При ещё большем превышении тока -- сработает электромагнитный расцепитель, 
		но об этом чуть позже.

		Всё это становится понятнее, если посмотреть на график на рисунке~\infig[breakertypes].


		Откуда взялись эти магические цифры? Из стандарта (у нас в стране -- ГОСТ 60898-1-220). 
		Разработчики стандарта условились, что разброс параметров срабатывания расцепителей 
		должны быть в этих пределах. Причём, скорее всего, взяли просто две удобные точки 
		времени -- 1 час и 1 минута, и воспользовались статистическими данными, чтобы 
		получить кратности номинального тока.

		И чтобы жизнь совсем мёдом не казалась, стоит добавить, что в зависимости 
		от температуры окружающей среды применяются поправочные коэффициенты тока 
		расцепления для разной окружающей температуры. На жаре тепловой расцепитель 
		прогревается и срабатывает быстрее, а вот на морозе наоборот.\footnote{Поэтому указывают,
		что испытания проводят на холодном автоматическом выключателе. Если сработавший
		автоматический выключатель сразу включить, пока тепловой расцепитель
		толком не остыл, то повторно он сработает гораздо быстрее} Электромагнитный 
		расцепитель на температуру окружающей среды не реагирует.

			\placefigure[left][tippingtemp]{Влияние окружающей температуры на времятоковую
			характеристику автоматического выключателя. 
			}{\externalfigure[tippingtemp.svg][width=0.4\textwidth]}

		Чтобы проиллюстрировать, почему важно держать в уме все эти особенности и почему
		автор призывает уменьшать рабочий ток проводников из таблицы, представим 
		роковое стечение обстоятельств. В частный дом заходит кабель сечением, например, \unit{1,5 square milli meter}. 
		Щиток с автоматическим выключателем находится в холодном предбаннике, когда 
		на улице мороз \unit {-35 celsius}. Кабель от щитка идёт через стену под слоем утеплителя. 
		Автоматический выключатель на \unit {16 ampere} почти час (!) будет пропускать ток в 
		(16 \times 1,45 \times 1,25 \footnote{поправочный коэффициент на температуру, взят из документации 
		выключателя} = \unit {29 ampere}). При \unit{19ampere} по табличке из ПУЭ 
		у нас жилы будут горячими: \unit{+65 celsius}, а под слоем утеплителя изоляция 
		уже начнёт плавиться.
%%TODO пофиксить29а
		Ещё раз резюмирую: {\bf Номинальный ток автоматического выключателя НЕ РАВЕН 
		предельно допустимому току кабеля. }\attention{} Превышение предельного тока через кабель 
		должно вызывать отключение автоматического выключателя в адекватное время.
	\stopsubject

	\startsubject[title={Тип электромагнитного расцепителя}]
		Тепловой расцепитель срабатывает медленно, что плохо при коротком замыкании -- токи 
		могут быть огромными, и даже за одну секунду могут наделать бед. Поэтому 
		для работы с короткими замыканиями в конструкции автоматического выключателя 
		есть электромагнитный расцепитель, который срабатывает за доли секунды. 
		Он включается в работу при превышении номинального тока в разы.

		Некоторые виды потребителей при включении потребляют ток, в разы превышающий 
		ток в рабочем режиме --  выше мы немного уже говорили про пусковые токи. Например, мотор в пылесосе в момент включения 
		кратковременно потребляет ток в 2--3 раза больший, но после разгона мотора -- то есть, через секунду или около того,
		потребление снижается до паспортных значений. Возможно, вы замечали, 
		как лампочки накаливания слегка притухают в момент включения техники с мощными 
		двигателями (пылесос, кухонный комбайн) именно по этой причине. График потребления 
		тока мотора пылесоса изображён на рисунке~\infig[startcurr]. Стоит отметить, что не любая мощная нагрузка имеет подобную пусковую характеристику: скажем, мощный нагреватель потребляет примерно одинаковый ток и при старте, и при длительной работе.


		\placefigure[here][startcurr]{График потребления тока нагрузкой в момент включения.
			Величина превышения пускового тока над номинальным рабочим, а также время выхода
			в рабочий режим отличаются у разных типов устройств.
			}{\externalfigure[startcurr.svg][width=\textwidth]}

		%https://www.researchgate.net/figure/Vacuum-cleaners-current-waveform_fig1_45812348
		Чтобы эти пусковые токи не заставляли срабатывать электромагнитный расцепитель при каждом включении, 
		его характеристику сдвинули в зону бóльших токов, чтобы такие кратковременные 
		превышения тока были в зоне теплового расцепителя, который в силу своей 
		инерционности такие краткосрочные процессы не замечает. 

		В итоге получилась линейка автоматических выключателей с одинаковыми 
		тепловыми расцепителями, но с разными электромагнитными расцепителями. Из-за огромного 
		разброса параметров электромагнитных расцепителей получились большие 
		разбросы кратности тока срабатывания:

		\startitemize[packed]
		\item Характеристика В -- электромагнитный расцепитель сработает при превышении тока в 3--5 раз

		\item Характеристика С -- электромагнитный расцепитель сработает при превышении тока в 5--10 раз

		\item Характеристика D -- электромагнитный расцепитель сработает при превышении тока в 10--20 раз
		\stopitemize
		
		Характеристики изображены на рисунке~\infig[breakertypes].

		\placefigure[here][breakertypes]{Времятоковые характеристики автоматических выключателей.
			Стандарт ГОСТ МЭК 60898-1 (слева) распространяется на бытовые автоматические выключатели,
			и типы электромагнитных расцепителей закреплены в стандарте. Справа времятоковые
			характеристики по ГОСТ МЭК 60947-2, который распространяется на промышленные автоматические выключатели.
			Типы электромагнитных расцепителей стандарт не описывает, производитель должен предоставить
			кривые времятоковых характеристик в документации. На графике нанесены кривые K и Z
			из документации компании АВВ.
			}{\externalfigure[tippingcurves.svg][width=\textwidth]}

		Есть и другие характеристики (K, Z и т.д), но встречаются крайне редко и 
		поставляются под заказ, поэтому опустим их.

		Если по какой-то причине стартовые токи кратковременно попадут в зону 
		действия электромагнитного расцепителя, возможны ложные срабатывания. 
		Именно для исключения таких ложных срабатываний и сделали несколько 
		типов характеристик.

		Некоторые производители для упрощения расчетов указывают стартовые токи. Например, на рисунке~\infig[LEDCurrent] 
		приведена выдержка из документации на  светодиодный драйвер уважаемой фирмы,
		при включении он кушает солидные \unit{55 ampere} (из-за зарядки конденсатора в блоке питания), 
		производитель даже сразу посчитал, сколько светодиодных драйверов можно 
		подключить параллельно на один автоматический выключатель.

		Получается, можно подключить всего 4 драйвера на автомат с характеристикой В и 7 драйверов на автомат с характеристикой С. 
		Кто бы мог подумать, что \unit{150 watt} светодиодного света могут 
		вышибать \unit{16 ampere} автомат! Ситуация становится ещё хуже, если 
		используются некачественные светодиодные светильники, где производитель 
		не только не предусмотрел плавный старт, и даже пусковой ток не регламентирует! 

		Если используется большое количество светодиодных светильников, то придётся 
		делить их на группы, чтобы одновременный пуск не вызывал срабатывание 
		автоматического выключателя\footnote{Или использовать реле ограничения 
		пускового тока, например Меандр МРП-101. Реле ограничивает ток нагрузки 
		встроенным резистором, которое закорачивается и не мешает нормальной 
		работе спустя пару секунд.}. 

		Пытливый читатель задастся вопросом -- а почему бы не взять просто 
		автоматический выключатель с характеристикой "C" или "D"? Тогда бы пусковые 
		токи не вызывали бы ложных срабатываний! Но не всё так просто....
	\stopsubject

	\startsubject[title={Ток короткого замыкания}]
		Можно иногда услышать выражение "сопротивление цепи фаза-нуль", оно по 
		сути про то же. Ток короткого замыкания --- это величина тока в цепи, который пойдёт,
		 если из-за повреждения случается короткое замыкание (прямое 
		соединение фазного проводника и нейтрального, или соединение фазного и 
		заземления) в самом дальнем участке. В идеальном мире с идеальными 
		проводниками ток короткого замыкания был бы бесконечным. Но в реальном 
		мире кабели имеют собственное сопротивление, и чем они длиннее и тоньше -- 
		тем выше их собственное сопротивление. При обычной работе это не так важно -- 
		их собственное сопротивление много меньше сопротивления нагрузки. 
		Но если случится короткое замыкание, ток будет ограничен именно этим 
		собственным сопротивлением всех проводников в цепи + внутреннее 
		сопротивление источника тока.

		А теперь смотрим. В деревне Вилларибо измеренный ток короткого замыкания 
		линии \unit{278 ampere}, и электрик поставил автоматический выключатель С16.
		Ток отмечен на графике, изображенном на рисунке~\infig[Villaribo]


		\placefigure[here][Villaribo]{Пояснение, как ток короткого замыкания влияет на работу защиты
			}{\externalfigure[shortcurr.svg][width=\textwidth]}

		Как видим, всё отлично -- при коротком замыкании, тока будет достаточно, 
		чтобы электромагнитный расцепитель сработал. А вот в деревне Виллабаджо 
		очень плохая проводка и ток короткого замыкания всего \unit {124 ampere}. 
		Точка на графике отмечена на рисунке.

		

		В самом худшем случае, электромагнитный расцепитель типа "С" сработает 
		при токе в 10 раз больше номинального (16\times10=\unit{160 ampere}). А значит, при \unit{124 ampere} 
		возможна ситуация, когда электромагнитный расцепитель при коротком 
		замыкании не сработает, а пока тепловой расцепитель успеет нагреться для 
		расцепления, по линии будет гулять ток в \unit{124 ampere}, что может закончиться 
		плохо. В таком случае деревне Виллабаджо нужно или менять проводку, 
		чтобы уменьшить потери, или использовать автоматический выключатель типа 
		В16, у которого электромагнитный расцепитель сработает в худшем случае 
		при токе 5\times16=\unit{80 ampere}. Теперь вы понимаете, почему характеристика типа D 
		(10--20\times I\low{ном}) в некоторых случаях может стать способом выстрелить себе в ногу?

		Как же определить ток короткого замыкания? Для проектируемых линий его 
		можно рассчитать -- длина кабеля известна, сечение тоже. Для линий, уже 
		находящихся в эксплуатации, можно только измерить, поскольку никто не знает, 
		на что пришлось пойти электрикам при ремонте повреждённых участков.

		Для определения тока короткого замыкания есть специальные приборы, 
		причём с соответствующими бумагами о прохождении поверки. Увы, в моей 
		коллекции таких современных приборов нет, но есть старые советские, 
		которые спешу показать читателю для общего развития. Небольшой 
		чемоданчик М-417, который изображён на рисунке~\infig[M417], измеряет сопротивление 
		цепи «фаза-нуль» путём измерения падения напряжения на известном 
		сопротивлении, а ток короткого замыкания необходимо рассчитывать.

		\placefigure[here][M417]{Измеритель сопротивления петли фаза-нуль M417. Производства СССР
			}{\externalfigure[M417.jpg][width=\textwidth]}


		Более продвинутый цифровой прибор Щ41160 изображён на рисунке~\infig[41160]. Устраивает короткое замыкание на 
		доли секунды и непосредственно измеряет ток. В коричневой коробочке на 
		проводе -- предохранитель на \unit{100 ampere}.
		
		\placefigure[here][41160]{Измеритель тока короткого замыкания Щ41160. Произведён в СССР
			}{\externalfigure[41160.JPG][width=\textwidth]}

		Как правило, ток короткого замыкания измеряют при введении линии в 
		эксплуатацию, и планово, раз в несколько лет. Только после измерения 
		тока короткого замыкания можно сказать, правильно ли подобрана защита.
	\stopsubject

	\startsubject[title={Отключающая способность}]
		А что, если ток короткого замыкания будет чересчур большим? Вот тут мы 
		сталкиваемся с отключающей способностью автоматического выключателя.  
		
		На автоматическом выключателе в прямоугольной рамке нанесена величина  
		отключающей способности в амперах -- это максимальный ток, который 
		способен разомкнуть автоматический выключатель без поломки. На рисунке~\infig[currcapacity] 
		изображены автоматические выключатели с отключающей способностью в 3000, 4500, 6000 и \unit{10000 ampere}.

		\placefigure[here][currcapacity]{Автоматические выключатели с разной отключающей способностью.
			}{\externalfigure[3-4500-10000.JPG][width=\textwidth]}

		Наглядно разницу внутренностей можно посмотреть на фото \infig[currinside]. 
		Увеличение отключающей способности автоматического выключателя 
		заставляет конструкторов не только усиливать дугогасительную камеру, 
		но и другие узлы, вплоть до прочности корпуса и путей выхода горячих газов.

		\placefigure[here][currinside]{Автоматические выключатели с разной отключающей способностью изнутри.
			}{\externalfigure[3-4500-10000-open.JPG][width=\textwidth]}

		Отключающая способность автоматического выключателя должна быть больше 
		тока короткого замыкания в линии. Как правило, \unit{6000 ampere} 
		достаточно для большинства применений. \unit{4500 ampere} обычно 
		достаточно для работы в линиях старых домов, но может быть недостаточным 
		в новых сетях.
	\stopsubject

	\startsubject[title={Коммутационная стойкость}]
		При каждом включении/отключении автомата меж контактов загорается дуга, 
		которая постепенно разрушает контактную группу. Производитель часто 
		указывает количество циклов включения/отключения, которые должны выдержать 
		контакты, выдержка из документации указана на рисунке~\infig[contactlife].

		\placefigure[here][contactlife]{Выдержка из документации на автоматический выключатель IEK
			}{\externalfigure[contactlife.jpg][width=\textwidth]}

		Отсюда легко видеть, что автоматический выключатель не замена нормальному 
		выключателю при частом использовании. Если пожадничать, и вместо пускателя 
		с контактором,  заставить сотрудника включать/отключать условную мешалку, дёргая 
		автомат по 10 раз в  день, то автомат может прийти в негодность менее 
		чем за пару лет, электрическая эрозия испортит контакты. Особенно это 
		заметно, если нагрузка имеет большое количество импульсных блоков питания
		(в том числе и светодиодные драйверы).

		Помните, каждая коммутация и каждое срабатывание автоматического выключателя "съедает" его ресурс.

	\stopsubject

	\startsubject[title={Класс токоограничения}]
		Наверное, самая мистическая характеристика. Указывается в виде цифры в 
		квадратике. Про неё в рунете написано мало и чаще ерунда. Класс 
		токоограничения, если упрощать, говорит о количестве электричества, 
		которое успеет пройти через автоматический выключатель при коротком 
		замыкании прежде, чем он отключит цепь, и  говорит о быстродействии. 
		Всего классов три, что изображено на рисунке~\infig[classes].
%https://www.asutpp.ru/tehnicheskie-harakteristiki-avtomaticheskih-vyklyuchateley.html
		\placefigure[here][classes]{приложение стандарта EN60898-1:2003
			}{\externalfigure[classes.jpg][width=\textwidth]}

		Что интересно, отечественными стандартами класс токоограничения не 
		регламентируется, поэтому на картинке выше нет кириллицы. Цифры в 
		таблице -- это величина интеграла Джоуля. Отечественные производители 
		указывают класс просто потому что "так принято", а не потому что этого требуют 
		отечественные стандарты :)  В быту на данный параметр можно не обращать 
		внимания -- классы хуже третьего встречаются в продаже нечасто.
	\stopsubject

	\startsubject[title={Селективность}]
		Вам бы не хотелось, чтобы при перегрузке или коротком замыкании 
		срабатывал автоматический выключатель где-то в опечатанном щите на 
		столбе у ввода в дом. При последовательном соединении автоматов защиты 
		подбором их характеристик можно добиться селективности -- свойству 
		срабатывать защите, расположенной ближе всего к повреждению, без срабатывания вышестоящей. 
		И у меня две новости. 

		Хорошая -- можно воспользоваться специальными таблицами, которые есть у 
		многих производителей, и подобрать пары автоматических выключателей, 
		которые при перегрузке будут обеспечивать селективность. На графике на рисунке~\infig[selectivity] это 
		видно как непересекающиеся графики работы расцепителей.

		\placefigure[here][selectivity]{Ток селективности двух автоматических выключателей.
		В зоне пересечения характеристик селективность не соблюдается. Можно делать ставки,
		какой из них сработает первым при коротком замыкании (возможно и оба).
			}{\externalfigure[selectivity.svg][height=0.9\textheight]}

		Но по графику вы могли понять, что плохая новость -- обеспечить полную 
		селективность автоматических выключателей при коротком замыкании 
		нельзя. \attention{} Кривые пересекаются в области больших токов. Поэтому 
		чаще всего речь идёт о частичной селективности. Например, если синий график -- 
		автомат В10, а фиолетовый В40, то ток селективности составит \unit{120 ampere} 
		\footnote{значение взято из таблиц одного производителя для конкретной модели автоматов.}.
		То есть, при токах меньше тока селективности, автоматы будут срабатывать как задумано. При токах 
		больше сработать могут оба устройства защиты. Каким будет ток короткого
		замыкания, предсказать сложно. Если коротнёт включенный в длинную переноску
		перфоратор -- то, возможно, селективность обеспечится. Если замкнется включенная
		в настенную розетку зарядка от телефона, то может выбить все автоматы вплоть
		до вводного.

		В бытовой серии модульных автоматических выключателей обеспечивать 
		селективность, даже частичную, довольно трудно. \attention{} Лишь большие и мощные 
		устройства защиты, например, на подстанциях, имеют тонкие настройки 
		уставок расцепителей для обеспечения селективности с вышестоящими 
		устройствами защиты.

        Справедливости ради, стоит сказать, что есть модели полностью 
		селективных модульных автоматических выключателей, например, в каталоге 
		компании АВВ\footnote{например, селективные автоматы ABB S750 DR}. 
		Увы, они стоят очень дорого, и такие решения применяют в быту крайне редко.
	\stopsubject

	\startsubject[title={Да скажи уже, что ставить?!}]
		Прежде всего, то, что предусмотрено проектом.
		
		Ну а если уж совсем среднестатистический случай с кучей оговорок, то:

		Линия \unit{1,5 square millimeter} -- Автомат В10 с отключающей способностью \unit{6000 ampere}

		Линия \unit{2,5 square millimeter} -- Автомат В16 с отключающей способностью \unit{6000 ampere}

		Применение автоматического выключателя с характеристикой "C" или "D" 
		вместо "B" должно иметь обоснование.
	\stopsubject

	\startsubject[title={Плюшки}]
		Государственные и межгосударственные стандарты регламентируют не только параметры автоматического выключателя,
		но и его корпус, вынуждая производителей конкурировать  меж собой разными 
		небольшими приятными доработками, например:

		\startitemize

		\item Различные шторки/колпачки/крышечки для пломбирования вводного автомата 
		по требованию электросетевой компании.

		\item Визуальный индикатор фактического состояния контактов. Такой индикатор 
		останется красным, если контакты из-за перегрузки сварились

		\item Окошки для дополнительных модулей-нашлёпок с электромагнитными расцепителями, 
		контактами, позволяющие отключить автоматический выключатель можно по внешнему 
		сигналу, или наоборот, внешней автоматикой проверить в каком состоянии 
		он сейчас находится.

		\item Дополнительные окошки у клемм для использования гребёнки при подключении.

		\stopitemize

		И много других особенностей, не влияющих на основную работу — защиту от перегрузки.
	\stopsubject
\stopchapter

\startchapter[title={Выключатели дифференциального тока}]
	Эти устройства более известны под своим старым названием -- УЗО -- 
	устройство защитного отключения. Это единственные устройства в электрощите, 
	основная цель которых -- защита человека от поражения электрическим током. 
	Но обо всём по порядку.

	Сейчас, в 21 веке, электричество есть практически в каждом доме. И почти 
	каждый гражданин знает, что электричество может убить. Новость о том, что 
	где-то кого-то убило током? для нас звучит обыденно, и в СМИ об этом пишут, 
	только если случай особенный, или убило известную личность, или раздолбайство 
	совсем уж вопиющее. Но в конце XIX -- начале XX века каждая смерть от удара 
	током была в центре внимания: электричество было в диковинку. На рисунке~\infig[electrodeath] 
	немного заметок, которые попались мне на глаза.

	\placefigure[here][electrodeath]{Вырезки новостей о смерти от удара электрическим током 
	из журнала The telegraphic journal and electrical review конца XIX века
			}{\externalfigure[electrodeath.jpg][width=\textwidth]}


	Тысячи разобранных случаев, когда кто-то был убит электричеством, позволили 
	инженерам выяснить некоторые закономерности и предпринять меры. А именно:

	Выяснилось, что случаев смерти, когда человек умер от обращения с напряжением 
	менее \unit{50 volt}, почти нет. Низкое напряжение, с кучей оговорок, вполне 
	себе безопасно. Кто лизал "Крону" в детстве, чтобы оценить, поработает ли ещё батарейка? 
	Использование низкого напряжения (\unit{12 volt}, \unit{24 volt}, \unit{36 volt} и т.д.)  
	даёт практически полную безопасность, например, при работе по пояс в воде. 
	Но повсеместный переход на безопасное низкое напряжение невозможен. 
	Если бы мы жили в альтернативной вселенной, где напряжение сети в домах вместо \unit{230 volt} 
	всего \unit{12 volt}, то чайник бы кушал не \unit{16 ampere} тока, а 
	почти \unit{300 ampere}, и подключался бы в розетку толстенным кабелем. 
	А всё потому, что при снижении напряжения придётся повышать ток, чтобы 
	мощность прибора оставалась прежней. Бóльший ток требует толстых кабелей, 
	иначе будут расти потери. Выходит, повсеместное использование безопасного низкого 
	напряжения не только экономически затратно, но и технически сложно. 
	Забавно, что страны, использующее напряжение \unit{120 volt}, например США, имеют весь спектр проблем низкого напряжения, хоть и в легкой форме:
	американский электрочайник вынужденно имеет мощность \unit{1,5 kilo watt}, что меньше типовых европейских \unit{2 kilowatt}, иначе провода слишком сильно нагреваются.
    Многие электроприборы для американского рынка заметно слабее своих европейских аналогов.

%Peng, Z., & Shikui, C. (1995). Study on electrocution death by low-voltage. Forensic Science International, 76(2), 115–119. doi:10.1016/0379-0738(95)01804-2
	Второе важное наблюдение. Ток течёт в замкнутой цепи; если Земля -- часть этой 
	цепи, то человек всегда в опасности. А вот если человека подключить к разным 
	цепям, изолированным друг от друга, например, если коснуться одной рукой одного 
	изолированного от земли генератора, а второй -- другого изолированного 
	генератора, то ничего не произойдёт. Цепь не замкнута -- ток не течёт. 
	Так появилась гальваническая развязка и развязывающие трансформаторы. 
	В старых домах устанавливали развязывающий трансформатор с розеткой в санузле, 
	с подписью "для электробритвы". Электробритвой на \unit{220 volt}, включённой 
	в эту розетку, можно было безопасно пользоваться, прикосновение к проводнику под 
	напряжением, для стоящего в заземлённой ванне человека, не могло убить. Правда, маленький 
	трансформатор мог потянуть только несколько десятков ватт мощности нагрузки, 
	включение в такую розетку фена или обогревателя просто бы его сожгло. 
	Поэтому в быту способ не прижился, у вас же нет отдельной комнаты под 
	трансформатор гальванической развязки?) 

%https://mastergrad.com/forums/t82050-rozetka-v-vannoy-i-razdelyayushchiy-transformator/
	Ну и наконец, усреднив индивидуальные особенности, учёные составили график зависимости 
	силы тока, времени воздействия 	и последствий для человека, изображённый на 
	рисунке~\infig[currentreactions]. Да простят меня авторы, я его немного упростил для понимания.

	\placefigure[here][currentreactions]{График зависимости воздействия на организм электрического тока
			от времени и силы. Для упрощения понимания автор пририсовал на график рожицы и линию
			30 мА. График взят из стандарта IEC 60479-1
			}{\externalfigure[Currentreactions.svg][width=\textwidth]}


	Оказалось, что убивает не напряжение само по себе, а протекающий через тело 
	ток.  При токах менее \unit{0,5 milli ampere} (светло-зелёная область) 
	человек ничего не чувствует. При токах 0,5--\unit{20 milli ampere} (темно-зелёная область) 
	ток уже неприятно щиплет, кусает. При токах 20--\unit{100 milli ampere} (жёлтая область) 
	тело человека уже конкретно трясёт, сводит мышцы (руку не отдёрнешь), и протекающий ток причиняет боль. При 
	токах более \unit{100 milli ampere} уже некоторые могут умереть. Из графика 
	можно понять откуда взялась величина \unit{30 milli ampere} (зелёная линия) -- 
	при меньших токах человек вряд ли умрёт и может сам принять меры, если 
	чувствует, что его бьёт током. А вот при токах больше -- нужно срочно спасать, 
	иначе помрёт. 

	\startsubject[title={Защита все-таки нужна}]
		Применение низкого напряжения или использование гальванической развязки 
		не очень удобный способ защиты человека, поэтому они применяются только в 
		узких областях, там, где иначе никак. А как же защитить  человека от 
		поражения электрическим током, не сильно изменяя существующие электросети? 
		Идея проста и гениальна -- нужно анализировать дифференциальный ток.
 
		Дифференциальный ток – это разница в токах  двух проводников, 
		например между фазным, уходящим в нагрузку и нулевым, возвращающимся из 
		нагрузки. Появление ощутимого дифференциального тока в цепи чаще всего 
		ненормально, это означает, что часть тока уходит вне контура, который, 
		очевидно, перестал быть замкнутым, и лучше отключить цепь, вдруг ток утекает в землю через человека? 
		Это можно проиллюстрировать на примере сравнения  расхода теплоносителя в батареи и из батареи отопления. 
		Если в батареи уходит 100 л/мин и возвращается 100 л/мин, то система герметична. 
		Если в батареи подаётся 100 л/мин, а возвращается по какой-то причине 
		только 98 л/мин, то 2 литра в минуту куда-то вытекает!

		В идеальном мире нам достаточно поставить устройство, контролирующее 
		сам факт появления дифференциального тока. Если всё в порядке -- то 
		дифференциального тока  нет. Если же ток появился -- отключаем нагрузку. 
		Но в реальном мире, к сожалению, дифференциальный ток (ток утечки) появляется 
		даже в полностью исправных устройствах, поэтому придётся пойти на компромисс и 
		выбрать некоторую пороговую величину дифференциального тока, превышение 
		которой будет вызывать отключение.

		Поставим себя на место инженеров начала 20 века и попробуем изобрести 
		устройство обнаружения дифференциального тока. Нам нужно обнаружить 
		появление утечки величиной \unit{30 milli ampere}, бóльшие токи, если протекают
		через человека, уже опасны для жизни.

		Первая конструкция --- два одинаковых электромагнита, друг напротив друга, 
		занимаются перетягиванием якоря. Протекающий в нагрузку и из нагрузки ток, 
		протекая через обмотки, создаёт магнитное поле, тем сильнее, чем больше ток. 
		Если в цепи нет утечек, токи через электромагниты равны, магнитное поле 
		они развивают одинаковое, и якорь стоит на месте. Если в цепи у нас появляется 
		утечка, то ток через один из электромагнитов будет меньше (ток нагрузки минус ток утечки), 
		чем через второй (ток нагрузки), якорь перетянется и разомкнёт контакты.

		\placefigure[here][oldcurrentbalance]{Механизм сравнения токов из патента US3213321 1910 год
			}{\externalfigure[oldcurrentbalance.jpg][width=\textwidth]}

		Теоретически схема рабочая, но чересчур капризная: она требовала очень 
		точного изготовления электромагнитов и тонкой настройки механики. Поэтому 
		инженеры стали думать, как избавиться от движущихся частей. Так пришли к 
		современной схеме с трансформатором, которая схематично изображена на рисунке~\infig[RCDPrinciple].

		\placefigure[here][RCDPrinciple]{Принцип работы трансформатора токов современного УЗО
			}{\externalfigure[RCDPrinciple.svg][width=\textwidth]}

		На замкнутом магнитопроводе делают две обмотки, включённые в противофазе, 
		и третью обмотку для привода соленоида. Если токи через первую и вторую 
		обмотку равны, то равны и магнитные поля. Так как они направлены 
		навстречу друг другу, то и суммарный магнитный поток через третью обмотку 
		будет равен нулю. Если же есть утечка, токи перестают быть равны, и через 
		третью обмотку начнёт циркулировать магнитное поле, пропорциональное этой 
		разнице. Там, где есть переменное магнитное поле -- там есть  индукция и 
		возбуждается ток. Если его достаточно для срабатывания соленоида, то якорь 
		высвободит защёлку и отключит цепь.

		Получилось гениальное в своей простоте и надежности устройство. Правда, дешёвым оно 
		не получилось -- механика всеравно оказалась нежной и капризной, шутка 
		ли -- обнаружить \unit{30 milli ampere} разницу при номинальном токе \unit{16 ampere}, 
		это всё равно, что расслышать писк мыши на фоне грохота поезда. Электромеханическое
		УЗО изображено на рисунке~\infig[rcdinside].

		\placefigure[here][rcdinside]{Устройство электромеханического УЗО. В чёрном корпусе,
			в который уходят тонкие проводники, находится расцепитель.
			}{\externalfigure[RCDInside.jpg][width=\textwidth]}

		Электромеханические УЗО для своей работы используют энергию 
		дифференциального тока -- именно он в конечном итоге превращается в 
		механическую силу, которая дёргает расцепитель. Само собой, такие нежные 
		механические узлы не способствовали дешевизне устройства.  С появлением 
		недорогой электроники  конструкцию УЗО модернизировали -- нежный расцепитель\footnote{В УЗО
		обычно используется хитрый расцепитель с постоянным магнитом. Магнит держит 
		якорь во включённом состоянии, при появлении дифференциального тока электромагнит
		"приглушает" магнитное поле магнита и пружинка уводит якорь в выключенное состояние.
		Из-за этого электромеханическое УЗО отключается, если поднести к нему магнит.
		Но так делать ни в коем случае нельзя! Если взять слишком сильный магнит, можно частично
		размагнитить магнит в расцепителе, и параметры УЗО по отключающему дифференциальному току уплывут.}, 
		срабатывавший на небольшой дифференциальный ток заменили дубовым соленоидом 
		с электронным усилителем. Расплатой послужила зависимость от 
		напряжения в сети, для работы усилителя приходится брать немного энергии 
		из сети, но такие УЗО получились компактнее и значительно дешевле.

		\placefigure[here][ussrrcd]{УЗОШ. Устройство защитного отключения школьное, 
		произведено в СССР в 1980е годы. Конструктивно представляет собой электронное УЗО.
			}{\externalfigure[uzosh.jpg][width=0.5\textwidth]}

		А теперь внимание, важный момент, что будет при коротком замыкании в 
		нагрузке? Ничего! Так как условия для срабатывания нет -- разницы токов 
		на входе в УЗО и на выходе из УЗО нет.  Провода накалятся докрасна, 
		изоляция расплавится и потечёт на пол, а УЗО не отключится, поскольку не имеет
		защиты от сверхтока. {\bf Поэтому УЗО без встроенной защиты от сверхтока 
		ВСЕГДА применяется в паре с автоматическим выключателем или с плавким 
		предохранителем.}\attention{} Путём скрещивания УЗО и автоматических выключателей 
		производители вывели гибрид -- АВДТ (автоматический выключатель 
		дифференциального тока), который чаще на жаргоне называют дифавтоматом, 
		такое устройство самодостаточно и наличия дополнительного автоматического 
		выключателя не требует.
	\stopsubject

	\startsubject[title={Типы ВДТ}]
		Изобретённое УЗО отлично работало, если бы не распространение 
		полупроводниковых устройств. Очень многие устройства стали преобразовывать 
		внутри себя напряжение и род тока -- делать из переменного тока постоянный, 
		потом снова переменный, иногда другой частоты или величины. Из-за этого 
		стали возможны всяческие неприятные особенности, например, если в устройстве 
		замкнёт на корпус одну из линий с постоянным током, ток утечки будет 
		пульсирующим -- в землю будут уходить только положительные полуволны тока. 
		\footnote{А если есть конденсаторы, то вообще утечка будет постоянного тока, без пульсаций}
		Обычное УЗО в таких случаях может не сработать. Для таких случаев 
		разработали специальные УЗО, рассчитанные срабатывать не только при 
		синусоидальной форме тока утечки, но и при постоянном пульсирующем токе 
		утечки и назвали их тип А. А старые УЗО, срабатывающие только на переменный 
		ток, назвали тип АС. А для совсем уж неприятных случаев (например, пробой 
		цепей после силовых ключей в преобразователях с высокими частотами 
		преобразования, или пробой выхода выпрямителя с мощными сглаживающими конденсаторами) 
		придумали тип В. Нагляднее всего разницу меж типов УЗО демонстрирует рисунок \infig[rcdtypes].


		\placefigure[here][rcdtypes]{Пример влияния внутренней схемы устройства на
			форму тока утечки и чувствительность к таким токам разных типов ВДТ (УЗО).
			По мотивам пособия Doepke AC-DC sensitive residual current devices (Type B RCDs).
			Instructions for use and technical information. 2017.
			}{\externalfigure[RCDtypes.svg][width=\textwidth]}

		\placefigure[left][rcdtypeslogos]{Знаки, которыми обозначают устройства разных типов
			}{\externalfigure[RCDTypesSymbols.svg][width=0.3\textwidth]}

		На рисунке видно (я значительно сократил таблицу оригинала), что тип B 
		самый универсальный, но при этом и самый дорогой. Конечно, при возможности стоит
		выбрать самое прогрессивное устройство типа B. Но к сожалению, в наших условиях
		чаще всего можно услышать вопрос: ставить тип A или тип AC? Ответить на него можно так:
		тип AC можно выбирать только если устройства на линии не имеют
		электроники внутри себя. Это, например, обогреватели, бойлеры, вентиляторы
		с электромеханическим управлением.  Наличие любого электронного блока
		сразу говорит о необходимости типа A, а лучше, конечно же, B.

		Для обеспечения селективности при последовательном соединении УЗО, 
		создали специальные селективные варианты, часто с обозначением S или G 
		в названии. Они имеют встроенную задержку отключения на несколько десятков-сотен 
		миллисекунд. Так, если на вводе в дом стоит селективное УЗО, а на этажном 
		щитке неселективное, то при замыкании напряжения на корпус стиральной 
		машины сначала сработает неселективное УЗО на этаже, пока селективное 
		даёт задержку. Если по окончании задержки дифференциальный ток не исчез -- 
		сработает селективное УЗО. Про селективность я писал в главе про предохранители. 
		Селективность не зависит от номинального порогового дифференциального тока, 
		то есть при пробое на корпус сработают сразу и УЗО на \unit{30 milliampere}, 
		и УЗО на \unit{100 milliampere}, поэтому и пришлось возиться с задержкой.

		А теперь, когда стало понятно, КАК работает УЗО самое время сказать про 
		заземление, будет ли работать УЗО, если в розетках нет заземляющего контакта? 
		Будет! С той лишь разницей, что если у стиральной машинки будет пробой 
		на корпус в сети с заземлением -- УЗО отключится сразу, так как 
		дифференциальный ток будет значительным (уйдёт с корпуса в заземляющий проводник). 
		А вот если в сети нет заземления, стиральная машинка будет, как партизан 
		в кустах, стоять с напряжением \unit{230 volt} на корпусе, и УЗО отключится, 
		только когда ток будет протекать через человека. То есть наличие заземления 
		повышает безопасность, но не является обязательным условием для 
		функционирования УЗО.
	\stopsubject

	\startsubject[title={Возвращаемся в реальный мир. Почему могут быть ложные срабатывания}]
		Одной из причин непринятия УЗО электриками старой закалки являются ложные 
		срабатывания. И ложные срабатывания (при условии, что устройство исправно) 
		могут быть только по одной причине -- есть утечка, и она ощутима. 
		А вот причины появления утечек разнообразные:

		\placefigure[here][leakage]{Пути утечки тока в PE проводник в исправном 
				состоянии. Сверху показана паразитная ёмкость между проводниками в кабеле.
				Внизу изображена схема типового фильтра электромагнитных помех, который 
				содержит любое электронное устройство.
				}{\externalfigure[leakage.svg][width=\textwidth]}

		\startitemize[n]

			\item Изоляция может быть нарушена. Если кабель старый, открытый солнцу, 
			то в  изоляции могут появиться трещины. Чуть намочим -- и имеем 
			непредсказуемую величину утечки.

			\item Штатная утечка в оборудовании. Даже в исправном оборудовании есть 
			некоторая величина утечки, причём при переменном токе не нужен 
			непосредственный контакт, достаточно ёмкостной связи. На рисунке~\infig[leakage]
			изображены два пути утечки -- через погонную ёмкость с PE проводником в кабеле
			и через конденсаторы в фильтре электромагнитных помех, которые есть 
			почти в любом электронном устройстве. Специальным прибором можно 
			измерить величину фактической утечки в линии со всеми подключёнными 
			устройствами. Если прямое измерение не доступно, можно воспользоваться 
			эмпирическим правилом (7.1.83 ПУЭ) -- считать, что на каждый \unit{1 ampere} 
			потребления тока прибором будет \unit{0,4 milli ampere} утечки, 
			а также \unit{0,01 milli ampere} утечки на каждый метр длины фазного проводника. 
			(Цифры очень сильно усреднённые, как средняя температура по больнице, 
			но хоть что-то)  Желательно, чтобы {\bf сумма всех утечек в цепи при штатной 
			работе не превышала 1/3 номинальной величины отключающего дифференциального 
			тока.} \attention{}  Ну и как вишенка на торте -- если на УЗО написано, что 
			отключающий дифференциальный ток \unit{30 milliampere}, это значит, что 
			при \unit{30 milliampere} оно точно отключится. А точно не будет отключаться 
			при половине этого тока -- \unit{15 milliampere}. А вот при значении дифференциального тока между этими значениями -- как повезёт.\footnote{ГОСТ МЭК 61008-1 п.5.3.4} 
			Если у вас стоит УЗО на \unit{30 milliampere}, 
			и в розетки воткнута куча устройств, может получиться суммарный ток утечки в \unit{20 milliampere}, и создаётся ситуация, 
			когда УЗО может самопроизвольно отключиться без видимых причин.

			\item Ошибка монтажа, и где-то (например, в одном из подрозетников)  
			присутствует соединение рабочего нейтрального проводника N и заземляющего 
			PE, или они перепутаны.
			\stopitemize

	\stopsubject

	\startsubject[title={Противопожарные УЗО? Они все противопожарные!}]
		Если открыть каталог производителей, можно заметить, что УЗО выпускаются 
		на разные дифференциальные токи. Если с причиной выбора тока в \unit{30 milliampere} 
		все понятно, с \unit{10 milliampere} тоже в принципе можно догадаться 
		(ещё более чувствительные устройства для более чуткой защиты), то зачем 
		нужны устройства с током \unit{100 milliampere} и даже \unit{300 milliampere}? 
		Человек же при таких токах умрёт!

		Такие УЗО часто называют "противопожарными", так как в силу большого 
		дифференциального тока защиту человека от поражения электрическим током 
		они обеспечивают слабо, а вот функцию защиты при повреждении изоляции 
		всё ещё выполняют. Если изоляция будет нарушена, и при контакте с другим 
		проводником загорится электрическая дуга, то начнётся обугливание изоляции 
		и выделение тепла, что может поджечь горючие материалы вокруг. Если  
		"повезёт", и ток в дуге будет небольшим, то автоматический выключатель 
		не сработает. А вот выделение тепла и температура могут быть достаточными 
		для пожара. Конечно, потом огонь нарушит изоляцию, произойдёт короткое 
		замыкание и автоматический выключатель сработает, только огонь это уже не погасит.

	\startsubject[title={Да будет срач!}]
		Отдельная дисциплина споров -- какое УЗО лучше, электромеханическое или 
		электронное. В электромеханическом УЗО для отключения используется 
		энергия дифференциального тока, поэтому оно может сработать при обрыве 
		нулевого проводника, да и в целом не содержит нежной электроники, но 
		содержит нежную механику. Электронное УЗО требует питания для работы 
		электронного усилителя, поэтому при обрыве нуля работать перестаёт, 
		часто не отключая цепь. У каждой конфигурации есть свои достоинства и 
		недостатки. А для защиты от обрыва нуля я настоятельно рекомендую 
		ставить реле контроля напряжения.

%https://download.schneider-electric.com/files?p_enDocType=White+Paper&p_File_Name=998-20482406_GMA-US+_Why+to+choose+B+type+protection+for+safe+and+efficient+people+protection.pdf&p_Doc_Ref=998-20482406_GMA-US
		Но так как большинство читателей ждёт от меня конкретного ответа -- 
		скажу, что это не важно. Есть требования стандартов, есть требуемые 
		характеристики, и конкурентная цена в конце концов. Поэтому производитель 
		даёт ровно то, что от него требуют, а вот как получено желаемое -- не 
		так важно. А если производитель рукожоп, то отсутствие электроники 
		автоматически не означает, что изделие выйдет годным. Кроме того, УЗО 
		типа B без добавления электроники изготовить не получилось ни у одного 
		производителя.

		Стоит отметить, что критика надёжности электронных УЗО в среде специалистов
		 была в 1990-2000 годах, когда это был новый продукт и переживал 
		«детские болезни», в том числе несовершенной элементной базы. Но как 
		утихли споры о вреде излучения при разговоре по мобильному телефону, 
		так и споры «электромеханическое или электронное» можно прекращать. 

		Для контроля исправности УЗО на передней панели есть кнопочка "тест", 
		которая создаёт утечку в обход одной из обмоток через резистор и имитирует появление дифференциального 
		тока. Если УЗО при нажатии на кнопку "тест" отключилось -- то оно исправно. 
		Проверку исправности УЗО производители рекомендуют производить ежемесячно 
		(какие оптимисты!), ну или я реалистично говорю о тесте раз в полгода.
	\stopsubject

	\startsubject[title={Когда нельзя никому доверять}]
		Некоторые устройства, например бойлеры, могут быть весьма опасны при поломке, 
		если в электрощите нет УЗО. Поэтому в некоторых странах при сертификации требуют, чтобы устройство имело свое УЗО, размещенное на кабеле питания сразу после вилки.
		

		\placefigure[here][devicercd]{УЗО на проводе питания устройства.
			}{\externalfigure[deviceRCD.jpg][width=\textwidth]}

		\placefigure[here][boilerheaterfail]{Прогнивший, потерявший герметичность ТЭН из бойлера, не защищённого УЗО. При наличии УЗО оно бы сработало на ранней стадии и пришлось бы диагностировать проблему, но этот ТЭН ещё долго работал, пока не сгнила спираль, а людей в ванне щипало током.
			}{\externalfigure[boilerheaterfail.jpg][width=\textwidth]}

		Часто это дополнительное УЗО для устройства в вилке или в виде коробочки на 
		шнуре. Если покупатель подключит бойлер пластиковыми трубами и не выполнит заземление корпуса согласно инструкции, то при потере герметичности ТЭНа электричество по воде в 
		трубах может потечь через человека в заземлённую ванну. Такое УЗО защищает 
		только одно устройство, и в некоторых странах существую нормативы, 
		обязывающие добавлять УЗО на некоторые типы устройств. Как можно 
		заметить, устройство также содержит кнопку "тест" для проверки 
		работоспособности защиты.

		На рисунке~\infig[boilerheaterfail] приведена фотография  ТЭНа, с которым
		автору пришлось столкнуться. Бойлер не имел заземления, был включён в сеть без УЗО и своего
		УЗО не имел. Со временем из-за коррозии ТЭН потерял герметичность, вода
		попала внутрь, он распух, и вода стала контактировать непосредственно
		с нагревающей нихромовой спиралью. Эти места на фотографии дополнительно 
		показаны вставками с крупными планами. Бойлер продолжал работать, но люди
		жаловались, что от воды "щиплет" электрическим током. Так бойлер и работал,
		пока коррозия не испортила нагревательную спираль и ТЭН не ушёл в обрыв. К
		счастью, никто не пострадал. 


	\stopsubject

	\startsubject[title={УЗО или дифавтомат? (ВДТ или АВДТ?)}]
		Производители с заботой о нас\footnote{А, скорее, с желанием нам что-нибудь продать} объединили в одном корпусе два устройства --
		УЗО для защиты от поражения электрическим током и автоматический выключатель 
		для защиты от сверхтока, назвав это АВДТ -- Автоматический Выключатель 
		Дифференциального Тока. Продавцы скорее отреагируют на жаргонное 
		название "дифавтомат". Достоинств у такого гибрида не так много -- оно 
		компактное, и оно интуитивно понятное (один рычажок, а не два). А вот 
		недостатки есть:

		\startitemize[n]
			\item Оно лишает гибкости проектировщиков, например, поставить одно УЗО на группу и 
			несколько автоматических выключателей, или, наоборот, один автоматический выключатель
			на группу и индивидуальные УЗО на линии.

			\item Оно усложняет поиск неисправности, так как обычно на АВДТ отсутствует раздельная
			индикация и сложно понять, почему оно отключилось (варианты: сработал 
			тепловой расцепитель, электромагнитный расцепитель, или электромагнит от 
			дифференциального тока)

			\item Запихивание нескольких устройств в компактный корпус всегда 
			заставляет разработчиков идти на компромиссы.
		\stopitemize

		На мой личный взгляд, применение АВДТ оправдано только при модернизации  
		электрощитка, когда места внутри нет, а дифференциальную защиту хочется. Тогда 
		можно вынуть автоматические выключатели шириной один  модуль и воткнуть 
		АВДТ шириной один модуль, и перекоммутировать провода. Щиток в таком 
		случае расширять не придётся. В остальных случаях, по моему мнению, 
		предпочтительнее комбинация УЗО+автоматический выключатель.
	\stopsubject

	\startsubject[title={Я умер. Почему УЗО не спасло?}]
		УЗО не панацея, а лишь дополнительная мера защиты. Но лучше пока ничего не придумали. 
		Если взяться одной 	рукой за фазный проводник, а второй рукой за нулевой, 
		то для электросети 	вы будете лишь очередным нагревателем, дифференциальный
		ток не появится и УЗО не сработает. Также если сунуть палец в патрон лампы -- ток 
		потечёт через палец, но утечки в землю не будет, УЗО не отключится. 
		Поэтому даже наличие такой защиты не означает, что можно терять 
		бдительность и осторожность. Опытный электрик даже жену не берёт 
		одновременно за две груди :)
	\stopsubject
\stopchapter

\startchapter[title={Устройства защиты от дугового пробоя}]
	Устройства защиты от дугового пробоя (УЗДП -- формулировка из ГОСТ), они же 
	Устройства защиты от искрения (УЗИс) они же  arc-fault detection device (AFDD), 
	они же  arc-fault circuit interrupter (AFCI).... Имён много, а суть одна: 
	это устройство призвано отключить линию, если обнаружится дуговой пробой где-то на линии.

	Представим, в вашей электропроводке случилось неладное -- мыши погрызли изоляцию,
	ослабла клемма, или в месте перегиба кабеля переломились жилы. Эти неисправности, как и ряд других, могут привести к дуговому пробою.

	Дуговой пробой происходит, когда два проводника оказываются на очень 
	маленьком расстоянии друг от друга, из-за чего проскакивает искра, 
	зажигается электрическая дуга, и электрический ток течёт уже по "по воздуху". 
	Электрическая дуга очень горячая, и за мгновения может зажечь горючие материалы 
	вокруг, обуглить изоляцию и наделать бед. При этом обуглившаяся изоляция становится 
	проводником, что сильно упрощает повторное зажигание дуги.

	Различают параллельный и последовательный дуговой пробой. Параллельный 
	дуговой пробой -- происходит когда дуга зажигается между проводниками L и N, или L и PE, 
	например, из-за ввёрнутого в кабель самореза или из-за придавленной ножкой стула изоляции.
	В таком случае, скорее всего, параллельный дуговой пробой 
	перерастёт в короткое замыкание, и сработает защита от сверхтока. 
	Последовательный дуговой пробой означает, что дуга горит в разрыве цепи последовательно 
	с нагрузкой, и он оказывается наиболее сложно обнаруживаемым, и потому самый опасный. Ни УЗО, ни автоматический выключатель при этом 
	не сработают! Нет условий для срабатывания этих видов защиты -- ток не превышен 
	(его величину ограничивает нагрузка), дифференциального тока тоже нет. Дуга 
	будет гореть, пока контакт случайно не восстановится или разорвётся. Впрочем, 
	наверняка вы с ней уже сталкивались -- это то самое "шкворчание" плохого 
	контакта в выключателе или розетке.

	\placefigure[here][arcingtypes]{Последовательный и параллельный дуговой пробой в
			кабеле, по причине пережатых жил ножкой табуретки и переломанной жилы в месте изгиба.
			}{\externalfigure[parallel-series-ark.svg][width=\textwidth]}

	Если ваша электропроводка в помещении выполнена в строгом соответствии со всеми нормативами, 
	то дуговой пробой не вызовет пожара, но породит потоки брани электрика, 
	который будет ремонтировать розетку, где из подрозетника торчат два 
	обугленных пенёчка проводов.

	Ключевое слово здесь "если". К сожалению, в суровой реальности может быть:
	
	\startitemize[packed]

		\item Старая алюминиевая проводка, которая ремонтировалась не пойми как и не пойми где.
		
		\item Потрескавшаяся от старости или солнца изоляция кабелей.


		\item Проводка, уложенная внутри сгораемых стен.

		\item Грызуны, сожравшие изоляцию проводов до голой меди.

		\item Горе строители, повредившие изоляцию проводов ввёрнутым саморезом.

		\item Огромное количество переносок, тройников и других электроизделий сомнительного 
		качества, лежащих в труднодоступных местах в окружении горючих предметов.
	\stopitemize

	При несчастном стечении обстоятельств дуговой пробой может вызвать пожар 
	с жертвами.

	Выходит опасная ситуация: при раздолбайском отношении к обслуживанию электрохозяйства мы 
	можем получить явление, способное привести к пожару, и которое ни одно из 
	используемых средств защиты обнаружить не может.

	\startsubject[title={Ловим призрака за хвост}]
		Инженеры до сих пор находятся  в поисках надёжного способа обнаружения 
		дугового пробоя. Если полистать публикации в научных журналах, то 
		можно увидеть попытки исследователей использовать разные методики, 
		включая модные нейронные сети. Чем лучше методика, тем выше вероятность 
		обнаружения дугового пробоя и ниже количество ложных срабатываний.

		\placefigure[here][AFDImetod]{Вырезка со сравнением разных методов обнаружения
			дугового пробоя. Взято из Hien Duc Vu. Arc fault detection with machine learning. Engineering Sciences [physics]. Université
de Lorraine, 2019. English. 
			}{\externalfigure[AFDImetod.jpg][width=\textwidth]}

		При этом устройству в электрощите доступен всего лишь один способ 
		обнаружения дугового пробоя -- анализ величины и формы тока, 
		потребляемого нагрузкой. Все производители модульных устройств защиты от 
		дугового пробоя снимают сигнал с датчика тока, но обрабатывают данные по-разному,
		и не раскрывают подробностей, ссылаясь на ноу-хау. Поэтому я 
		могу лишь рассказать общие подходы, которые раскрыты в научных 
		публикациях, а вот в охоте за подробностями придётся ловить и спаивать 
		разработчиков в баре.

		Обнаружить дуговой пробой можно из-за одной особенности -- дуга 
		зажигается не сразу. Напряжение должно вырасти до напряжения пробоя, 
		после чего в зазоре проскакивает искра, которая ионизирует воздух и 
		позволяет устойчиво загореться электрической дуге. А так как у нас в 
		сети переменный ток, и ток меняет направление 50 раз в секунду, переходя 
		через нулевое значение, то дуга загорается и гаснет 100 раз в секунду, 
		приводя к специфическим искажениям формы протекающего тока!

		Звучит непонятно? Давайте посмотрим глазами. Для этого я сделал небольшой 
		стенд и сделал снимки экрана осциллографа -- прибора, который может 
		записывать и показывать на экран график изменения напряжения во времени.
		Ток в цепи измеряется трансформатором тока (голубая линия), напряжение -- через делитель 
		(жёлтая линия), масштаб в данном случае не важен. Графики для почти 
		идеальной нагрузки -- тепловентилятора -- показаны на рисунке~\infig[heatercurrform].

		\placefigure[left][heatercurrform]{Форма тока потребляемого тепловентилятором.
			}{\externalfigure[heatercurrform.jpg][width=0.5\textwidth]}

		Все просто, растёт напряжение в линии -- пропорционально растёт ток. 
		Напряжение падает -- ток в цепи падает. Обратите внимание в месте 
		перехода напряжения через ноль -- ток растет сразу. А на рисунке~\infig[AFDIstep]
		график 	тока в той же цепи, если я развожу контакты прерывателя и вызываю дуговой 
		пробой последовательно в цепи. Появляется ступенька -- ток появляется 
		только после того, как напряжение достигнет напряжения пробоя зазора 
		между проводниками.

		\placefigure[right][AFDIstep]{Форма тока с последовательным дуговым пробоем
			}{\externalfigure[arcing.jpg][width=0.5\textwidth]}

		Можно подумать, что достаточно просто следить за тем, есть ли ступенька 
		в потреблении тока при переходе напряжения через ноль. Но, увы, этот 
		способ не работает, поскольку такая ступенька появляется у многих видов 
		нагрузки. Например, у устройств с регулировкой мощности 
		тиристорным регулятором, который такую ступеньку создаёт, и, меняя её 
		ширину, регулирует суммарную потребляемую мощность в нагрузке. На рисунке~\infig[regulatorcurr], 
		показан график тока у пылесоса с регулятором мощности.\footnote{Более 
		детально углубляться в этот механизм не будем, подробнее можно прочитать 
		поискав по ключевым словам "фазовая регулировка  мощности". Отмечу лишь, 
		что он весьма часто применялся в конце ХХ века и продолжает применяться 
		для мощных активных нагрузок, поскольку обеспечивает практически идеальный КПД,
		правда генерируя изрядное количество помех в электросеть.}

		\placefigure[left][regulatorcurr]{Форма тока, потребляемого пылесосом с тиристорным регулятором мощности.
			}{\externalfigure[regulatorcurr.jpg][width=0.5\textwidth]}

		Кроме того, идеальный случай, когда в линии всего одна нагрузка, 
		встречается редко. Чаще на линии несколько потребителей, и их токи 
		суммируются. В итоге график начинает выглядеть совершенно ненаглядно. 
		На рисунке~\infig[currsumm] приведён график тока и напряжения цепи, в 
		которой четыре потребителя: обогреватель \unit{1 kilo watt}, 
		электрочайник \unit{2 kilowatt}, пылесос с регулятором на половинной 
		мощности (примерно \unit{800 watt}) и мощный импульсный блок питания, 
		нагруженный на балласт (примерно \unit{180 watt}). На графике слева нет дугового 
		пробоя, а график справа показывает последовательный дуговой пробой обогревателя на \unit{1 kilowatt}, 
		то есть, ток дуги составляет только четверть от всего потребляемого тока.
		
		\pagebreak[yes]
%Костыль, иначе сноска перекрывает текст, не нашел как побороть

		\placefigure[here][currsumm]{Формы суммарного тока нескольких нагрузок, при
			нормальной работе и при дуговом пробое последовательно одной из нагрузок.
			}{\externalfigure[currsumm.jpg][width=\textwidth]}

		Что делать? Посмотрим внимательно на график с искрением -- скорость 
		нарастания тока в цепи после пробоя огромная, ступенька практически 
		вертикальная! А значит, нужно смотреть не на появление ступеньки, 
		а на её отвесность, или, говоря научно, на производную тока (скорость нарастания). Проще всего это сделать, анализируя спектр сигнала; 
		чем отвеснее ступенька, тем шире её спектр. Спектрограммы тока в цепях с дуговым пробоем и без приведены на рисунке~\infig[spectre].

		\placefigure[here][spectre]{Спектры токов при нормальной работе и при дуговом 
			пробое. Из работы Arc Faults Circuit Interrupter (AFCI) for PV systems. Technical
			White Paper. Huawei Technologies Co. 2020.
			}{\externalfigure[realspectre.jpg][width=\textwidth]}

		\placefigure[here][spectreprop]{Характерные искажения при  искрении.
			}{\externalfigure[arcfeatures.svg][height=\textheight]}

		В результате принцип работы защиты прост -- постоянно анализируем спектр 
		сигнала с датчика тока. Если вдруг он резко изменяется,  определяем, как 
		он изменился. Если наблюдаем подъём в высокочастотной части спектра -- значит, 
		это дуговой пробой, и отключаем нагрузку. Правда, в реальности есть нюансы....
	\stopsubject

	\startsubject[title={Ложные срабатывания и шапка невидимка}]
		Ложные срабатывания -- головная боль разработчиков УЗДП. В электросети 
		творится полная анархия, каждая нагрузка потребляет ток как хочет, 
		некачественные устройства ещё активно создают помехи.

		На рисунке~\infig[deadrotor] ток, когда я просто включил 
		шлифмашинку с умирающим двигателем.

		\placefigure[left][deadrotor]{Форма тока, потребляемая шлифмашинкой с 
			коротким замыканием в обмотке ротора.
			}{\externalfigure[deadrotor.jpg][width=0.5\textwidth]}

		А на рисунке~\infig[arcweld] записан ток сварочного аппарата (я взял 
		обычный трансформатор и варил скрутку угольным электродом).

		\placefigure[right][arcweld]{Форма тока, потребляемая трансформатором при
			сварке угольным электродом.
			}{\externalfigure[arcweld.jpg][width=0.5\textwidth]}

		При этом формально устройство не должно сработать -- дугового пробоя нет. 
		А теперь представьте, что у вас таких устройств на одной линии с десяток -- 
		их токи сложатся, шумы просуммируются, а разработчик может уйти в запой, от безнадежности решения задачи точного обнаружения дугового пробоя.

		Получается довольно нетривиальная задача, с одной стороны, нужно повышать 
		чувствительность, а с другой -- не допускать ложные срабатывания. 
		Поэтому разработчики не спешат раскрывать свои хитрые алгоритмы. 
		Единственное подробное  описание одного из таких алгоритмов работы я нашёл в документе "5SM6 AFD Unit
		Technology Primer" от Siemens.


		И тут важно отметить: ни одно УЗДП не застраховано от ложных срабатываний! 
		Более того, из всех устройств защиты, УЗДП, наверное, единственное, 
		которое {\bf может дать ложное срабатывание в исправном состоянии.} \attention{} 
		Это важно помнить при проектировании! (примеры приведены в конце главы). 
		Например, найдётся гад, вроде меня, который откопает старую советскую 
		лампу УФО-Б (ртутная дуговая лампа высокого давления с резистивным балластом) 
		и включит её в сеть. График тока через лампу в процессе  розжига, до выхода на номинальный режим, изображён на рисунке~\infig[UFOB].

		\placefigure[left][UFOB]{Форма тока при розжиге ультрафиолетовой лампы УФО-Б.
			}{\externalfigure[ufolamp.jpg][width=0.5\textwidth]}



		В лампе происходит дуговой пробой на поджигающем электроде, и лампа 
		вызывает ложное срабатывание при каждом включении! При том что для неё 
		это абсолютно штатный режим запуска. Такие проблемные 
		устройства отыскать было трудно, но у меня получилось. В процессе тестов 
		УЗДП отечественных производителей я пробовал разные виды нагрузок и нашёл свою Ахиллесову пяту на каждую 
		модель УЗДП. Впрочем, подавляющее большинство бытовых устройств проблем 
		не вызывает.

		Любое государство не терпит анархии, и поэтому с ней борется. Во многих 
		странах есть требования по электромагнитной совместимости для устройств -- 
		они не должны мешать работе других устройств в электросети. Поэтому 
		мощность и спектр помех, которые могут просачиваться с устройства 
		обратно в сеть, законодательно ограничивается. Следствием этого стала установка фильтров помех 
		в устройства. Фильтр ослабляет высокочастотные помехи, которые генерирует 
		устройство. Например, любой импульсный блок питания имеет в своей схеме 
		такой фильтр, на рисунке~\infig[psfilter] приведена схема типового импульсного
		блока питания.

		\placefigure[here][psfilter]{Схема импульсного блока питания с сайта oshwlab.com за авторством 
				пользователя abdouu. Фильтр я обвёл пунктиром.			
			}{\externalfigure[filterinPSW.svg][width=\textwidth]}

%https://oshwlab.com/abdouu/switch-mode-power-supply
		Сетевой фильтр является шапкой-невидимкой: всё, что происходит за ним, 
		становится невидимым для УЗДП. Технически, в схеме сетевого фильтра, помимо стандартной LC-схемы на дросселях и паре конденсаторов, 
		можно использовать разделительный трансформатор. По этой причине мой 
		эрзац-сварочный аппарат для сварки скруток не вызывал ложных срабатываний -- 
		дуговой пробой был во вторичной обмотке, трансформатор за счёт индуктивности работал 
		как сглаживающий фильтр. Добавление простого фильтра, например,  
		в виде синфазного дросселя, взятого из микроволновки, полностью устранило проблему ложного 
		срабатывания с лампой УФО-Б которое я описал выше.

		Отсюда следует, что вероятность ложных срабатываний резко возрастает, 
		если в сеть включается устройство, у которого:

		\startitemize[n]
			\item Нет таких фильтров, просто потому что, например, разработки 1960х 
			годов, когда требования были попроще.

			\item Фильтры есть, но не эффективны из-за кривой схемотехники или экономии. 
			Этим часто грешат noname-устройства, где для экономии выбрасывается всё, 
			что отвечает за качество или безопасность. Хороший фильтр, как правило, материалоёмок, а от того дорог и тяжёл.
		\stopitemize

		Выходит, что качественные, соответствующие современным требованиям 
		электроустройства для УЗДП проблем доставлять не должны. Если же у вас 
		есть одно такое проблемное устройство (например, любимая электробритва 
		дедушки), то его можно "скрыть" от УЗДП шапкой-невидимкой в виде 
		дополнительного сетевого фильтра. Специализированные фильтры выпускаются 
		готовыми узлами и продаются в магазинах радиодеталей, хотя, надеюсь, 
		у производителей УЗДП появится такое изделие, как опция.

		Я думаю, что многих беспокоит вопрос: а не срабатывает ли УЗДП на сварку? 
		Нет, я опробовал несколько инверторных сварочных аппаратов -- всё в порядке.
	\stopsubject

		\startsubject[title={Из крайности в крайность}]
		Противоположной проблемой является потеря чувствительности на длинных 
		линиях. Любой кусок проводника обладает собственной индуктивностью и 
		распределённой ёмкостью. Если у нас есть длинная линия, то  на рисунке~\infig[longline] 
		изображено как будет отличаться идеальное представление от реального.

		\placefigure[left][longline]{Реальная линия электропередачи имеет распределённую 
			ёмкость, сопротивление и индуктивность.
			}{\externalfigure[inreallife.svg][width=0.6\textwidth]}


		Длинная линия сама начинает работать как сетевой фильтр, и 
		высокочастотная часть спектра затухает тем сильнее, чем длиннее линия. 
		Поэтому есть некая предельная дальность, на которой УЗДП способно 
		обнаружить дуговой пробой. Также фильтры, установленные в аппаратуре, 
		могут влиять на затухание сигнала вокруг себя, поэтому возможны 
		мистические случаи, что УЗДП видит пробой, когда фен с переломанным 
		проводом включен в розетку в одиночестве, и не видит пробой, если в 
		соседнюю розетку дополнительно включена стиральная машинка. Только у 
		одного из опробованных при подготовке книги производителей УЗДП есть в комплекте имитатор, который позволяет 
		не только проверить исправность УЗДП, но и определить, не потеряло 
		ли оно чувствительность из-за длинной линии. Поэтому, например, УЗДП может не 
		сработать из-за искрения в будке охраны, от которой до щита с устройствами 
		защиты пару сотен метров кабеля. Как правило, на линиях короче \unit{100 meter} 
		проблем не возникает. 
		\stopsubject

		\startsubject[title={Почему только сейчас?}]
		Если предохранители известны более сотни лет, автоматические выключатели 
		примерно столько же, УЗО -- полсотни лет, то УЗДП появились совсем недавно -- уже
		 в конце XX века. А всё потому, что без электроники обнаружение дугового 
		пробоя сделать невозможно. А относительно дешёвые микроконтроллеры, на 
		которых можно реализовать цифровую обработку сигналов, появились совсем 
		недавно. Вот и получается, что только сейчас, стало возможным не только 
		технически реализовать такой вид защиты, но и сделать это по цене, 
		доступной частным лицам. 

		Законодательство тоже активно меняется -- новое устройство вносят в 
		различные правила и нормы, делая обязательным к применению в некоторых 
		задачах. На момент написания книги, в России -- УЗДП начали появляться в документах начиная с ГОСТ IEC 62606-2016, который является 
		переводом стандарта МЭК 62606. Собственно, стандарт не только определяет 
		требуемые характеристики УЗДП и методику тестирования, но и, наконец, 
		само название этого типа устройств -- УЗДП.
		\stopsubject

		\startsubject[title={Куда включать?}]
		УЗДП не самостоятельное устройство -- обычно оно требует отдельного 
		автоматического выключателя. Производители в погоне за нашими 
		кошельками и компактностью могут совмещать УЗДП с автоматическим 
		выключателем -- такой гибрид уже можно использовать самостоятельно. 
		При использовании нескольких типов устройств защиты, последовательность 
		соединения не влияет на работоспособность устройств, но учитывая возможное влияние индуктивности трансформатора в УЗО на чувствительность, УЗДП желательно устанавливать после УЗО.

		Обратите внимание, у некоторых моделей УЗДП ввод сделан {\it снизу}, причём 
		это не придурь разработчиков, и встречается и у именитых западных 
		производителей. Я уверен, конструкторы до последнего старались сделать 
		всё, как все привыкли, но что-то помешало.

		Типовая схема включения УЗДП изображена на рисунке~\infig[afdiinstall].

		\placefigure[here][afdiinstall]{Схема включения УЗДП. Установка УЗО не обязательна. Внимательно изучите документацию,
		УЗДП могут иметь ввод как снизу, так и сверху. Фазировка УЗДП и УЗО обычно обозначена на корпусе.
			}{\externalfigure[AFCI_connect.svg][width=\textwidth]}

		Учитывая ненулевую вероятность ложных срабатываний, имеет смысл 
		использовать несколько УЗДП и разделить линии по типу нагрузки -- условно, 
		стационарные и переменные. В стационарные включить потребители, 
		профиль потребления тока которых не меняется годами -- насосы циркуляции, 
		холодильники, вентиляция и т.п.  Внезапное срабатывание УЗДП на такой 
		группе, скорее всего, будет сигнализировать о реальной проблеме. В 
		переменные стоит отнести все розетки, в которые втыкают постоянно что 
		попало -- блендеры, чайники, пылесосы, освещение и т.п. Срабатывание 
		УЗДП на этой линии должно настораживать, но его значительно проще 
		связать с новым прибором в сети.

		В идеальном мире, конечно же, каждая группа розеток имеет свою линию, свой автомат и УЗДП, но 
		учитывая цены и средние зарплаты -- это мечта. Но одно УЗДП на целый 
		частный дом может создать много проблем: как в случае его срабатывания 
		искать место проблемы? Поэтому хоть какое-то разделение на группы стоит 
		предусмотреть.

		Отдельной осторожности требует использование УЗДП на линиях с важными 
		нагрузками, отключение которых может наделать бед не меньших, чем дуговой 
		пробой. Циркуляционные насосы, сетевые коммутаторы и т.п. Более того, в 
		стандартах явно запрещают использовать УЗДП для некоторых потребителей, 
		например с аппаратами искусственной вентиляции лёгких.
		\stopsubject

		\startsubject[title={Искрит у соседа, а отключается у меня.}]
		К сожалению, такое возможно с некачественными УЗДП. Хоть УЗДП 
		анализирует ток нагрузки, и, казалось бы, оно должно быть слепо ко всему, 
		что происходит до него, возможно описанное в заголовке явление. Линии электропередач -- неидеальный источник тока, 
		и обладают внутренним сопротивлением. Поэтому на длинных линиях искрение 
		мощной нагрузки вызовет заметные колебания напряжения питания, что, в 
		свою очередь, вызовет колебания тока потребления (весьма солидные, если 
		нагрузка нелинейная). Это называется перекрёстными помехами. 
		Разработчики принимают меры, и различными приёмами снижают чувствительность 
		к перекрёстным помехам с переменным успехом.
		\stopsubject

		\startsubject[title={Оно сработало -- дальше что?}]
		Наверное, самый интересный вопрос. Я уверен, при срабатывании защиты 
		большинство людей просто пойдёт и включит всё обратно, не попытавшись разобраться 
		в причинах. Но мы же не такие?)

		Если сработало УЗДП -- значит, была определенная причина, ложная или нет, и желательно попытаться её 
		найти. Задача упрощается, если при включении УЗДП снова отключится, 
		значит, проблема устойчива. Используя автоматические выключатели 
		(теперь вы понимаете, что чем детальнее были разбиты потребители на группы -- тем 
		проще искать проблему?), последовательно включаем группы. Если при 
		подключении очередной группы, например, "гараж", УЗДП снова срабатывает -- 
		начинаем искать проблему уже там. Поиск неисправности может быть нудным, 
		но, в общем-то, он ничем не отличается от поиска причин срабатывания любого 
		другого устройства защиты, например УЗО.

		Если при включении УЗДП повторного отключения не происходит,  достаточно 
		провести профилактический осмотр: все ли розетки целы, нет ли оплавлений 
		и потемнений на пластике.\footnote{Каждую вилку нужно вынуть из розетки и осмотреть.
		При нагреве пластик вокруг контакта меняет цвет} Можно включить напряжение обратно и внимательно 
		послушать -- плохой контакт иногда слышно по характерному гудению и "шкворчанию", треску. 
		Проведите осмотр гибких шнуров и переносок на предмет повреждений изоляции. 
		Часто жилы переламываются в месте изгиба сетевого шнура при входе в корпус.
		Для выявления этого дефекта у включенного устройства делают 2-3 перегиба шнуром по кругу.
		Устройство при этом  должно стабильно работать. Если при перегибах
		например электродрель дёргается и заикается, то шнур придётся обрезать ниже дефекта
		и подключить заново.
		

		Теперь очевидно: чем более развито деление на группы потребителей,  тем 
		меньше работы по локализации проблемы. Одно дело проводить осмотр ВСЕЙ 
		электрики дома, так как УЗДП одно, и другое дело проводить осмотр детской 
		комнаты, если сработало УЗДП на детскую.
		\stopsubject

		\startsubject[title={Ещё функции, причём бесплатно}]
		Если УЗДП имеет в своём составе довольно продвинутые электронные "мозги" 
		для выполнения основной функции, то почему бы не добавить ещё функций с 
		минимальными изменениями железа? Почти все отечественные УЗДП, что я тестировал, имеют 
		функцию защиты от превышения напряжения -- если напряжение в сети 
		повысится выше нормативного, например, из-за отвалившегося "нуля" 
		прилетело не \unit{230 volt}, а все \unit{400 volt}, то УЗДП также штатно 
		отключится. Увы, когда напряжение придёт в норму -- оно обратно не 
		включится из-за механизма свободного расцепления. Таким образом, 
		использование некоторых моделей УЗДП позволяет получить дополнительную 
		защиту от обрыва нуля практически даром. Оговорки: автоматического 
		повторного включения не предусмотрено ГОСТом -- когда напряжение нормализуется, 
		автоматически ничего не включится. Защиты от пониженного напряжения  
		у многих моделей УЗДП тоже нет.
		\stopsubject

		\startsubject[title={Оно ещё и самотестируется?!}]
		Да, если присмотреться к расшифровке показаний индикаторов на фасаде УЗДП, 
		то можно увидеть вариант "УЗДП неисправно". Устройство содержит в своём 
		составе дополнительные цепи, позволяющие подать на вход самому себе
		образцовый сигнал и удостовериться, что сигнал воспринимается как положено. 
		При этом проверяется исправность аналоговой части прибора, но не 
		проверяется исправность механизма расцепления, это бы привело к 
		самоотключению, что непростительно.

		
		\stopsubject
	
		\startsubject[title={Критика}]
		Для объективности стоит сказать, что у повсеместного использования УЗДП 
		есть и критики. Наиболее весомым является аргумент, что роль дугового 
		пробоя, как первопричины пожара, неоднозначна. При нагреве проводников от 
		перегрузки по току дуговой пробой образуется на поздних стадиях плавления 
		токопроводящей жилы, когда изоляция от нагрева вовсю уже дымится и стекает. 
		Срабатывание УЗДП в таком случае уже пожар может не предотвратить. 
		Остаётся открытым вопрос, что является основной причиной пожара: возгорание от перегрузки 
		(которое должны предотвратить автоматические выключатели и предохранители), 
		или всё-таки дуговой пробой. 

		Моё мнение иллюстрируется фразой "Если вы пытаетесь автоматизировать 
		бардак -- вы получаете автоматизированный бардак". Если электрохозяйство 
		находится в состоянии, когда провода вываливаются из клемм, то УЗДП не станет 
		панацеей (хотя наверняка будет постоянно срабатывать и нервировать электриков, 
		и, возможно, заставит найти проблемные места). Многие уже привычные нам 
		меры безопасности, вроде ремней в автомобиле, тоже внедрялись со скрипом 
		и находили своих критиков, весьма убедительно высказывавшихся в ненужности 
		и избыточности таких мер :) Если повсеместное внедрение УЗО объективно 
		снизило количество смертей от поражения электрическим током, будем надеяться, что 
		широкое внедрение УЗДП как-то уменьшит статистику пожаров по причине 
		неисправности электропроводки.

		Впрочем, личное мнение какого-то автора в интернете не отменяет 
		нормативных требований.
		\stopsubject

		\startsubject[title={Тесты УЗДП}]
		По моей просьбе производители прислали мне на растерзание свои модели УЗДП, и в 
		2021 году я протестировал все доступные в 2021 году отечественные УЗДП, о результатах
		поделился в публикации на своём сайте. Я надеюсь, что книга будет оставаться актуальной дольше, чем одна из статей, хоть и составленных с любовью,
		поэтому итогов здесь не привожу, устройства постоянно дорабатываются.
		Тем не менее, стоит упомянуть, что в 2021 году некоторые отечественные производители выпускают вполне качественный продукт. Поддержим отечественное
		производство!:) 
		\stopsubject

\stopchapter

\startchapter[title={Термоиндикаторные наклейки.}]
	Возможно, вы слышали шутку от электронщиков "Электроника -- наука о контактах". 
	Действительно, большое количество неисправностей связано с тем, что нарушен 
	контакт где-то в разъёме, или образовалась трещина в пайке, из-за чего устройство не работает. 
	Но электронщики не одиноки, плохой контакт в энергетике, где токи и напряжения 
	большие, сам о себе даст знать повышенным нагревом. Я уверен, что любой мой 
	читатель, даже не будучи связанным с техникой, хоть раз в жизни видел 
	оплавившийся обугленный контакт. 

	Повышенный нагрев любого соединения проводников, кроме случаев, когда это 
	заранее предусмотрено, прежде всего действует разрушительно на изоляцию. 
	Если нагрев будет чрезмерным, то возможно образование электрической дуги с 
	возгоранием того, что окажется рядом. К счастью,  человечество быстро делает 
	выводы, поэтому на сегодняшний день во всех странах мира действуют стандарты 
	разной степени строгости на электрическое оборудование. В том числе 
	регламентируется степень горючести корпусов электрических приборов, изоляции 
	проводников, да и сами щиты чаще всего делают из металла, что локализует 
	неприятности от раскалённых докрасна контактов. На рисунке-демотиваторе~\infig[brutalled]
	ниже как раз отлично видно последствия нагрева.

	\placefigure[here][brutalled]{Популярный демотиватор из интернета. Автор фотографии неизвестен.
			}{\externalfigure[demotivator.jpg][width=\textwidth]}

	Более того, практически для всех остальных причин появления нежелательного 
	нагрева в электрической цепи уже придуманы устройства защиты, с которыми вы 
	познакомились в предыдущих главах. Только нагревающиеся контакты до недавнего 
	времени не имели своих устройств выявления и защиты. А значит, защита строилась 
	пассивно, не на выявлении проблемных контактов, а на локализации последствий их 
	появления -- воздушных зазорах, негорючей изоляции, металлических щитах и т.д.

	\startsubject[title={Почему контакты становятся плохими и зачем за ними наблюдать}]
		Проблеме получения надёжного электрического соединения проводников посвящено 
		огромное количество научных работ. И можно сказать только то, что 
		надёжными являются только неразъёмные соединения, когда проводники 
		соединены намертво опрессовкой или сваркой, образуя монолит. Любая 
		техника и инженерные коммуникации иногда требуют ремонта и обслуживания, 
		поэтому вынужденно применяются разъёмные соединения. Не будешь же 
		отпиливать, а затем приваривать барахлящий выключатель. И такие соединения 
		иногда доставляют проблемы -- контакт может ухудшиться, и тогда ток, 
		протекая через него, приводит к повышенному нагреву. Длительный небольшой 
		нагрев ускоряет старение изоляции. Большой нагрев  может вызвать 
		плавление проводника с зажиганием электрической дуги. Любое из 
		последствий этого нежелательно -- как пожар в щитовой, так и просто 
		отключение критического оборудования.

		Производители всячески стараются улучшить ситуацию, используя разные 
		виды покрытий, насечек, прижимных пружин и прочих ухищрений, но на 
		сегодня ситуация такова:

		\startitemize[n]
			\item Даже идеально выполненное соединение с соблюдением всех технологических 
			требований со временем может ухудшиться, в силу агрессивности среды или 
			внутренних причин, вроде ползучести металла. Строгое соблюдение 
			требований к качеству монтажа уменьшает, но не исключает такую опасность.

			\item Регулярное изменение температуры соединения ускоряет процессы деградации. 
			Не важно,  меняется ли температура от изменений погоды или из-за 
			кратковременного протекания больших токов. Поэтому электрохозяйство вне 
			отапливаемых помещений требует особенного внимания.

			\item Процесс нагрева обладает положительной обратной связью. То есть, от 
			нагрева металл окисляется, от этого переходное сопротивление возрастает, 
			из-за этого нагрев усиливается, и так по нарастающей. Значит если 
			был нагрев, контакт со временем будет только ухудшаться.

			\item В зависимости от нагрузки оборудования, материалов, конструкции 
			контакта процесс превращения просто нагревающегося соединения в 
			брызгающую расплавленным металлом электрическую дугу может занимать 
			от часов до нескольких лет.
		\stopitemize

		Вывод довольно простой -- в щите любое из соединений может стать плохим, 
		и оно начнёт выдавать себя небольшим нагревом. Если это не заметить 
		вовремя, со временем ситуация станет только хуже и соединение будет греться сильнее. 
		Сильный нагрев может закончиться или разрушением цепи с последующим 
		ремонтом, или пожаром.

		Для своевременного выявления проблемных контактов в  электрических 
		сетях и оборудовании есть регламент -- регулярный осмотр, иногда с 
		проверкой моментов затяжки всех соединений. Если при осмотре будет 
		выявлено подозрительное соединение, то можно провести его профилактику 
		ДО наступления дорогих и опасных поломок с оплавлением и электрической 
		дугой. В зависимости от оборудования и объекта периодичность осмотра 
		может меняться, но часто не реже 2 раз в год. Осмотр часто проводится 
		без отключения оборудования, но с соблюдением положенных предосторожностей. 
		Если не верите автору -- послушайте вашего стоматолога, он подтвердит:
		профилактика всегда дешевле ремонта.
	\stopsubject	

	\startsubject[title={Человеческий фактор}]
		Как вообще можно увидеть плохой контакт, нагревающийся время от времени? 
		Опытный электрик может увидеть это по характерным имениям цвета изоляции от 
		нагрева, изменению блеска металла крепежа. У некоторых людей со стажем 
		появляется удивительная "чуйка", не только электриков. Например, мне 
		рассказывали про сотрудника целлюлозно-бумажной фабрики, который мог на 
		спор определить влажность бумаги с точностью в несколько процентов, просто 
		положив руку на пачку бумаги. После подтверждения  влажности лабораторией на 
		приборе довольный сотрудник уходил с выигрышем. Но мы не можем полагаться 
		на такое чутьё, из-за трудновоспроизводимых результатов. Да и не всегда 
		внутри электрических щитов всё хорошо освещено и чисто. Необходимо 
		использовать инструментальные методы, где результат мало зависит от 
		состояния самого электрика, но обеспечивается соблюдением определённых процедур. 

		Одним из таких способов является тепловизионный контроль. Тепловизор -- 
		это особая фото/видеокамера, оптика и сенсор которой позволяет ей видеть 
		в длинах волн порядка 7-14 мкм, то есть в инфракрасном диапазоне. На 
		экране прибора нагретые предметы будут выглядеть ярче, холодные -- темнее. 
		Способ невероятно эффективен, судите сами, вы без обучения и инструктажа 
		видите на рисунке~\infig[thermal] подозрительный контакт. 

		\placefigure[here][thermal]{Фотография стенда на тепловизор Seek Thermal.
			}{\externalfigure[thermalimaging.jpg][width=\textwidth]}

		Это как раз фотография стенда, который я собрал для испытаний наклеек 
		из главы. Сразу видно, как тепловизор раскрасил в ярко-соломенный цвет 
		объекты, температура которых аномально высока. Возможна даже автоматизация -- 
		просто поднимать тревогу, если в кадре появляется что-то нагретое выше 
		пороговой температуры.

		Способ давно и успешно используется на производствах, при обслуживании 
		зданий, но у способа есть свои недостатки, из наиболее значительных два:

		\startitemize[n]

			\item Тепловизор это штука дорогая. Прогресс, конечно, привёл к появлению 
			недорогих бытовых моделей, и в Китае освоили производство некоторых 
			моделей, но профессиональные приборы по-прежнему удовольствие не из дешёвых. 
			А так как тепловизор это устройство двойного назначения (угадайте, почему), 
			то их экспорт внимательно контролируется. 

			\item Тепловизор показывает температуру здесь и сейчас. Если контакт 
			нагревается только в определённые периоды времени, например, когда все 
			готовят себе обед, то пришедший после обеда электрик не увидит проблем, 
			так как контакт к тому времени остынет.
		\stopitemize

		Второй недостаток существенно замедляет процесс контроля, ведь если 
		делать всё как следует, то нужно создать в цепи нагрузку, подождать, пока 
		изменится температура, и только потом проводить осмотр. И если в небольшой 
		квартире можно включить обогреватель с чайником, неторопливо заварить чай 
		и после идти осматривать проводку в поисках проблемных распаечных коробок, 
		то как быть электрику, например, в школе, где линии идут в каждый класс и 
		во время уроков школьников беспокоить нельзя?
	\stopsubject

	\startsubject[title={Наклейки с памятью}]
		Способом решить проблему обнаружения контакта, который греется только 
		иногда, а не в момент, когда на него смотрят, будет использование 
		специальных термоиндикаторных наклеек. Такие наклейки нужно разместить 
		рядом с каждым контактом. Если хоть раз температура превысила пороговую, 
		они меняют цвет. Наклейки реализуют на разных физических принципах, но 
		наиболее популярны стали термоиндикаторы плавления.

		Идея достаточно проста -- на цветную подложку наносится состав из 
		частичек легкоплавкого вещества со связующим. Так как состав неоднороден, 
		то свет на границах частичек рассеивается и состав выглядит белым. Если 
		хоть раз температура превысила температуру плавления, состав плавится, 
		частично растворяется в связующем и застывает прозрачной массой, через 
		которую просвечивает подложка. Меняя состав покрытия можно довольно 
		точно задать температуру, при которой наклейка изменит цвет. Так как 
		используется явление плавления, то этот тип индикаторов так и называется -- 
		термоиндикаторы плавления. Наиболее близкая аналогия принципа действия 
		таких наклеек -- сахар, насыпанный в блюдце. Он выглядит белым, но стоит 
		хоть раз подняться температуре выше \unit{186 celsius}, сахар расплавится 
		и застынет прозрачной карамелью, сквозь которую просвечивает рисунок блюдца.
		Такие наклейки выпускает несколько  компаний в мире. На фото  заказанные 
		мной  британские safe connect \footnote{\hyperlink{https://www.safe-connect.co.uk/cable-wrap/}} 
		и отечественные LESIV \footnote{\hyperlink{https://www.lesiv.pro/}}.

		\placefigure[here][stickers]{Наклейки -- термоиндикаторы отечественного и британского производства.
			Температура срабатывания круглых британских наклеек 52°C, полосковых британских 70°C. 
			У отечественных -- точки с температурами срабатывания 50°С, 70°С, 80°С, 90°С, 100°С, 
			квадратные на 70°С и 90°С, полосковые на 90°С. Набор  возможных температур 
			индикаторов плавления весьма широк, я встречал варианты наклеек от 29°C до 290°C
			}{\externalfigure[stickersRUEN.jpg][width=\textwidth]}

		 

		Здесь я могу порадоваться, так как отечественные наклейки (а LESIV это, 
		кстати, фамилия разработчика, химик Алексей Лесив) работают не хуже 
		импортных, я проверил, но при этом ЗНАЧИТЕЛЬНО дешевле. \footnote{Наклейка 
		L-mark XL 250 р/шт без НДС. Наклейка safe connect  17 фунтов за 5 шт без 
		налога, по  курсу на момент покупки  это 520 руб за штуку. 
		Еще без учёта стоимости доставки и услуг мсье контрабандиста}  
		Подозреваю, оптовые цены у производителей значительно ниже.

		Для проверки наклеек я сделал стенд, изображённый на рисунке~\infig[testeqv]. 
		\placefigure[here][testeqv]{Фотография стенда, в котором автор испытывал наклейки.
			}{\externalfigure[testeq.jpg][width=\textwidth]}

		Все наклейки стабильно меняют цвет в районе указанной температуры. На рисунке~
		\infig[transition] коллаж из кадров процесса изменения цвета отечественной наклейки, 
		а на рисунке~\infig[transitionEN] изображён аналогичный процесс изменения цвета британской наклейки.
		
		\placefigure[here][transition]{Процесс изменения цвета наклейки от компании "Термоэлектрика" при нагреве. 
			}{\externalfigure[RUtransition.jpg][width=\textwidth]}

		\placefigure[here][transitionEN]{Процесс изменения цвета наклейки от компании "Safe Connect" при нагреве. 
			}{\externalfigure[ENtransition.jpg][width=\textwidth]}

		Термоиндикаторы могут быть не только в форме наклеек, но и в форме 
		пластиковых клипс, защёлкивающихся на провод, такие выпускает британская 
		компания Safe Connect. Почему они меняют цвет при нагреве (хотя сами, похоже, 
		сделаны на термопластавтомате!) -- я пока не смог разобраться, если вы 
		знаете -- напишите мне. При нагреве клипса меняет цвет с фиолетового на 
		розовый. К сожалению, стоимость клипсы ещё больше стоимости наклеек, и в 
		России их официально не купить.

		\placefigure[here][clips]{Термоиндикаторные пластиковые клипсы разных размеров от компании Safe Connect
			}{\externalfigure[clip.jpg][width=\textwidth]}

		Кадры видеозаписи процесса изменения цвета изображены на рисунке~\infig[clipstransition].

		\placefigure[here][clipstransition]{Процесс изменения цвета термоиндикаторной клипсы от 
			компании Safe Connect.
			}{\externalfigure[cliptrans.jpg][width=\textwidth]}

		По секрету скажу, что наклейка сохраняет работоспособность при разрезании, 
		поэтому для различных экспериментов и исследований её можно порезать на 
		мелкие квадратики. И, например, проверить какие части электронной платы 
		перегреваются в закрытом корпусе, не используя многоканальный регистратор 
		и кучу термопар. А ещё её можно клеить на корпуса редукторов, подшипников 
		и прочей техники, не обязательно электрической, чтобы отказывать в гарантии, если изделие 
		злостно перегревали. 

		Процесс выявления нагревающихся контактов с использованием наклеек 
		становится очень простым -- открываем щит и внимательно осматриваем все 
		наклейки,  не поменяла ли какая из них цвет. Если поменяла -- принимаем 
		меры к профилактике. Если щит имеет прозрачное защитное ограждение, то 
		для такого осмотра не нужно звать специально обученного человека, это 
		может сделать даже завхоз, и если какая-то из наклеек поменяла цвет, то 
		тогда вызывать электрика.

		Но есть у наклеек один существенный недостаток -- они абсолютно бесполезны, 
		если на них никто не смотрит. 
	\stopsubject

	\startsubject[title={Пустить газ!}]
		Допустим, у нас используется какое-то силовое оборудование, отказ которого 
		повлечёт просто астрономический ущерб, например, щит питания опасного 
		химического производства. Оборудование нагруженное, с высоким напряжением 
		и большими токами, поэтому процессы деградации плохого контакта протекают 
		весьма быстро -- за считаные недели, а то и дни. В таком случае полагаться 
		на регламентный осмотр нельзя -- путь от лёгкого нагрева до полыхающего 
		пламени с высокой вероятностью может быть пройден в период между осмотрами. В таком случае вопрос: 
		а можно ли сделать так, чтобы наклейка при срабатывании от нагрева контакта 
		могла об этом сообщить сама? Можно!

		Это отечественная система "Термосенсор". Представляет собой наклейку, 
		которая при нагревании выше пороговой температуры начинает бурно выделять 
		индикаторный газ. В щит также должен быть установлен  датчик, заточенный 
		обнаруживать появление газа из наклейки. На рисунке~\infig[gasstickrs] показаны наклейки и датчик, 
		присланные по моей просьбе производителем.

		\placefigure[here][gasstickrs]{Газовыделяющие наклейки системы "термосенсор"
			с датчиком газа. Корпус датчика -- обычный DIN-модуль на рейку.
			}{\externalfigure[stickerswithsensor.jpg][width=\textwidth]}

		Всё ноу-хау заключается в материале выделяющего газ полимера. В нём в инкапсулированном 
		состоянии заключён газ. Состав газа и самого полимера подобраны так, что 
		при температурах ниже пороговой -- выделение газа незначительно. При 
		повышении температуры выше пороговой, наклейка начинает потрескивать, а 
		её поверхность пучиться -- оболочки микрокапсул разрывает заключённым в 
		них газом, и он вырывается наружу. В качестве индикаторного газа используется 
		разновидность фреона, вроде широко разрекламированного Novec 1230, вы 
		наверняка видели его в видеороликах с "сухой водой". Он негорючий, не 
		токсичный, не вонючий, при комнатной температуре жидкий, химически не 
		активный, достаточно высокомолекулярен, чтобы долго сохраняться в полимерных 
		капсулах, и, главное, в нормальных условиях ему неоткуда взяться в электрощите! 
		Вместе с оборудованием в щит необходимо установить датчик газа, он постоянно 
		мониторит атмосферу внутри щита, и поднимает тревогу при появлении фреона 
		из наклейки -- замыкает контакт и сообщает по сети на пульт. Что делать 
		дальше --  зависит  от места установки. Где-то можно 
		произвести аварийное отключение, где-то направить аварийную бригаду для 
		разбирательства на месте. Чтобы упростить поиск сработавшей наклейки, на 
		ней есть термоиндикаторы нагрева, такие же, как в разделе выше.

		\placefigure[left][gasexpand]{Газовыделяющая наклейка до и после нагрева. 
			}{\externalfigure[gasexpand.jpg][width=0.5\textwidth]}

		На рисунке~\infig[gasexpand] изображена наклейка до и после нагрева. Выходящий
		при разрыве микрокапсул газ нарушил целостность полимера, поэтому он стал неровным.
		Чтобы упростить поиск наклейки, пустившей газ, на ней есть термоиндикаторные точки.
		При установке наклеек эти термоиндикаторные точки должны быть видны и прилегать
		к поверхности.\footnote{Если взять слишком длинную наклейку и приклеить её
		обернув в два слоя, то наклейка пустит газ, но термоиндикаторные точки 
		могут не сработать, что осложнит поиск неисправности.}

		Наклейка содержит довольно много фреона, я взвешивал её до и после 
		нагрева -- в маленькой наклейке почти 1 грамм газа, это почти 50\% от массы 
		наклейки! Газ хранится в наклейке вполне надёжно:  у меня они пролежали год,
		прежде чем у меня дошли руки до испытаний и написания поста, а учитывая дату 
		производства,  на момент моих испытаний им было два года. При нагревании 
		они с потрескиванием выделили газ, что успешно учуял датчик.

		\placefigure[right][gaswater]{Выделение газа из материала наклейки при 
			погружении в горячую воду с температурой выше температуры активации.
			}{\externalfigure[bubbles.jpg][width=0.4\textwidth]}

		На рисунке~\infig[gaswater] изображён кадр из видео, в котором я погружаю
		наклейку в горячую воду. Газ выделяется бурно. Взята наклейка с 
		температурой активации \unit{80 celsius}.

		Получается автоматическая система мониторинга состояния контактов, исключающая 
		человеческий фактор. Рядом с каждым контактом размещаем газовыделяющую 
		термоиндикаторную наклейку. Внутрь щита устанавливаем датчик газа. Если 
		какой-то из контактов начнёт нагреваться -- наклейка сработает, выпустит газ, 
		о чём нам на пульт сообщит датчик. Ну а дальше можно принять меры до наступления аварии.

		Но у этой системы есть свои недостатки, что ограничивает её повсеместное применение:
	
		\startitemize[n, packed]
			
				\item Система не работает в проветриваемых щитах. Думаю, очевидно, что 
				если вентиляторы или естественная конвекция будут гонять воздух через щит, 
				то и индикаторный газ будет уноситься. Из-за этого есть шанс, 
				что достаточная для срабатывания датчика концентрация не наберётся.

				\item Срок службы наклеек и датчиков газа ограничен и значительно 
				меньше, чем у термоиндикаторов, просто меняющих цвет. Так что это 
				не система "поставил и она навеки работает", это как огнетушитель: 
				"поставил и через n лет по плану заменил на новые".

				\item Из-за неидеальной селективности датчика он срабатывает 
				также на летучие органические соединения, например, бензин, этанол 
				и т.д. Поэтому на время лакокрасочных ремонтных работ систему 
				следует временно отключить -- будет ложное срабатывание.  
				Соответственно будут проблемы при эксплуатациях в помещениях, 
				где в воздух может попасть разное нехорошее: гараж, склад 
				горючего и т.д. В некоторых случаях это можно скомпенсировать 
				работой системы из нескольких датчиков, некоторые из них должны 
				быть снаружи щита и определять фон, привнесённый извне.
		\stopitemize
	\stopsubject	

	\startsubject[title={Куда и как закреплять наклейки, чтобы  вас не поминали добрым словом каждый раз}]
		Тут всё просто -- как можно ближе к месту потенциального нагрева, учитывая следующее:

		\startitemize[n, packed]
			
			\item Это обычная наклейка, поэтому, как и любая другая наклейка, она 
			плохо приклеивается на морозе, а также на грязные, жирные поверхности.
			 Поэтому желательно поверхность предварительно очистить. Неплотный 
			контакт с поверхностью увеличивает температуру, при которой наклейка сработает.

			\item Идеальное место для наклейки в 10-15 мм от контакта. Тепло при 
			нагреве отводится и рассеивается металлом проводников, поэтому чем 
			дальше от контакта, тем больше перепад температуры.

			\item Изоляция проводов также хороший теплоизолятор. Одинарная 
			изоляция провода -- даёт примерно 30-\unit{60 celsius} разницы температур 
			между поверхностью и жилой. Двойная изоляция -- больше, про это стоит помнить.

			\item Наклейка не должна мешать обслуживанию и блокировать доступ. 
			Иначе электрик её оборвёт, а вас помянет добрым словом.

		\stopitemize
		
		На рисунках \infig[stickplace] и \infig[sticktemp] изображены наглядно места установки наклеек.
		\placefigure[here][stickplace]{Оптимальные места для размещения термоиндикаторных наклеек.
				}{\externalfigure[stickerplace.svg][width=\textwidth]}

	
		\placefigure[here][sticktemp]{График зависимости температуры поверхности провода от расстояния.
				}{\externalfigure[templength.svg][width=\textwidth]}

		Наклейки можно размещать на корпусах приборов, нагрев которых косвенно 
		говорит о проблемах. Например, круглые британские наклейки предназначены 
		для наклеивания на корпус электрической вилки (у британцев они \overstrike{идио} 
		своеобразные, с плоскостью, где можно разместить наклейку). Срабатывание 
		наклейки на \unit{52 celsius}  на корпусе вилки говорит о нагреве контактов 
		вилки -- а значит, проблема или в самой вилке, или в розетке (а в Британии 
		ещё и своеобразная система с объединением проводки в кольцо). Чем 
		больше барьеров между наклейкой и контактом, тем ниже должна быть выбрана 
		температура срабатывания.
	\stopsubject
	
	\startsubject[title={Народная смекалка}]
		В комментариях под публикациями про термоиндикаторные наклейки почти всегда появляется
		предложение использовать наклейки из термобумаги, на которой печатают этикетки.
		И в самом деле, покрытие термобумаги обладает свойством менять цвет при нагреве.
		На рисунке~\infig[thermalpaper] изображён график изменения цвета термобумаги  CZ208 
		от компании Балтийская целлюлоза при нагреве.

%https://baltcell.ru/images/site/Description/termobumaga_cz208.pdf		
		\placefigure[left][thermalpaper]{График зависимости оптической плотности термобумаги  CZ208
				при нагреве.
				}{\externalfigure[thermalpaper.jpg][width=0.4\textwidth]}
		
		Идея кажется рабочей, но дьявол, как всегда, кроется в деталях:
		\startitemize
		\item Термобумага со временем выцветает, вы и сами могли видеть, как 
			чеки становятся за год-два едва читабельны, особенно если хранились 
			на свету. А значит, каждый профилактический осмотр приведёт к плановой замене всех наклеек.

		\item Наклейка выцветает при взаимодействии с некоторыми химическими веществами. Где гарантия,
			что наклейка не прореагирует с пластификатором изоляции и не потеряет 
			цвет раньше срока? Солнце они тоже не любят.

		\item Клеевой слой наклеек весьма слабый, особенно при температуре 60--\unit{70 celsius}. 
			Может так получиться, что наклейка отвалится раньше, чем сработает.
		
		\item И, наконец, основа наклейки -- бумага. Обычная горючая бумага. В книгу не вошла
			публикация, где я сравнивал термоиндикаторные наклейки. Британская 
			наклейка в качестве основы для термоиндикаторной массы использует чёрную бумагу.
			Отечественная -- винил, когда я их поджёг, наша потухла, как изолента, 
			а британская стала интенсивно гореть, что я от неожиданности даже растерялся.

		\stopitemize
	
		Получается, схожесть только внешняя. Легковой автомобиль и телега похожи -- 
		имеют по четыре колеса, катятся по дороге, но имеют разные задачи и выполняют их по-разному.
		Поэтому, к сожалению, наклейки из термобумаги не способны заменить наклейки 
		с термоиндикаторами плавления.

	\stopsubject

\stopchapter

\startchapter[title={Реле контроля напряжения}]

	Устройства защиты из предыдущих глав боролись с выделением тепла там, где это не нужно. 
	В этой и следующих главах рассматриваются устройства иного назначения, они защищают сеть и нагрузку от 
	некондиционного электричества. И наиболее важным является реле контроля 
	напряжения, которое окупается за 1/100 секунды.

	Почти наверняка вам попадались новости с описанием того, как "из-за скачка 
	электроэнергии сгорела бытовая техника в подъезде многоэтажки". К счастью, 
	чаще всего новость не содержит информации о пожаре или погибших, но убытки 
	часто исчисляются миллионами рублей.

	\placefigure[here][newsvolt]{Коллаж из новостей инцидентов с сгоревшей техникой.
			}{\externalfigure[newsvolt.jpg][width=\textwidth]}

	Чаще всего возмещение убытков со стороны виновного лица происходит после 
	долгих и изматывающих юридических процедур, и часто далеко не полностью покрывает потери пострадавших.

	\startsubject[title={Природа мифического "скачка"}]
		И правда, при обрыве нейтрального проводника возможна ситуация под 
		жаргонным названием "перекос фаз", когда напряжение в розетке вместо \unit{230 volt}
		может как понизиться, так и повыситься вплоть до \unit{400 volt}. 
		\footnote{Напомню, что последние -нцать лет наша страна переходит с \unit{220 volt} 
		на \unit{230 volt}, для унификации с европейскими стандартами. Соответственно, 
		вместо \unit{380 volt} теперь \unit{400 volt}} Причём, это не кратковременный 
		всплеск из-за переходных процессов от коммутации мощных нагрузок, а 
		длительное явление, при котором начинает выходить из строя бытовая техника. 
		Разберёмся, откуда  этот "скачок" электроэнергии берётся. 

		Исторически так сложилось, что в энергетике обрела популярность система 
		переменного тока, имеющая три фазы. Возможны системы с иным количеством 
		фаз, но именно трёхфазная стала самой популярной в силу своих достоинств. 
		Генератор (или трансформатор на подстанции)  имеет три обмотки, на каждой 
		из которых наводится ток, который и передаётся потребителю. 

		\placefigure[here][threephase]{Коллаж из новостей инцидентов со сгоревшей техникой.
			}{\externalfigure[3ph.svg][width=\textwidth]}

		Ток наводится в обмотках с небольшой разницей во времени. Для удобства 
		эту разницу выражают не в секундах, а как величину угла, где за 360~градусов, 
		или полный круг, принимают один период тока. Для европейской системы с 
		50~Гц это составляет 1/50 долю секунды, или 20 миллисекунд. Чтобы узнать напряжение между фазами, 
		нужно выполнить векторное сложение фазных напряжений. Это можно сделать графически на 
		масштабной бумаге, либо использовать тригонометрию.

		Идеальным для такой системы электроснабжения является трёхфазный потребитель, 
		например, асинхронный электродвигатель. Он забирает ток от генератора поровну 
		по всем трём фазам, и баланс токов не нарушается. На картинке выше показано 
		соединение генератора и нагрузки с нейтральным проводником N ("нуль" на 
		жаргоне электриков). Если величина нагрузки по всем трём фазам одинаковая, 
		при сложении всех векторов напряжений и токов потенциал точки N будет 
		равным нулю. Это часто изображают векторной диаграмме, на ней часто также 
		обозначают три вектора линейных напряжений и располагают так, чтобы 
		получился треугольник, я заменил их пунктиром на рисунке~\infig[vectors].

		\placefigure[here][vectors]{Векторная диаграмма.
			}{\externalfigure[vector.svg][width=\textwidth]}

		Для упрощения изложения будем считать, что у тока нет реактивной 
		составляющей. Если нагрузка в цепи имеет значительную ёмкость или 
		индуктивность, нарастание тока начинает отставать или опережать 
		нарастание напряжения (их фазы различаются). В быту подобные 
		особенности можно не учитывать, хотя любой энергетик про них знает.

		Увы, не все потребители такие удобные. Почти все бытовые квартирные 
		электроприборы используют лишь одну фазу переменного тока. В таком 
		случае всех потребителей делят на три примерно равные по мощности 
		группы и подключают к источнику тока. Например, в многоквартирном доме на 
		каждую из фаз подключается примерно 1/3 квартир, и для трансформатора 
		на подстанции весь дом - просто ещё один трёхфазный потребитель. Но в 
		реальности идеального баланса нагрузок по всем трём фазам добиться 
		невозможно, поэтому нейтральный проводник начинает играть важную роль -- 
		по нему начинает протекать уравнивающий ток, и чем больше дисбаланс 
		потребления токов по фазам, тем больше уравнивающий ток.
 
		Если потребителей достаточно много и они распределены по фазам 
		равномерно, то можно посчитать статистику и обнаружить, что 
		уравнивающий ток через нулевой проводник по величине обычно меньше, 
		чем ток любой из фаз. А если проводник не используется в полной мере, 
		то его сечение можно сократить, сэкономив ценный металл. В некоторых 
		старых домах такое можно встретить -- нейтральный проводник от дома до подстанции имеет 
		сечение меньше, чем фазный (внутри квартир они одинакового сечения). И это работало, до недавнего времени.

		Повторю основную мысль: в трёхфазных сетях при сбалансированной нагрузке 
		через нейтральный проводник ("нуль") ток к генератору отсутствует. 
		Если нагрузки по фазам не сбалансированы -- то нейтральный проводник 
		становится критически важным для поддержания равного напряжения по фазам, 
		но ток через него заметно меньше тока любого из фазных проводников.
	\stopsubject

	\startsubject[title={Так почему же отгорает ноль?}]
		Есть две проблемы, которые приводят к росту значения тока через 
		нейтральный проводник -- это сильная асимметрия нагрузки, на которую 
		посмотрим чуть позже, и гармоники тока, кратные трём. А так как в старых 
		сетях нейтральный и защитный проводник совмещены (система TN-C), то 
		никаких устройств защиты его от перегрузки (предохранитель, автоматический 
		выключатель) не устанавливается. Это и приводит к тому, что через 
		нейтральный проводник незамеченным может течь ток свыше предельно 
		допустимого. А если по проводнику гуляют токи -- он нагревается, и при 
		больших токах может перегореть. Чаще всего это происходит в местах 
		подключения, плохой контакт тоже греется и порождает шутки про суровый 
		светодиод, уже появившийся в книге на рисунке~\infig[brutalled].


		Откуда берутся гармоники и почему они приводят к росту тока через 
		нейтральный проводник? Если нагрузка нелинейная, например, в виде 
		импульсного блока питания, то ток из сети каждый период колебаний 
		напряжения потребляется неравномерно, что очень сильно искажает 
		форму питающего напряжения. Если подключить осциллограф к сети, 
		то вместо красивенькой ровненькой синусоиды мы можем увидеть 
		странную горбатую кривую, как изображённую на рисунке~\infig[harmonics]. 
		Небольшое количество чёрной математической 
		магии – а именно, преобразования Фурье – позволяет разложить любую 
		периодическую, сколь угодно горбатую, кривую на сумму простых 
		синусоид, которые составляют её спектр. Синусоиды спектра, частота
		которых кратна основной, называются гармониками. Эту магию мы упоминали 
		в главе про устройства защиты от дугового пробоя, рассказывая 
		про помехи от пылесоса, импульсных блоков питания и дуги, но не затрагивали детальнее.

		\placefigure[here][harmonics]{Пояснение принципа разложения исходного 
			сигнала на составляющие гармоники.
			}{\externalfigure[harmonics.svg][height=\textheight]}

		Видно, что корявую исходную кривую можно заменить суммой простых синусоид. 
		Каждая газоразрядная лампа, сварочный аппарат, светодиодная лампа с 
		импульсным драйвером и т.д. из-за своей нелинейности искажают форму 
		сетевого напряжения, что можно представить как протекание токов, частота 
		которых кратно выше частоты сети. И чем сильнее форма потребляемого 
		тока отличается от синусоиды, тем мощнее вклад гармоник.

		
		
		Наиболее вредны для нас гармоники, частота которых кратна трём -- то есть 
		\unit{150 hertz}, \unit{300 hertz}, \unit{450 hertz} и т.д. Их особенность 
		в том, что они синхронны во всех трёх фазах! Смотрите картинку \infig[harmonicsN]

		\placefigure[here][harmonicsN]{Пояснение принципа сложения гармоник в общем проводе
			}{\externalfigure[harmonicsN.svg][width=\textwidth]}

		Токи нагрузок по фазам складываются в общей точке, итоговый суммарный 
		ток зависит от разницы фаз токов. Если сумма первых гармоник может быть 
		равна нулю, то сумма токов третьих гармоник может быть значительна, 
		ведь их токи синхронны. Складывается парадоксальная ситуация, когда 
		нагрузки по трём фазам идеально сбалансированы по мощности, но суммарный 
		ток нулевого проводника  может быть даже больше, чем ток любой из фаз! А значит и его нагрев от протекающего тока.

		\placefigure[here][normal3phase]{Токи в проводниках при нормальной работе нейтрального проводника
			}{\externalfigure[normal3phase.svg][width=\textwidth]}

		Различные нормативные документы строго ограничивают величину помех и 
		гармоник, создаваемых устройствами при работе от электросети как раз в 
		том числе из-за этой проблемы. Конструкторы вынуждены добавлять в 
		электроприборы фильтры, блоки коррекции коэффициента мощности (PFC) что 
		делает их дороже, но иначе сертификационные испытания не пройти. Но, увы, 
		не все товары ввозятся легально и не за каждой бумажкой стоят реальные 
		испытания, поэтому на полках можно встретить кучу блоков питания, 
		зарядников, лампочек и т. д. в которых как раз для удешевления и 
		выкинули эти блоки. Без фильтра оно тоже работает, но создаёт проблемы с 
		помехами и гармониками. К счастью, пока таких дешёвых и кривых устройств в общей массе немного – это не создаёт ощутимых проблем.

		Вторая причина протекания через нейтральный проводник тока -- асимметричная 
		нагрузка по фазам. Для иллюстрации представим, что у нас многоквартирный 
		дом с тремя подъездами, и электрики подключили каждый подъезд на одну 
		фазу. Вверху над домом подписана суммарная мощность потребителей каждого 
		подъезда. При такой конфигурации по нулевому проводнику будет течь 
		уравнивающий ток около \unit{27 ampere}.

		Когда значение токов и напряжений по трём фазам начинает значительно 
		отличаться, то это явление жаргонно называют "перекос фаз". 

		Представим, что нейтральный проводник не выдержал протекающего по нему 
		тока (как было сказано выше -- в некоторых старых проектах его сечение 
		меньше фазных, так как в нормальных условиях ток через него небольшой), 
		и перегорел. В таком случае уравнивающий ток не протекает, и напряжение, 
		получаемое потребителем каждой фазы, зависит от соотношения мощности нагрузок на 
		соседних фазах. В худшем случае оно может стать равным линейному -- 
		\unit{400 volt} (\unit{380 volt} по старинке), например, если у соседей 
		включены обогреватели, а у вас только одна маленькая лампочка. 
		Понятное дело, что электроприборы рассчитанные на \unit{230 volt}, 
		повышение напряжения воспринимают с энтузиазмом в виде дыма и других 
		пиротехнических эффектов. В нашем примере обрыв нейтрального проводника 
		вызовет следующие изменения напряжений в каждом из подъездов, указанных 
		на рисунке~\infig[PENfail].

		\placefigure[here][PENfail]{Токи в проводниках при обрыве нейтрального проводника.
			}{\externalfigure[PENfail.svg][width=\textwidth]}


		Теперь вы понимаете, откуда взялся "скачок" напряжения. Причём такого 
		рода аварии происходят не только в старом жилом фонде или у нерадивых УК, 
		которые в принципе решили экономить на плановом обслуживании электрохозяйства. 
		Такого рода аварии случаются иногда и при ошибке персонала -- электричество 
		отключили для плановых работ на подстанции, включают обратно, а лампочки 
		как-то подозрительно ярко горят и гарью начинает пахнуть...
	\stopsubject

	\startsubject[title={Защита от повышенного напряжения}]
		Специально для защиты от таких аварийных ситуаций, когда напряжение в 
		сети начинает превышать норму, придумали устройства под названием "Реле 
		контроля напряжения". Это как раз то, что называется "маст хэв",\footnote{
		must have (англ.) - обязан иметь} поскольку 
		окупается практически мгновенно при первой аварийной ситуации. Несмотря 
		на простую функцию устройств, на рынке представлены разные варианты 
		подходов к реализации данной функции. На рисунке~\infig[relays] представлены разные варианты реле 
		контроля напряжения, всё что я наскрёб у себя по сусекам.

		\placefigure[here][relays]{Реле контроля напряжения различных производителей.
			}{\externalfigure[relays.jpg][width=\textwidth]}

		В самом простом случае это некоторый пороговый элемент: если напряжение 
		превысило допустимое -- устройство отключает нагрузку. А вот дальше есть нюансы:

		\startitemize[n]

			\item Устройство не должно быть чересчур быстродействующим, так как по сети 
			гуляют помехи, которые можно наблюдать как "иголку" амплитудой выше 
			допустимого, но в силу очень малой ширины отключать что-то бесполезно: реле срабатывает банально дольше, чем длится эта помеха. 
			Для борьбы с такими помехами служат другие устройства (фильтры, УЗИП), 
			а реле контроля напряжения на такие помехи реагировать не должно.

			\item Устройства часто имеют регулировку пороговых значений напряжения 
			отключения. По ГОСТ допускается \pm 10\% от номинального значения. 
			Но бывают, например, длинные линии в посёлках и коллективных садах, когда 
			напряжение систематически снижено на 5-\unit{10 volt} ниже нижнего допустимого порога. 
			Если не иметь настройки, то такое реле будет постоянно отключать дом, хотя 
			у владельца в доме нет потребителей, для которых такое напряжение является 
			фатальным.

			\item Наличие гистерезиса и таймера повторного включения. Многие реле 
			контроля напряжения предназначены включить всех потребителей, как 
			только напряжение нормализовалось. Если это делать сразу, да ещё без 
			гистерезиса (то есть, разницы между порогом отключения и порогом включения), 
			то можно получить неприятное циклическое включение-отключение. Реле будет 
			быстро отключать нагрузку, от чего напряжение в сети изменяется (у 
			проводов есть своё ненулевое сопротивление), и реле вынуждено снова включить нагрузку, 
			от чего напряжение снова уползает за порог и нужно опять отключать... 
			Кроме того, некоторые компрессоры холодильников могут не 
			запуститься сразу после повторного включения, пока давление не выровнялось. 
			Для них адекватной будет задержка перед повторным включением в несколько минут!
		\stopitemize

	\stopsubject

	\startsubject[title={Почему пониженное напряжение -- тоже плохо}]
		Увы, пониженное напряжение тоже может закончиться бедой. Пониженное 
		напряжение опасно для асинхронных электродвигателей. При низком 
		напряжении пусковой момент электродвигателя снижается, ему просто не 
		хватит сил тронуться и раскрутиться с механизмом до номинальной скорости и перейти 
		в рабочий режим. Это значит, что через обмотки двигателя будет протекать пусковой ток, который гораздо больше 
		номинального, и будет разогревать обмотки мотора не доли секунды, а десятки 
		секунд. Если защита двигателя не сработает должным образом, то двигатель сгорит.

		Особой изюминки добавляет то, что часто единственный асинхронный 
		электродвигатель в доме расположен в компрессоре холодильника (и кондиционера). 
		А двигатель, мало того, что работает в герметичном корпусе, частично 
		погружённый в масло, так и в качестве хладагента всё чаще используется 
		не фреон, а горючий изобутан (r600a). Если по какой-то причине подведёт 
		встроенная защита компрессора, то дело может обернуться пожаром. Помимо 
		компрессоров холодильников, асинхронные двигатели устанавливаются в 
		циркуляционные насосы, вентиляторы, компрессоры и помпы.
		
		\attention{} Ещё один контринтуитивный эффект от пониженного напряжения: потребляемый ток может повыситься,
		со всеми вытекающими последствиями в виде нагрева кабелей, если нагрузка состоит в основном из импульсных
		источников питания. Например, майнинг-ферма. Импульсный блок питания, благодаря
		обратной связи ШИМ контроллера с выходом, работает в широких диапазонах напряжения
		питания. Посмотрите на блок питания своего ноутбука или зарядник от телефона, там 
		наверняка написано, что входное напряжение 110--240В. Если блок питания обязуется
		выдать на выходе \unit{12 volt} и \unit{10 ampere} = \unit{120 watt}, то
		он эти \unit{120 watt} заберёт из сети. При \unit{220 volt} ему понадобится
		\unit{0.54 ampere}, но если напряжение на входе будет \unit{110 volt}, то 
		потребление вырастет до \unit{1.09 Ampere}. Эта нелинейность может 
		сильно озадачить, так как при классической нагрузке (будь она активной или реактивной) при снижении напряжения снижается и ток потребления,
		а тут при снижении напряжения ввод начинает греться и выбивать правильно
		рассчитанный автомат защиты.
		

		Остальные приборы при пониженном напряжении в сети просто работают хуже -- 
		обогреватели нагреваются меньше. Микроволновые печи перестают греть, 
		но при этом вращая блюдо как ни в чём не бывало. Лампы накаливания 
		светят тускло. Устройства с импульсными блоками питания -- зарядники, 
		компьютеры, светодиодные лампы и т.д. вообще не замечают низкого напряжения. 
		То, что напряжение в сети провалилось до \unit{190 volt}, я узнал только потому, что 
		мне пожаловались что микроволновая печь плохо греет. Светодиодные лампы, 
		телевизор, компьютер, холодильник работали нормально.

		Поэтому, если среди потребителей есть устройства с асинхронными 
		электродвигателями, необходимо отключение как по повышенному, так и по 
		пониженному напряжению. Если же защищается, например, сторожка с телевизором 
		и обогревателем, то защита от пониженного напряжения будет избыточна, 
		нужна защита только от повышенного напряжения.
	\stopsubject

	\startsubject[title={Особые потребности трёхфазных потребителей}]
		Нельзя просто так взять и поставить три обычных реле контроля напряжения, 
		если у вас трёхфазный ввод. Три отдельных устройства вместо 
		специализированного, трёхфазного, не позволят вам реализовать две важные функции:

		\startitemize[n]

			\item Контроль обрыва одной из фаз. Если пропустить этот момент, то 
			трёхфазным электродвигателям станет плохо, и если они не имеют своей 
			защиты, то это чревато аварийным режимом работы.

			\item Контроль последовательности фаз. Если где-то ошибётся электрик и 
			перепутает две фазы, то изменится их последовательность, а значит 
			направление вращения всех подключённых к сети трехфазных двигателей, 
			что опять-таки может привести к механическим поломкам.
		\stopitemize

		Поэтому, если у вас дома/в мастерской/цеху/гараже есть потребители, 
		использующие одновременно три фазы, то и реле напряжения должно быть трёхфазным. 
	\stopsubject

	\startsubject[title={Это так не работает}]
		Возможно, читатель, уже прочитавший главу про УЗИП, может задаться 
		вопросом -- а может просто поставить на входе УЗИП? Ведь они предназначены 
		как раз срабатывать при превышении номинального напряжения, при превышении 
		напряжения они сработают, устроят короткое замыкание и отключат вводной 
		автомат. Рассуждение не лишено логики, но так не делают -- защита 
		получается очень дорогой и одноразовой, и служить заменой реле контроля 
		напряжения они не могут. Кроме того, ограничители импульсных перенапряжений 
		часто делают на номинальное напряжение \unit{400 volt}, то есть в нашей 
		задаче они вообще будут бесполезны.

		Также не стоит полагаться на стабилизаторы напряжения как на защиту. 
		К сожалению, некоторые модели стабилизаторов столь упрощены, что 
		выполнять функцию защиты при обрыве нуля не будут, и \unit{400 volt} на 
		входе их убьёт столь же быстро, как и остальную бытовую технику.
	\stopsubject

	\startsubject[title={Практическая реализация}]
		Существует как минимум три варианта реализации устройств защиты от 
		обрыва нуля.

		1. Использование специализированных устройств всё-в-одном. Например, 
		устройства Новатек РН-104 и Меандр УЗМ-51МД на рисунке~\infig[rn104].

		\placefigure[here][rn104]{Законченное реле напряжение.
			}{\externalfigure[rn104.jpg][width=\textwidth]}

		Внутри устройства уже есть реле, которое своими контактами будет 
		отключать нагрузку, поэтому никаких дополнительных манипуляций для 
		подключения не требуется. Впрочем, компактность заставляет идти на 
		компромиссы, поэтому максимальная нагрузка по току таких устройств 
		всё же ограничена.

		2. Реле напряжения, требующее отдельного контактора. На рисунке~\infig[relayext] такое реле 
		IEK OV-01 и контактор КМ20-11М (контактор взял для демонстрации, в 
		реальном применении стоит взять контактор помощнее).

		\placefigure[here][relayext]{Реле контроля напряжения в паре с контактором.
			}{\externalfigure[relayext.jpg][width=\textwidth]}

		Преимущество тут в том, что контактор может быть большим и брутальным, 
		чьи контакты в состоянии выносить мощные броски тока, а также в 
		состоянии разорвать цепь при больших токах или большой индуктивной 
		составляющей. Огромное количество импульсных блоков питания в 
		современной технике создаёт весьма ощутимые токи при включении, 
		способные сварить маленькие контакты встроенных реле. Контакторы 
		гораздо более устойчивы к этому просто в силу размеров и создаваемых 
		усилий. 

		Если вместо контактора использовать внешний электромагнитный 
		расцепитель к автоматическому выключателю, то мы потеряем возможность 
		включиться обратно при нормализации напряжения, но зато у нас не 
		будет постоянно включенного (гудящего и греющегося) контактора. 
		Возможность задать свои собственные уставки(пороги) срабатывания при этом сохраняются.

		Также внешний контактор можно всегда подключить и к устройствам 
		"все-в-одном", но стоимость такого решения будет выше.

		3. Аксессуары к автоматическим выключателям. На рисунке~\infig[relayacc] такой вариант, 
		РММ47 к автоматическим выключателям IEK ВА47-29

		\placefigure[here][relayacc]{Внешний расцепитель максимального напряжения к автоматическому выключателю
			}{\externalfigure[relayacc.jpg][width=\textwidth]}

		Такая "нашлёпка" на автоматический выключатель имеет рычажок, 
		которым способна его отключить, если напряжение превысит пороговое. 
		Автоматическое повторное включение в таком случае невозможно, но 
		схема получается крайне простая, дешёвая и сердитая, имеющая право 
		на жизнь, например, в щите управления уличным освещением. Или, если 
		защиту добавить очень хочется, а места в щите осталось всего на 1 модуль.

		Такие внешние расцепители есть в каталогах многих производителей 
		модульных автоматов защиты, но чаще всего они отключают только по 
		превышению напряжения, внимательно смотрите документацию.

		4. Почти бесплатно --- защита от повышенного напряжения как часть УЗДП 
		(устройств защиты от дугового пробоя).

		\placefigure[here][AFDD]{УЗДП разных производителей.
			}{\externalfigure[AFDD.jpg][width=\textwidth]}

		Многие УЗДП представленные на отечественном рынке имеют встроенную 
		защиту -- они отключаются, если напряжение питания превышает порог, 
		который, как правило не настраивается. Такая защита удовлетворяет не 
		всегда, но в некоторых вариантах вполне достаточна. Если из стоимости 
		УЗДП вычесть стоимость самого простого реле контроля напряжения, то этот 
		вид защиты становится гораздо более привлекательным.
	\stopsubject

\stopchapter

\startchapter[title={Устройства защиты от импульсных перенапряжений}]

	Жители частных домов и владельцы садовых домиков беззащитны перед стихией, 
	разряд молнии в грозу может попасть в дом по воздушной линии электропередач 
	и наделать бед. Именно для защиты потребителей в доме от таких наведённых 
	импульсов напряжения служат устройства защиты от импульсных перенапряжений.

	Сейчас почти каждый знает, что молния представляет собой разряд электричества, 
	иногда ударяет в рукотворные объекты и способна испортить технику. Хоть это 
	предложение и звучит по детски, но человечеству понадобились века для 
	понимания таких простых и очевидных сегодня вещей. Знание о природе и 
	характеристиках разряда не далось человечеству без жертв, помянем Георга Вильгельма Рихмана.

	Первыми регулярный ущерб от удара молниями стали испытывать связисты -- 
	телеграфные линии, растянутые по полям на столбах, регулярно приносили к 
	дорогому и нежному оборудованию станций кратковременные всплески высокого 
	напряжения. Причём не только от ударов молнии в сами провода, но даже от 
	ударов молний неподалёку от линий! И уже тогда пришлось изобретать способы 
	защиты оборудования  от этих всплесков.  Когда, спустя десятилетия, свои 
	провода стали растягивать на столбах уже энергетики для только появившегося 
	электрического освещения, некоторые наработки телеграфистов пригодились.

	\placefigure[here][stat1885]{Вырезка из журнала The telegraphic journal and electrical review за 1885 год.
			в правой части график количества ударов молнии, поломавших телеграф в зависимости от месяца и времени суток.
			}{\externalfigure[stat1885.jpg][width=\textwidth]}



	Стоит сказать, что для современной техники молния уже не является чем-то 
	запредельно мощным и умопомрачительным. Если взять все эти миллионы вольт и 
	сотни тысяч ампер и умножить на время – мы получим энергию разряда, а это 
	всего порядка \unit{1 giga joule} энергии. Если перевести в привычные \unit{kilowatt hour}, 
	то это всего \unit{277 kilowatt hour}, можно даже посчитать стоимость одного 
	разряда молнии:) Проблема лишь в том, что это количество энергии выделяется 
	за доли секунды, что порождает проблемы, с которыми и борются разными 
	техническими приёмами. 
	
%[https://van.physics.illinois.edu/qa/listing.php?id=2280&t=energy-of-a-lightning-strike ]
	Что происходит при ударе молнии в линию электропередач? Энергия молнии 
	растекается по проводникам в поисках пути ухода в землю. Это вызывает рост 
	напряжения до огромных величин, из-за чего изоляция не выдерживает, и её 
	пробивает. В тех местах, где протекал разряд, повреждения оставляет как 
	нагрев, так и электромагнитные силы. И про электромагнитные силы хочу 
	отметить особо: из-за очень большой скорости нарастания тока при ударе молнии, 
	разряд, даже если не попал в нашу линию, наводит токи в окружающих проводниках. 
	Поэтому  если молния ударила в молниеотвод на крыше и ушла по 
	металлоконструкциям в землю, на проводах внутри здания могут появиться 
	всплески напряжения опасной величины. Следовательно защита строится не только от 
	прямых попаданий молнией, но и от различных наведённых ею явлений.

	Вопрос защиты от атмосферного электричества и от импульсных перенапряжений 
	достаточно обширен, поэтому глава рассчитана дать лишь крайне поверхностное 
	представление и не претендует на полноту. Для более полного и глубокого 
	изучения темы в конце есть ссылки на дополнительные материалы. Если кратко
	сформулировать  физический смысл устройств защиты, то их задача 
	сбросить в заземление всю энергию, наведённую в линиях  молнией, не 
	допуская чрезмерного роста напряжения. Эти устройства назвали УЗИП -- 
	устройства защиты от импульсных перенапряжений. 

	\startsubject[title={Приманиваем молнию и отправляем её в землю}]
		Про громоотводы (они же молниеотводы, и они же молниеприёмники. 
		Каждый раз, когда я в тексте использую просторечное «громоотвод», где-то 
		икает специалист по молниезащите. Правильный термин по ГОСТ Р 62561.2-2014 
		«молниеприёмник», но от применения просторечных именований принцип 
		работы защиты не изменяется) наверняка слышали и видели все. Это не 
		обязательно торчащий в небо шпиль, у линий электропередач он выполнен 
		в виде грозозащитного троса, который натянут выше основных линий.

		\placefigure[here][lightningrod]{Опора линии электропередач. В самом вверху натянут грозозащитный трос,
		который работает молниеприемником.
		Для безопасности пилотов на тросе закреплены яркие пластиковые сферы, так 
		линии электропередач гораздо заметнее с высоты.
			}{\externalfigure[powerline.jpg][height=0.9\textheight]}

		Принцип громоотвода простой -- это проводник, электрически соединённый 
		с землёй, и размещённый как можно выше. Если на данном участке создадутся 
		условия, для удара молнией, то наиболее вероятно (но не 100\% гарантированно!) 
		разряд произойдёт именно в заземлённый проводник, а не в окружающие 
		объекты. Сечение проводника выбирается достаточным, чтобы провести 
		разряд к заземлению без повреждений. Громоотвод выполняет собой роль 
		"зонтика", принимая всю стихию на себя. Аналогия с зонтиком становится 
		ещё более явной, если посмотреть на формулы расчёта радиуса защищаемой 
		громоотводом площади, она тем больше, чем выше громоотвод. Стоит 
		отметить, что существует несколько методик определения защищаемой 
		молниеотводом области, и даже среди специалистов по молниезащите нет 
		единогласного мнения, какая методика точнее. 

		\placefigure[here][lightningrodradius]{Область пространства, которую защищает молниеотвод.
			Удар в область отмеченную голубым цветом маловероятен, но не исключён. Слева использована методика
			защитного угла \alpha, величина которого берётся из стандарта МЭК 62305 в зависимости от высоты
			и требуемого уровня защиты. Справа использована методика катящейся сферы, радиус которой так же
			берётся из стандарта и зависит от высоты и требуемого уровня защиты.
			}{\externalfigure[safearea.svg][width=\textwidth]}

		Громоотвод оказался чертовски важен для использования в деревянных 
		домах. Если раньше удар молнии в крышу мог устроить пожар (энергия 
		разряда на пути в землю частично превращалась в тепло, поджигавшее 
		всё вокруг), то перенаправление разряда по металлическому штырю в 
		землю спасало от таких страшных последствий. И, если присмотреться, 
		 все современные здания и строения имеют на крыше громоотвод. А 
		особо важные объекты вообще могут иметь довольно сложные конструкции 
		громоотводов. В тех местах, где надлежащее заземление сделать трудно 
		(на скале, песках), молниезащита становится совсем нетривиальной задачей.  
		
		Но, если бы этот способ работал без нареканий, то текст бы оборвался на этом 
		месте. Он и обрывался, до появления чувствительной и нежной аппаратуры.
	\stopsubject

	\startsubject[title={Минимолнии}]

		Не все высоко поднятые проводники могут быть заземлены для успешного 
		перенаправления энергии разряда в землю. Например, антенны -- она должна 
		быть высоко, и заземлять её нельзя, иначе она перестанет принимать сигналы. 
		А можно ли сделать устройство, которое соединяло бы  антенну 
		с землёй только в момент удара молнии, и при этом не оказывало влияния 
		в остальное время?

		Можно, и устройство это называется искровой разрядник. На рисунке~\infig[vintagearrester] пример 
		разрядника для электрооборудования конца 19 века:

		\placefigure[here][vintagearrester]{Искровой разрядник для защиты линий электропередач.
			Иллюстрация из журнала Western Electrician от 12 апреля 1890 года.
			}{\externalfigure[vintagearrester.jpg][width=\textwidth]}

		Идея защиты проста -- между защищаемым проводником и заземлением в 
		разряднике создаётся минимально допустимый зазор так, чтобы при 
		нормальной работе напряжение не превышало напряжение пробоя зазора. 
		Если в защищаемой линии по какой-то причине напряжение возрастёт (из-за 
		удара молнии или из-за всплесков от работы электрооборудования), то в 
		зазоре происходит электрический пробой -- зажигается электрическая дуга, 
		которая из-за ионизации газа неплохо проводит ток. Именно эта дуга 
		обеспечивает временное электрическое соединение с землёй и гаснет, 
		если напряжение понизилось ниже напряжения гашения дуги. 

		Но есть две проблемы. Первая -- малопредсказуемое напряжение пробоя 
		разрядника: любое изменение температуры, влажности воздуха, и напряжение пробоя 
		изменилось. Немного коррозии -- напряжение пробоя изменилось. Кривые ручки 
		регулировщика -- очень сильно изменилось. Второй недостаток более 
		фундаментальный -- напряжение, при котором происходит пробой, и 
		напряжение, при котором дуга гаснет, отличаются. Причём напряжение 
		зажигания дуги ещё зависит от скорости нарастания напряжения. 
		График на рисунке~\infig[arresters_graph] как раз показывает "горб" -- пока разрядник не 
		сработал, напряжение успевает вырасти, затем зажигается дуга, и 
		напряжение падает. Синим цветом показан график напряжения при защите 
		варистором.

		\placefigure[here][arresters_graph]{Работа ограничителей перенапряжения различных типов
			}{\externalfigure[arresters_graph.svg][width=\textwidth]}


		Если первый недостаток получилось побороть, заключив разрядник в 
		герметичную колбу, заполненную заранее приготовленной смесью газов, то 
		со вторым ничего поделать не получилось. Да, разными ухищрениями можно 
		уменьшить разницу между напряжением пробоя и напряжением, когда дуга 
		гаснет, но не радикально. Причём напряжение гашения должно быть ВЫШЕ 
		напряжения источника питания (*с оговорками). Иначе может получиться 
		неприятная ситуация, когда разряд молнии пробил разрядник и ушёл в 
		землю, но дуге уже не даёт погаснуть генератор, питающий линию. 
		И дуга в разряднике будет гореть, пока кто-то из них не сломается. 
		
		На рисунке~\infig[rb5] разрядник РБ-5 отечественного производства из аппаратуры 
		связи -- колба герметична и заполнена инертным газом.

		\placefigure[here][rb5]{Разрядник РБ-5 производства СССР.
			}{\externalfigure[rb5.jpg][width=\textwidth]}

		В принципе, до широкого распространения полупроводниковых приборов 
		(где-то до середины 1960\high{х}) защита в виде разрядников всех устраивала. 
		При должном запасе прочности изоляции, кратковременный всплеск 
		напряжения на пару кВ (пока не сработает разрядник) большая часть 
		аппаратуры могло вынести. Но потом в широкий обиход вошли полупроводниковые 
		устройства, для которых даже небольшое кратковременное повышение напряжения 
		означало смерть. 

		Разрядники применяются до сих пор и очень широко. Причём разрядники 
		выпускаются огромным ассортиментом на все случаи жизни, от маленьких 
		для защиты линий связи до огромных для зашиты линий электропередач. 
		На рисунке~\infig[phonearresters] изображены разрядники в плате мини-АТС (цилиндрические с 
		торговой маркой производителя EPCOS), для защиты от импульсов высокого 
		напряжения, которые могут оказаться в телефонной линии:

		\placefigure[here][phonearresters]{Цилиндрические элементы на плате -- искровые разрядники, производства компании EPCOS
			}{\externalfigure[phonearresters.jpg][width=\textwidth]}

	\stopsubject

	\startsubject[title={Полупроводники защищают полупроводники}]

		На замену разрядникам в деле защиты линий электропередач и линий связи пришли варисторы. Это особый тип 
		резисторов, сопротивление которых зависит от приложенного напряжения. 
		На рисунке~\infig[VIcurve] изображена Вольт-амперная характеристика, она показывает 
		зависимость тока через прибор и приложенного напряжения.

		\placefigure[here][VIcurve]{Вольт-амперная характеристика (ВАХ) варистора
			}{\externalfigure[VIcurve_MOV.svg][width=\textwidth]}

		То есть они ведут себя примерно как разрядники. Если напряжение ниже 
		порогового -- их сопротивление велико, есть только мизерный ток утечки. 
		Если напряжение превышает пороговое, то варистор довольно сильно меняет 
		своё сопротивление, начиная хорошо проводить ток. В отличие от 
		разрядника, возвращается в исходное состояние с высоким сопротивлением, 
		стоит лишь напряжению опуститься ниже порогового. В итоге напряжение на 
		контактах варистора получается относительно стабильным, повышение 
		напряжения он компенсирует увеличением тока через себя, что не даст 
		напряжению расти.
		
		Чисто технически, варистор представляет собой таблетку спечённой 
		керамики из вещества, которое обладает свойством полупроводника, 
		например, гранул оксида цинка в матрице из смеси оксидов металлов, 
		поэтому его и называют MOV -- Metal Oxide Varistor. Гранулы создают 
		огромное количество p-n переходов, проводящих ток в одном направлении. 
		Но так как их образуется много и в случайном порядке, для выпрямления 
		тока они бесполезны. Свойство устраивать электрический пробóй при 
		превышении определённого напряжения (а электрический пробой p-n перехода обратим) 
		оказалось очень кстати. Регулируя толщину таблетки, можно добиться 
		достаточно стабильного порогового напряжения при производстве. 
		А увеличивая объём шайбы, можно увеличить максимальную энергию импульса, 
		который способен поглотить варистор. 
		
		Варистор получился не идеальным, поэтому он не заменил, а лишь дополнил 
		разрядники. За огромный плюс -- отсутствие разницы между напряжением 
		пробоя и напряжением восстановления, варисторам прощают токи утечки, 
		ограниченный ресурс (после некоторого количества срабатываний может 
		потерять характеристики), достаточно крупные габариты при скромных допустимых 
		энергиях разряда. График напряжения в цепи с варистором на фоне всплесков изображён на рисунке~\infig[varistorwork].

		\placefigure[here][varistorwork]{Принцип работы ограничителя перенапряжения -- варистора
			}{\externalfigure[varistorwork.svg][width=\textwidth]}


		Варистор, от тяжелой работы по поглощению всплесков напряжения может со временем деградировать,
		устроив короткое замыкание. На этот случай необходимо предусмотреть защиту, в виде предохранителя. Большие 
		могучие варисторы на DIN рейку для защиты силовых линий часто содержат 
		в себе дополнительную встроенную защиту, реагирующую на перегрев. На рисунке~\infig[varistorinside] начинка варистора 
		в щиток от IEK.

		\placefigure[here][varistorinside]{Внутренности ограничителя перенапряжений ОПС-1
			от компании IEK.
			}{\externalfigure[varistorinside.jpg][width=\textwidth]}

		
		Видно саму таблетку варистора (синего цвета). К ней присоединены 
		электроды и подпружиненный флажок опирается на электрод, припаянный 
		легкоплавким припоем. Если варистор нагревается свыше разумного (неважно, 
		от пришедшего импульса с молнии, или по причине деградации), то 
		припой плавится, электрод отсоединяется, разрывая цепь, и пружина 
		опускает флажок, показывая неисправность варистора. Если защиты не 
		предусмотреть, неконтролируемый нагрев варистора может устроить пожарчик. 
		
		Варисторы небольших размеров можно встретить во множестве электронных 
		устройств для защиты от случайно пришедших по сети всплесков высокого 
		напряжения. В большинстве удлинителей, именующих себя "сетевыми фильтрами", 
		вся фильтрация сводится к наличию пары варисторов внутри. На рисунке~\infig[pwrcords] можно 
		разглядеть варисторы (синего цвета) в разных удлинителях.

		\placefigure[here][pwrcords]{Неисправные удлинители (именуемые "сетевыми фильтрами").
			}{\externalfigure[pwrcords.jpg][width=\textwidth]}

	\stopsubject

	\startsubject[title={Защита для самых нежных}]
		Помимо варисторов и разрядников есть ещё одни устройства защиты -- полупроводниковые 
		супрессоры (TVS-transient voltage suppressor), они же TVS-диоды, они же 
		полупроводниковые ограничители напряжения. Это специально спроектированные 
		диоды, которые работают на обратной ветви вольт-амперной характеристики 
		(да, той самой, где происходит обратимый электрический пробой у варисторов). 
		Физически они выполняют ту же самую функцию, что и остальные устройства 
		защиты -- не проводят ток, если напряжение в норме, и начинают проводить 
		ток, если напряжение почему-то превысило допустимое значение, тем самым 
		выполняя роль ограничителя.  На фото довольно крупный экземпляр, они 
		бывают совсем миниатюрные:

		\placefigure[here][]{Полупроводниковые супрессоры P6KE18CA двунаправленные,
			напряжение ограничения 25 Вольт. (Классификационное напряжение 18 Вольт, при нём ток утечки составит 1 мА).
			Способен рассеивать \unit{600 watt}, но только в импульсе 10/1000.			
			}{\externalfigure[TVS.jpg][width=\textwidth]}


		Полупроводниковые ограничители напряжения прекрасны почти всем, кроме 
		одного -- величина энергии импульса, который они способны ограничить, 
		поглотив излишки, очень мала. Создание на их базе защиты, способной хоть 
		как-то сравниться по характеристикам с разрядниками или варисторами, будет 
		слишком дорогой. Поэтому они нашли применение там, где нужна компактная 
		защита самой нежной и чувствительной электроники от небольших по мощности 
		всплесков, например, от статического электричества. Будьте уверены -- в 
		вашем телефоне все контакты, что ведут внутрь (USB, наушники) защищены 
		маленькими TVS диодами, которые не позволяют напряжению на этих контактах 
		превысить \unit{5 volt}, даже если вы случайно "щёлкнете" по ним 
		электричеством, снимая свитер.
		
		Этот раздел был добавлен для полноты, в энергетике TVS диоды не 
		применяются, и вы можете встретиться с ними только как с компонентом 
		электронной техники. 
	\stopsubject
	
	\startsubject[title={Концепция зональной защиты}]
		Нетерпеливый читатель захочет спросить: можно ли поставить в электрощиток на вводе в дом универсальное устройство 
		защиты от импульсных перенапряжений, и не знать проблем? К сожалению -- нет. 

		Хотя бы потому, что даже если вы подавили все нежелательные всплески на 
		входе в дом, можно повторно словить их проводкой внутри здания, например, 
		когда ток разряда молнии будет следовать от громоотвода в землю где-то 
		за стенкой -- электромагнитное поле столь мощное, что в любом проводнике 
		наведёт импульс тока. Или, например, что в сеть импульс повторно проникнет 
		через телефонный аппарат, придя по телефонной линии. Поэтому процесс 
		построения защиты усложняется -- нужно анализировать все пути проникновения 
		электромагнитного импульса от молнии внутрь защищаемого объекта.

		Чтобы не оснащать каждое устройство в здании полным комплектом  
		защиты от прямого попадания молнии (это было бы слишком дорого), придумали 
		концепцию зональной защиты.  Объект, 
		электрическая начинка которого защищается от повреждения молнией, 
		разделяется на зоны, согласно степени воздействия молнией. Все линии 
		(силовые, канала связи), переходящие из зоны в зону, на границе зон оснащаются 
		устройствами защиты соответствующего класса. Проще понять это на абстрактном примере дома:
	
		\placefigure[here][lpz]{Пояснение принципа разделения на зоны
			}{\externalfigure[LPZ.svg][width=\textwidth]}


	
		LPZ --- lightning protection zone -- зона защиты от молнии.

		\startitemize
		\item Зона 0а -- это зона, куда непосредственно может ударить молния. 
		В проводнике может оказаться полный ток молнии

		\item Зона 0b -- это зона, куда молния напрямую уже не ударит, но в 
		проводнике может оказаться частичный ток молнии -- как из-за 
		электромагнитного поля, так и просто из-за пробоя изоляции.

		\item Зона 1 -- это зона, где может появиться только наведённый молнией ток.

		\item Зона 2,3,4 и т.д. -- зона, где наведённый молнией ток ослаблен и меньше, 
		чем в вышестоящей зоне. Зон может быть сколь угодно много, как в матрёшке.
		\stopitemize


		То есть, при переходе из зоны в зону, электромагнитный импульс 
		молнии ослабевает, в том числе из-за устройств защиты на границах зон, 
		и за счёт экранирования и ослабления в пространстве. Например, бетонная 
		стенка с заземлённой арматурой внутри может служить таким экраном. 
		Зоны обычно разделяются по естественным препятствиям -- стена, корпус 
		шкафа, корпус прибора и т.д.

		Устройства защиты также разделили на классы.\footnote{Класс -- это 
		параметры испытаний. Если устройство проходит испытания, то оно соответствует
		заявленному классу. Производитель может произвести испытания сразу на соответствие
		нескольким классам, тогда может быть заявлено, что устройство например II+III класса.} 
		Когда понятно деление на зоны, достаточно взять из каталога 
		устройство соответствующего класса.
		
		\startitemize[packed]
		\item Класс I (B) -- это устройства, способные выдержать частичный ток молнии 
		(зона 0), и они предназначены для установки на вводном щите. (где зона 0 переходит в зону 1)

		\item Класс II (C) -- это устройства, способные выдержать меньший ток, чем 
		устройство класса I, но они дешевле и напряжение, до которого они срежут 
		импульс, ниже. Предназначены для установки на распределительном щите. 
		(Как раз, где  зона~1 переходит в зону~2)

		\item Класс III (D) -- это устройства, способные выдержать импульс ещё 
		меньшей величины, чем класс II, но зато срезающие импульс почти 
		полностью. Они предназначены для установки уже на щит конечного потребителя. 
		Многие грамотно спроектированные устройства имеют подобную защиту уже внутри себя.
		\stopitemize

		Почему бы не ставить везде устройства защиты  класса I? А просто потому, 
		что установка устройства класса I там, где с лихвой хватит класса III, 
		например, у конечного потребителя -- неоправданный перерасход бюджета. 
		Это как строить полностью укомплектованную пожарную часть там, где 
		достаточно поставить огнетушитель. Кроме того, чем брутальнее и мощнее 
		устройство защиты, тем больше величина напряжения импульса, который 
		просачивается через неё в потребителя. (тем выше напряжение ограничения, 
		см картинку выше)

		\placefigure[here][arrestercacade]{Пояснение различных типов (классов) УЗИП.
			}{\externalfigure[arrestercacade.svg][width=\textwidth]}

		Но если хочется всё и сразу, существуют комбинированные устройства, 
		например, класс I+II которые соответствуют параметрам сразу нескольких 
		классов, но за такую универсальность производитель попросит дополнительных 
		денег. 
	\stopsubject

	\startsubject[title={Стандартная молния}]
		Каждый удар молнии уникален по своим характеристикам. Но устройства 
		защиты нужно как-то тестировать, сравнивать, разрабатывать, поэтому 
		пришлось договариваться о некоторых характеристиках электромагнитного 
		импульса, который наводит молния. На лицевой панели устройств 
		защиты, а также в документации можно увидеть (поглядите маркировку на 
		распиленном УЗИПе от IEK на фото выше):

		\startitemize[n]
			\item Пиковое значение тока, который проходит через прибор без его повреждения, 
			в тысячах ампер (кА). Например, \unit{50 kilo ampere} -- означает, что 
			пиковый ток в импульсе достигает \unit{50000 ampere}. 
	
			\item Запись о длительности импульса, в микросекундах. Она указывается 
			через дробь. Например, 10/350 означает, что импульс нарастает до 
			максимального значения тока за 10 микросекунд, а потом плавно спадает до 
			нуля за 350~микросекунд. Или, например, 8/20. (10/350 -- длинный и мощный 
			импульс, характерный для прямого попадания разрядом, а 8/20 -- короткий, 
			более характерный наведённому от молнии неподалёку)

			\item Рабочее напряжение. Это нормальное напряжение в линии, к которой 
			подключается защита.

			\item Напряжение ограничения, в вольтах. Это величина остаточного напряжения 
			импульса на клеммах устройства (позже укажу, почему это важно), до 
			которого устройство защиты сможет его уменьшить.

			\item Класс устройства (см. часть про зональную концепцию).

		\stopitemize

		Стоит отметить, что даже многолетняя собранная статистика не исключает, 
		что конкретно вы не согрешили настолько, что по вам ударит аномально 
		мощная молния, но вероятность этого весьма низкая. (Например МЭК 62305-1 
		считает, что даже по самым отъявленным грешникам молнии с зарядом более 
		300~Кл выпускаются менее чем в 1\% случаев.)

		На рисунке~\infig[LPL] наглядно показано распределение количества молний по току
		разряда, и уровни защиты, с указанием зоны покрытия. 

		\placefigure[here][LPL]{Пояснение уровней защиты LPL по ГОСТ Р МЭК 62305-1-2010. Иллюстрация по мотивам 
			руководства Lightning protection guide от компании OBO BETTERMANN.
			}{\externalfigure[LPL.svg][width=\textwidth]}


		Так как процесс предсказания тока у молнии, которая ударит в объект в 
		будущем сродни процессу предсказания курса биткоина (то есть гадание), 
		то были введены разные уровни защит от молний. Картинка~\infig[LPL]  наглядно 
		показывает, как они соотносятся. Необходимый уровень защиты выбирается 
		согласно оценке рисков ущерба от попадания молнии. Очевидно, что если 
		для защиты оборудования мачты связи экономически целесообразно сделать
		защиту хотя бы от 99\% случаев попаданий молнии (то есть, прорвётся одна молния из сотни),
		то для газоперекачивающей станции может быть мало и 99,9\% защиты (прорвётся одна из тысячи).

	\stopsubject

	\startsubject[title={Портим всё, забыв про мелочи}]
		Описанное выше актуально для сферического коня в вакууме. В реальной 
		жизни есть огромное количество тонкостей, которые опускаются для 
		упрощения, но рано или поздно дадут о себе знать. Вот примеры некоторых из них:
		
		{\it 1. Собственная индуктивность и сопротивление проводников. }
		
		Отрезок провода длиной 1 метр обладает индуктивностью примерно 1 мкГн и 
		ненулевым сопротивлением.\footnote{ГОСТ Р МЭК 61643-12--2011 п6.1.3} 
		А значит, при высоких темпах нарастания тока 
		(а для молний они как раз характерны \footnote{При скорости подъёма тока 
		в импульсе \unit{1 kilo ampere per microsecond} индуктивность одного метра 
		проводника даст падение напряжения примерно \unit{1 kilovolt}}) лишний запас провода может свести 
		смысл защиты к нулю. Многие производители в своих руководствах явно 
		указывают, что длина проводников от линии к клеммам устройства защиты 
		должны быть максимально короткой, и в сумме не превышать \unit{0,5 meter}. 
		На рисунке~\infig[lengthaffect] показано, как лишние 2 метра проводника
		влияют на конечное напряжение ограничения. Если УЗИП  срезает пришедший 
		импульс до величины \unit{1,5 kilo volt}, то на проводниках падает 
		дополнительно \unit{2 kilo volt}, и в итоге в нагрузку придёт импульс 
		напряжением \unit{3,5 kilo volt}.
		
		Весьма изящным способом для уменьшения влияния проводников является 
		 V-образное подключение, что изображено на рисунке~\infig[lengthaffect].

		\placefigure[here][lengthaffect]{Влияние длины проводников при подключении УЗИП на степень ограничения перенапряжений
			}{\externalfigure[lengthaffect.svg][height=0.6\textheight]}

		
		Некоторые производители для удобства монтажа  предусматривают 
		двойные клеммы, например, как на устройстве с  рисунка~\infig[uzp-220] (отечественное кстати).

		\placefigure[here][uzp-220]{Устройство защиты УЗП-220 производства компании "Тахион".
			иллюстрация взята с сайта производителя \hyperlink{https://tahion.spb.ru}
			}{\externalfigure[uzp-220.png][width=\textwidth]}


		
		{ \it 2. Сопротивление играет роль}
		
		При токе разряда молнии в \unit{50 kilo ampere}, на проводнике с 
		сопротивлением в \unit{0,1 Ohm} при протекании тока создастся разница 
		напряжения в \unit{5 kilo volt}. Поэтому УЗИП следует подключать 
		максимально толстым проводником, не менее \unit{6 square millimeter}, 
		даже если сама по себе линия 2,5 или даже \unit{1,5 square millimeter}. 
		Если вы подключили УЗИП V-образно как на фото выше, то толстым у вас 
		останется только заземляющий проводник.
		
		{ \it 3. Устройства защиты без согласования бесполезно соединять параллельно}
		
		Может закрасться мысль, что если параллельно поставить несколько 
		устройств защиты, то мы получим мегазащиту. Но это так не работает. 
		Когда по линии прилетит импульс -- то первым сработает кто-то один, и 
		примет на себя весь удар. Чтобы каскад из защит работал согласованно, и 
		по мере необходимости в дело поглощения импульса подключались все более 
		и более мощные устройства, они должны согласоваться специальными дросселями. 
		Но так как расчет такого каскада задача непростая, то и устройства 
		согласования в каталогах производителей УЗИП найти крайне трудно. 
		Производитель  стал выпускать комбинированные устройства, согласуя их 
		внутри сам. То есть, вместо установки рядом УЗИП II и УЗИП III класса 
		нужно взять готовое устройство II+III класса.
		
		{ \it 4. Ставим автомат вместо предохранителя}
		
		Если вы внимательно прочитаете документацию на устройства защиты от 
		импульсных перенапряжений, то увидите, что многие производители требуют установку 
		предохранителей для защиты от короткого замыкания -- если устройство 
		выйдет из строя, оно может устроить короткое замыкание защищаемой линии 
		на землю. И при таком сценарии лучше, если сгорит предохранитель и 
		отключит устройство защиты от линии, чем это сделает вводной автомат, 
		обесточив нагрузку. Но см. п.1 -- глупо сначала добиваться минимальной 
		индуктивности проводников, чтобы затем воткнуть автоматический 
		выключатель, внутри которого  электромагнитный расцепитель в виде 
		катушки индуктивности. В итоге автоматический выключатель будет 
		работать как дополнительные виртуальные несколько метров провода (см п.1), 
		увеличивая напряжение импульса, дошедшего в нагрузку. И именно поэтому 
		крайне желательно использовать именно предохранители. (Стоит ещё принять во внимание
		опасность, что импульс тока в 10-50-\unit{100 kilo ampere} 
		вызовет спекание контактов в автомате)
		
		{ \it 5. УЗИП на базе варисторов имеют ток утечки}
		
		Он небольшой, но при этом не нулевой. И тут здравый смысл отходит на 
		второй план перед электросетевой компанией, которая имеет своё мнение 
		на то, где должно быть установлено УЗИП. Так что может получиться так, 
		что УЗИП вы поставите после счётчика. Но так как счётчик -- собственность 
		электросетевой компании, можете делать кулфейс, когда он сгорит после грозы,  
		 и вам придут его менять.
		
		{ \it 6. Отсутствие контроля}
		
		Представьте, что вы оснастили УЗИПами электрощит, который питает 
		метеостанцию в безлюдном месте. Рядом прошла гроза, УЗИПы выполнили свою 
		функцию, спасли начинку станции от повреждения, но погибли сами -- их 
		отключила защита. И получается ситуация, когда станция нормально работает, 
		но при этом не имеет защиты, и следующая гроза может вывести её из строя. 
		Именно от таких неприятных ситуаций существуют УЗИП с контактами, 
		которые размыкаются/замыкаются, когда защита выходит из строя (например, 
		на фото~\infig[uzp-220] УЗП-220 это контакты 4 и 5). В таком случае умерший УЗИП может 
		подать сигнал в систему диспетчеризации, что пора высылать монтажника 
		для замены защиты.

	\stopsubject

	\startsubject[title={Практика.}]
		Дочитавший до этого места наверняка уже задался вопросом -- а зачем мне 
		надо УЗИП и как его включать? Переходим к конкретике.
		
		Если вы живете в частном доме и электричество в дом поступает по 
		воздушной линии электропередач, то вам требуется УЗИП, причём класса I. 
		В некоторых случаях может хватить и II класса, но тут уже  очень много "но".
		Если вы живете в многоквартирном доме, все инженерные системы которого в 
		порядке, то УЗИП  не является устройством первой необходимости, но 
		хуже не станет, если в щит вы добавите УЗИП класса II. Типовая схема 
		использования УЗИПов изображена на рисунке~\infig[arresteruse].

		\placefigure[here][arresteruse]{Пример применения УЗИП
			}{\externalfigure[arresteruse.svg][width=\textwidth]}


		Ввод слева. УЗИПы класса I располагаются сразу после вводного автомата 
		(ну или после электросчётчика, если электросетевая компания так желает) по 
		одному на каждую фазу.  Видно повторное заземление, и TN-C превращается 
		в TN-C-S.  Без заземления УЗИП не работает -- куда ему отводить энергию 
		импульса, кроме как в землю?
		
		Внутри здания на промежуточном щите, например, этажном, используются 
		УЗИП класса II, которые подавят то, что смогло пройти через УЗИПы на 
		вводе. Обратите внимание -- между N и PE стоит УЗИП, специально для 
		этого предназначенный, так как в норме напряжение между N и PE невелико.
		
		Ну и, наконец, рядом с потребителем ставится УЗИП класс III. У хорошо 
		спроектированных устройств, внутри уже есть защита от перенапряжений, которая предусмотрена производителем.
	\stopsubject

\stopchapter

\startchapter[title={Осторожно, подделки!}]


	Увы, устройства защиты для применения в быту и промышленности тоже 
	подделывают. Причём подделывают всё, не только дорогие импортные изделия, 
	но и даже самые бюджетные, про которые подумаешь «ну вот их-то подделывать 
	смысла нет». Себестоимость подделки должна быть ниже оригинала, и жулики 
	создают изделие путём упрощения, которое внешне похоже на оригинал, и как-то 
	работает, но не удовлетворяет требованиям стандартов. К счастью, у меня 
	есть пример, на рисунке~\infig[fake] оригинальный выключатель и поддельный.

	\placefigure[here][fake]{Слева оригинальный автоматический выключатель, справа поддельный.
			}{\externalfigure[fake.jpg][width=\textwidth]}

	Что в подделке не так?

	\startitemize [n]

		\item Вместо биметаллической пластинки теплового расцепителя просто кусок 
		жести. Автоматический выключатель не сработает ни при двукратном, ни при пятикратном 
		превышении номинального тока. Заодно выкинули регулировочный винт, 
		которым на заводе подстраивают тепловой расцепитель для попадания в допуск -- 
		в подделке нет расцепителя, настраивать нечего.

		\item Выкинули половину пластин из дугогасящей камеры, сами пластины тоньше. 
		Ни о какой отключающей способности в \unit{4,5 kilo ampere} речь не идёт.

		\item Выбросили теплоизолирующий вкладыш от прогара вбок.

		\item Электромагнитный расцепитель условный, вместо 5,5 витка в оригинале, 
		в подделке сделали 3,5 витка проводом потоньше. Это сделано, чтобы при 
		коротком замыкании он хоть как-то отключился (вдруг кто решит кустарно 
		проверить), но ни о каком попадании в допуск речи нет, хорошо, если он 
		сработает при $30 \times I\low{ном}.$

		\item Разница в материалах, даже на фото видно отличие в гальванических 
		покрытиях клемм, у подделки покрытие попроще, лишь бы не заржавело до 
		продажи, ни о каком сроке службы речь не идёт. Собственно, подделка и 
		сгорела раньше времени из-за нагрева в клемме или контактной паре. Сама 
		контактная пара не имеет ни покрытий, ни специальных напаек для низкого 
		переходного сопротивления.

	\stopitemize 

	Даже не нужно нагонять жути, рассказывая о последствиях установки таких 
	«изделий» -- они не выполняют свою основную функцию, и сами могут 
	стать источником проблем.

	Как отличить поддельное изделие от оригинала? Можно, конечно, написать 
	целую инструкцию по разглядыванию незначительных отличий во внешнем 
	виде, но это всё не имеет смысла. Подделки меняются, у одной партии 
	кривая заклёпка, а  другую выдаёт лишняя точка в маркировке. 
	Поэтому рекомендация простая -- не покупайте устройства где попало. 
	В хозяйственных магазинах, в лотках на рынке, в ларьке на остановке -- 
	в таких местах всегда есть шанс нарваться на подделку. Определитесь с 
	производителем, посмотрите на его  сайте, кто и где является его официальным 
	дилером, и покупайте там. В таком случае цепочка поставки понятна и 
	крупный торговец, потративший время на получение заветного статуса, не 
	сильно заинтересован в продаже подделок.\footnote{Хотя мы помним классику:
	капитал боится отсутствия прибыли или слишком маленькой прибыли, как 
	природа боится пустоты. Но раз имеется в наличии достаточная прибыль, 
	капитал становится смелым. Обеспечьте 10 процентов, и капитал согласен 
	на всякое применение, при 20 процентах он становится оживлённым, при 50 
	процентах положительно готов сломать себе голову, при 100 процентах он 
	попирает все человеческие законы, при 300 процентах нет такого преступления, 
	на которое он не рискнул бы, хотя бы под страхом виселицы.---Томас Джозеф Даннинг}


\stopchapter

\startchapter[title={Дополнительное чтение.}]

	Для расширения и углубления своих знаний рекомендую следующие источники, 
	помимо нормативных документов:

	К главе про предохранители:

	\startitemize [n, packed]
	\item Книга К.К. Намитоков, Р.С. Хмельницкий, К.Н.Аникеева. Плавкие 
	предохранители. М.: Энергия 1979.

	\item ГОСТ Р 50339.0-2003 (МЭК 60269-1-98) Предохранители плавкие низковольтные. 
	Часть 1. Общие требования
	\stopitemize

	На английском языке:
	\startitemize [n, packed]
	\item Замечательная статья Артура Стила в журнале Electronic World 1965 
	года про правильный подбор предохранителей: 
    \hyperlink{https://www.rfcafe.com/references/electronics-world/selecting-proper-fuse-august-1965-electronics-world.htm}

	\item Простое и доступное руководство по предохранителям "fuseology", там есть 
	всё и по расчётам, и по устройству предохранителей. 
	\hyperlink{http://www.cooperindustries.com/content/dam/public/bussmann/Electrical/Resources/technical-literature/bus-ele-br-10757-fuseology.pdf}
	\stopitemize

	К главе про выключатели дифференциального тока:	
	\startitemize [n, packed]
	\item В.К. Монаков УЗО. Теория и практика Москва, Издательство "Энергосервис", 2007 г.

	Книжка шикарная в своей полноте и довольно простом языке изложения. 
	Автор -- директор компании АСТРО-УЗО (uzo.ru) - отечественного разработчика и производителя УЗО. 

	\item \hyperlink{http://www.uzo.ru/books/normative-document/} Выжимка нормативных 
	документов, имеющих отношение к УЗО. Там же есть ещё один документ 
	заслуживающий внимания \hyperlink{http://www.uzo.ru/books/uzo.pdf}
	\stopitemize

	К главе про устройства защиты от импульсных перенапряжений:
	\startitemize [n, packed]
	\item Прежде всего нормативная документация. Говорим: Окей, гугл, 
	"Устройство молниезащиты зданий, сооружений и промышленных коммуникаций: 
	Сборник документов. Серия 17. Выпуск 27" и внимательно изучаем, в сборнике 
	собраны нормативные документы: Инструкция по устройству молниезащиты зданий 
	и сооружений (РД 34.21.122-87) и Инструкция по устройству молниезащиты 
	зданий, сооружений и промышленных коммуникаций (СО 153-34.21.122-2003), а 
	также отдельно гуглим и смотрим ГОСТ Р МЭК 62305. Он состоит из большого 
	количества частей, но ни один блогер в интернете не может быть выше 
	нормативных требований.

	\item Есть прекрасный сайт \hyperlink{https://zandz.com}  Ребята не только записали вебинары 
	с приглашёнными специалистами сферы, но и сделали их стенограммы, так что 
	можно быстро прочитать вместо просмотра видео. Всё это великолепие они 
	выложили бесплатно, но потребуется регистрация. За что им почёт и уважение. Видеозаписи 
	вебинаров лежат у них на ютуб канале и доступны без регистрации, например, 
	вебинары проф. Базеляна (\hyperlink{https://www.youtube.com/watch?v=R-KbjRb4Yuw}

	\item Неплохая статья на хабрахабре \hyperlink{https://habr.com/ru/post/188972/}

	\item Многие производители выпускают руководства по проектированию -- такая 
	завуалированная реклама, где простым языком объясняются основы и заодно 
	приводятся выдержки из каталога оборудования, которое решает проблему. 
	На русском языке есть прекрасное руководство от шнайдер электрик 
    \hyperlink{https://www.se.com/ru/ru/download/document/MKP-CAT-ELGUIDE-19/}, нас 
	интересует раздел J, посвящённый защите от перенапряжений. В нём всё 
	довольно просто, наглядно и точно.

	\item Если вы владеете английским языком, то фирмы, производящие всё для 
	молниезащиты, выпустили замечательные руководства. Конечно с перекосом в 
	свою продукцию, но как видите, некоторые иллюстрации я позаимствовал у них.
	Это OBO BETTERMANN lightning protection guide \hyperlink{https://www.obo.hu/out/media/Blitzschutz-Leitfaden_en.pdf},
	  Dehn lightning protection guide 
	\hyperlink{https://www.dehn-international.com/sites/default/files/media/files/lpg-2015-e-complete.pdf}
	\stopitemize
	
    Отлично дополняют эту книгу:
	
	\placefigure[left][perebaskin]{Обложка книги А.В. Перебаскин. Реальная помощь домашнему электрику.
			}{\externalfigure[perebaskin.svg][width=0.4\textwidth]}
			
	Книга "Влезай, не убъёт" А. В. Перебаскина. В этой книге рассматриваются 
	вопросы составления проекта, осуществления электромонтажных работ,
	заземления -- все те вопросы, которые не вошли в мою книгу. Я бы сказал эта 
	книга -- идеальное дополнение, пересечение и повторения минимальны. 
	Автор пишет живым языком и понимает, о чем пишет, в отличии от некоторых блогеров в интернете.

	
	Также, абсолютно бесплатно,  расширить, углубить свои знания можно на учебном портале IEK Academy \goto{\hyphenatedurl{https://academy.iek.group/}}[url(https://academy.iek.group/?utm_source=ebook&utm_medium=organic&utm_campaign=serkov)], где 
	учебные материалы сгруппированы по темам и предусмотрены контрольные задания. 
	Формат учебных курсов как текстовый, так и в виде видеороликов, на любой вкус.
	Конечно, в практической части рассматривается продукция IEK (а также компаний, входящих в холдинг -- ONI, ITK и другие.), но теоретические 
	вопросы универсальны для всех производителей. И, что немаловажно, на портале
	IEK Academy можно задать технический вопрос и получить на него компетентный (!) ответ.
	
	\placefigure[here][iekacademy]{Заглавная страница портала IEK Academy \goto{\hyphenatedurl{https://academy.iek.group/}}[url(https://academy.iek.group/?utm_source=ebook&utm_medium=organic&utm_campaign=serkov)]
			}{\externalfigure[iek_academy.jpg][width=\textwidth]}
	
%https://academy.iek.group/?utm_source=ebook&utm_medium=organic&utm_campaign=serkov	

\stopchapter

\startchapter[title={Заключение.}]
	Если, прочитав книгу, вы не узнали для себя ничего нового -- то я, как автор, 
	буду рад, что меня окружают столь квалифицированные специалисты. Если вы 
	 узнали для себя что-то новое -- то я буду рад, 
	что сделал мир лучше и поделился знаниями. В любом случае автор рад, 
	когда его труд читают. Поэтому дайте почитать бумажную книгу друзьям, 
	пришлите ссылку на электронную версию коллегам, способствуйте максимальному 
	распространению знаний.
	
	Если вам понравилось моё творчество -- можете подписаться на меня в удобных вам соцсетях,
	список которых есть на моем сайте \goto{https://serkov.su}[url(https://serkov.su/blog/?utm_source=ebook&utm_medium=organic&utm_campaign=serkov)]. Так точно не упустите, когда я опубликую что-то новенькое
	в области техпросвета. Также на моем сайте есть форма, через которую звонкой
	монетой можно отблагодарить автора и поучаствовать в будущих творческих проектах.

	Книга создана полностью с использованием только свободного программного обеспечения: 
	операционная система Linux, пакеты для работы с графикой GIMP и inkscape, офисный пакет 
	LibreOffice. Верстка осуществлена в системе ConTeXt. Рисунки и фотографии -- автора.
	Вариант обложки был выбран читателями в голосовании в телеграм-канале автора.
\stopchapter
\startTEXpage
    \externalfigure[Book_backside.svg][width=\paperwidth, height=\paperheight]
 \stopTEXpage
\stoptext 






%Чтобы отключить верхние и нижние колонтитулы на определенной странице, 
%используйте команду\noheaderandfooterlines, которая действует исключительно 
%на странице, на которой она расположена. Если мы хотим удалить только номер 
%страницы на определенной странице, мы должныиспользовать команду\page[blank].










%Level Numbered sections Unnumbered sections
%1 		\part
%2		\chapter			\title
%3		\section			\subject
%4		\subsection			\subsubject
%5		\subsubsection		\subsubsubject
%6		\subsubsubsection	\subsubsubsubject

%\footnote

%Общий инструмент для написания списков в ConTEXt - это среда\itemize, синтаксис которойследующий:\startitemize[Options][Configuration] ... \stopitemize


%\startnarrower[..,...,..]... \stopnarrower...n*leftn*middlen*right
%\note[ref]

%$математика$


